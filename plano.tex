
\documentclass[12pt]{article}

\usepackage[a4paper,margin=2.5cm]{geometry}
\usepackage[T1]{fontenc}
\usepackage[utf8]{inputenc}
\usepackage[brazil]{babel}
\usepackage{lmodern}
\usepackage{microtype}
\usepackage{amsmath, amssymb}
\usepackage{booktabs}
\usepackage{array}
\usepackage{xcolor}
\usepackage{hyperref}
\usepackage{enumitem}
\usepackage[most]{tcolorbox}

\hypersetup{colorlinks=true,linkcolor=blue,urlcolor=blue,citecolor=teal}

% ---------- Colors (pastel palette) ----------
\definecolor{pastelblue}{HTML}{E6F0FF}
\definecolor{pastelgray}{HTML}{F2F2F2}
\definecolor{pastelorange}{HTML}{FFEBD6}
\definecolor{pastelteal}{HTML}{DFF7F2}

% ---------- tcolorbox styles (compatível com o livro) ----------
\tcbset{sharp corners, boxrule=0pt, coltitle=black, fonttitle=\bfseries, left=8pt, right=8pt, top=6pt, bottom=6pt}

\newtcolorbox{NoteBox}[1][]{colback=pastelteal, coltitle=white, title=Nota, #1}
\newtcolorbox{SolvedBox}[1][]{colback=pastelorange, coltitle=white, title=Entrega (Checklist), #1}

\title{\Huge \textbf{Plano de Ensino}\\ \Large Análise Exploratória de Dados e Storytelling (2026.1)}
\author{Sistemas de Informação -- UNIFACOL}
\date{Fevereiro -- Maio/Junho de 2026}

\begin{document}
\maketitle

\section{Visão Geral}
Esta disciplina tem dois objetivos complementares:
\begin{itemize}[leftmargin=*]
	\item \textbf{Unidade I (semanas 01--07):} construir a base de \textbf{AED} com \textbf{estatística prática} e \textbf{Python (Pandas)} para limpar, resumir e interpretar dados;
	\item \textbf{Unidade II (semanas 08--12):} transformar insights em \textbf{comunicação persuasiva} com \textbf{Tableau} e \textbf{storytelling}.
\end{itemize}

\begin{NoteBox}
\textbf{Regra do ritmo (aula noturna 19h--22h):} em semanas com trabalho, as \textbf{apresentações ocupam no máximo 1 hora}. O restante da aula continua com teoria + prática guiada.
\end{NoteBox}

\section{Calendário Letivo (12 semanas de aula)}
O calendário abaixo está alinhado ao arquivo \texttt{aed/calendario.md}. Semanas marcadas como \textit{não conta} (feriados/provas) não entram na contagem das 12 semanas.

\begin{center}
\renewcommand{\arraystretch}{1.15}
\begin{tabular}{@{}c p{3.2cm} p{4.2cm} p{5.7cm}@{}}
\toprule
\textbf{Semana} & \textbf{Período (Seg--Qua)} & \textbf{Tipo da semana} & \textbf{Observação} \\
\midrule
01 & 02/02 a 04/02 & Teoria + lab & Início das aulas \\
02 & 09/02 a 11/02 & Teoria + \textbf{Trabalho 1} & Entrega quinzenal \\
\midrule
\multicolumn{4}{@{}l}{\textit{14 a 18/02 --- Carnaval [F] (não conta semana)}}\\
\midrule
03 & 23/02 a 25/02 & Teoria + lab & Semana normal \\
04 & 02/03 a 04/03 & Teoria + \textbf{Trabalho 2} & Entrega quinzenal \\
05 & 09/03 a 11/03 & Teoria + lab & Semana normal \\
06 & 16/03 a 18/03 & Teoria + \textbf{Trabalho 3} & Entrega quinzenal \\
07 & 23/03 a 25/03 & Teoria + lab & Semana normal \\
\midrule
\multicolumn{4}{@{}l}{\textit{02 a 04/04 --- Semana Santa [F] (não conta semana)}}\\
\multicolumn{4}{@{}l}{\textit{06 a 14/04 --- Prova I Unidade (não conta semana)}}\\
\midrule
08 & 20/04 & Teoria + \textbf{Trabalho 4} & Aula apenas na segunda (21/04 é feriado) \\
09 & 27/04 a 29/04 & Teoria + lab & Semana normal \\
10 & 04/05 a 06/05 & Teoria + \textbf{Trabalho 5} & 06/05 é feriado (Vitória) \\
11 & 11/05 a 13/05 & Teoria + lab & Semana normal \\
12 & 18/05 a 20/05 & Teoria + \textbf{Trabalho 6} & Entrega quinzenal (encerramento do conteúdo) \\
\midrule
\multicolumn{4}{@{}l}{\textit{25/05 a 01/06 --- Prova II Unidade (não conta semana)}}\\
\bottomrule
\end{tabular}
\end{center}

\section{Avaliação (visão operacional)}
\begin{itemize}[leftmargin=*]
	\item \textbf{Trabalhos quinzenais (T1--T6):} entregas práticas curtas, com apresentação e checklist.
	\item \textbf{Prova I Unidade:} janela 06/04 a 14/04 (conforme calendário institucional).
	\item \textbf{Prova II Unidade:} janela 25/05 a 01/06.
\end{itemize}

\begin{NoteBox}
\textbf{Formato padrão de entrega (em todos os trabalhos):}
\begin{itemize}[leftmargin=*]
	\item \textbf{Python}: 1 notebook com ETL + checagens (tipos, ausências, duplicadas, filtros).
	\item \textbf{Dados}: 1 CSV ``limpo'' pronto para visualização.
	\item \textbf{Tableau}: 1 workbook (ou export) com visuais/dash.
	\item \textbf{Comunicação}: 1 slide (ou texto curto) com \textbf{Big Idea} + 3 bullets de evidências.
\end{itemize}
\end{NoteBox}

\section{Plano Semana a Semana (conteúdo + aprendizagem + entregas)}

\subsection*{Semana 01 (02/02--04/02) --- Do Caos à Clareza: AED e a \textit{Big Idea}}
\textbf{O que vai ter na semana:}
\begin{itemize}[leftmargin=*]
	\item Abertura com \textbf{gancho} e \textbf{Big Idea} do semestre: ``dados brutos são ruído; valor é extrair uma história que force uma decisão''.
	\item Diferença entre \textbf{análise exploratória} (cozinha) e \textbf{análise explanativa} (jantar).
	\item Primeiro \textbf{ETL com Python/Pandas}: leitura, inspeção, tipos, ausências, conversão de datas, remoção de colunas e exportação de CSV limpo.
	\item Primeiro contato com \textbf{Tableau}: conectar CSV e gerar visuais rápidos.
	\item Design básico: \textbf{Teste do Relance (3 segundos)} e remoção de ``lixo visual''.
\end{itemize}

\textbf{O aluno aprende a:}
\begin{itemize}[leftmargin=*]
	\item Diferenciar \textbf{descobrir} insight (AED) de \textbf{comunicar} insight (storytelling).
	\item Rodar um ETL simples e produzir um dataset ``pronto para viz''.
	\item Construir o primeiro visual no Tableau e avaliar clareza pelo Teste do Relance.
\end{itemize}

\begin{SolvedBox}
\textbf{Preparação para o Trabalho 1 (entrega na Semana 02):}
\begin{itemize}[leftmargin=*]
	\item Escolher um dataset simples (Kaggle ou base sugerida pelo professor).
	\item Definir 1 pergunta de negócio e 1 Big Idea.
\end{itemize}
\end{SolvedBox}

\subsection*{Semana 02 (09/02--11/02) --- Trabalho 1 (apresentações) + Fundamentos de dados retangulares}
\textbf{Formato da aula:}
\begin{itemize}[leftmargin=*]
	\item \textbf{Até 1 hora}: apresentações relâmpago do T1 (2--3 min por aluno/grupo).
	\item \textbf{Restante}: teoria + prática guiada.
\end{itemize}

\textbf{Teoria/Prática da semana:}
\begin{itemize}[leftmargin=*]
	\item \textbf{Dados retangulares}: linhas x colunas, variável x registro.
	\item \textbf{Dicionário de dados} e perguntas: ``o que cada coluna significa?''.
	\item Qualidade básica: ausências, duplicadas, categorias sujas, datas e textos.
\end{itemize}

\textbf{O aluno aprende a:}
\begin{itemize}[leftmargin=*]
	\item Formular uma pergunta e traduzir em colunas/transformações.
	\item Entregar uma história curta com evidência (Big Idea + visuais).
\end{itemize}

\begin{SolvedBox}
\textbf{Trabalho 1 (entrega na Semana 02):} ETL básico + \textbf{3 visuais} no Tableau + Big Idea (1 slide).
\end{SolvedBox}

\subsection*{Semana 03 (23/02--25/02) --- Estatística descritiva I: medidas de posição}
\textbf{O que vai ter na semana:}
\begin{itemize}[leftmargin=*]
	\item \textbf{Média, mediana, moda} e quando usar cada uma.
	\item \textbf{Resumo com Pandas}: \texttt{describe}, \texttt{value\_counts}, \texttt{groupby}.
	\item Leitura crítica: por que ``média'' pode mentir (assimetria/outliers).
\end{itemize}

\textbf{O aluno aprende a:}
\begin{itemize}[leftmargin=*]
	\item Resumir uma coluna numérica e uma coluna categórica.
	\item Explicar em linguagem de negócio o que um resumo indica.
\end{itemize}

\subsection*{Semana 04 (02/03--04/03) --- Trabalho 2 (apresentações) + Estatística descritiva II: variabilidade}
\textbf{Formato da aula:}
\begin{itemize}[leftmargin=*]
	\item \textbf{Até 1 hora}: apresentações do T2.
	\item \textbf{Restante}: teoria + prática (boxplot, IQR, dispersão).
\end{itemize}

\textbf{Teoria/Prática da semana:}
\begin{itemize}[leftmargin=*]
	\item \textbf{Variância, desvio-padrão, amplitude, IQR} e leitura prática.
	\item \textbf{Outliers}: por que aparecem e como investigar antes de remover.
\end{itemize}

\textbf{O aluno aprende a:}
\begin{itemize}[leftmargin=*]
	\item Comparar dois grupos (ex.: por categoria) usando medidas de posição e dispersão.
	\item Justificar escolhas (mediana vs média; IQR vs DP) com base no dado.
\end{itemize}

\begin{SolvedBox}
\textbf{Trabalho 2 (entrega na Semana 04):} mini-relatório AED (posição + dispersão) + 2 visuais (boxplot + histograma) + Big Idea.
\end{SolvedBox}

\subsection*{Semana 05 (09/03--11/03) --- Distribuições e dados ausentes: o ``formato'' dos dados}
\textbf{O que vai ter na semana:}
\begin{itemize}[leftmargin=*]
	\item \textbf{Histogramas} e leitura de forma (simetria/assimetria; caudas).
	\item \textbf{Valores ausentes}: tipos de missing e estratégias (imputar, remover, sinalizar).
	\item \textbf{Regras práticas} para preparar dados para o Tableau (tipos, datas, categorias).
\end{itemize}

\textbf{O aluno aprende a:}
\begin{itemize}[leftmargin=*]
	\item Identificar distribuições problemáticas e propor transformações simples.
	\item Tratar missing sem ``matar'' o significado do dado.
\end{itemize}

\subsection*{Semana 06 (16/03--18/03) --- Trabalho 3 (apresentações) + Correlação introdutória}
\textbf{Formato da aula:}
\begin{itemize}[leftmargin=*]
	\item \textbf{Até 1 hora}: apresentações do T3.
	\item \textbf{Restante}: teoria + prática (correlação e gráficos bivariados).
\end{itemize}

\textbf{Teoria/Prática da semana:}
\begin{itemize}[leftmargin=*]
	\item \textbf{Dispersão (scatter)} e o que \textbf{correlação} mede (e o que não mede).
	\item \textbf{Segmentação} por categoria (cor por classe, facetas simples).
\end{itemize}

\textbf{O aluno aprende a:}
\begin{itemize}[leftmargin=*]
	\item Investigar relações entre duas variáveis sem cair em conclusões causais.
	\item Contar uma história com evidência bivariada (2 visuais coerentes).
\end{itemize}

\begin{SolvedBox}
\textbf{Trabalho 3 (entrega na Semana 06):} diagnóstico (missing + outliers + distribuição) + 1 relação bivariada (scatter) + Big Idea.
\end{SolvedBox}

\subsection*{Semana 07 (23/03--25/03) --- Variáveis categóricas e revisão para Prova I}
\textbf{O que vai ter na semana:}
\begin{itemize}[leftmargin=*]
	\item \textbf{Categóricas}: proporções, tabelas de frequência e gráficos de barras.
	\item \textbf{Boas práticas} de escolha de gráfico (evitar pizza quando não faz sentido).
	\item Revisão guiada dos conceitos da Unidade I (ETL, resumos, distribuições, correlação).
\end{itemize}

\textbf{O aluno aprende a:}
\begin{itemize}[leftmargin=*]
	\item Comparar categorias com clareza e consistência visual.
	\item Consolidar um ``kit'' de AED para a Prova I.
\end{itemize}

\subsection*{Semana 08 (20/04) --- Trabalho 4 (apresentações) + Storytelling: da análise à decisão}
\textbf{Observação:} aula concentrada (apenas segunda-feira).

\textbf{Formato da aula:}
\begin{itemize}[leftmargin=*]
	\item \textbf{Até 1 hora}: apresentações do T4.
	\item \textbf{Restante}: storytelling aplicado (estrutura e mensagem).
\end{itemize}

\textbf{Teoria/Prática da semana:}
\begin{itemize}[leftmargin=*]
	\item \textbf{AED vs explanativa} (retomada): o que entra e o que não entra numa apresentação.
	\item \textbf{Big Idea} e \textbf{pergunta de decisão}: ``o que eu quero que a audiência faça?''
	\item Primeiro \textbf{storyboard} (3 a 5 telas) para um dashboard.
\end{itemize}

\textbf{O aluno aprende a:}
\begin{itemize}[leftmargin=*]
	\item Transformar achados em uma linha narrativa curta.
	\item Planejar um dashboard antes de construir.
\end{itemize}

\begin{SolvedBox}
\textbf{Trabalho 4 (entrega na Semana 08):} roteiro curto (Big Idea + 3 evidências) + esboço de dashboard (wireframe) + 1 visual no Tableau.
\end{SolvedBox}

\subsection*{Semana 09 (27/04--29/04) --- Design de informação: clareza, hierarquia e consistência}
\textbf{O que vai ter na semana:}
\begin{itemize}[leftmargin=*]
	\item \textbf{Teste do Relance} em profundidade: título que diz a mensagem, não o tipo de gráfico.
	\item \textbf{Hierarquia visual}: cor, tamanho, espaço em branco, alinhamento.
	\item Padrões práticos para dashboards: filtros, contexto, explicações curtas.
\end{itemize}

\textbf{O aluno aprende a:}
\begin{itemize}[leftmargin=*]
	\item Melhorar um visual ``ruim'' com pequenas decisões de design.
	\item Escrever títulos orientados a insight (mensagem).
\end{itemize}

\subsection*{Semana 10 (04/05--06/05) --- Trabalho 5 (apresentações) + Tableau interativo}
\textbf{Observação:} 06/05 é feriado; ajustar a prática para caber em 2 encontros.

\textbf{Formato da aula:}
\begin{itemize}[leftmargin=*]
	\item \textbf{Até 1 hora}: apresentações do T5.
	\item \textbf{Restante}: Tableau avançado para interatividade.
\end{itemize}

\textbf{Teoria/Prática da semana:}
\begin{itemize}[leftmargin=*]
	\item Filtros, ações, parâmetros e interação guiada.
	\item Publico-alvo: o que o usuário precisa explorar sozinho.
\end{itemize}

\textbf{O aluno aprende a:}
\begin{itemize}[leftmargin=*]
	\item Criar pelo menos 1 mecanismo de exploração (filtro/parâmetro/ação).
	\item Preservar clareza mesmo com interatividade.
\end{itemize}

\begin{SolvedBox}
\textbf{Trabalho 5 (entrega na Semana 10):} dashboard com 1 a 2 mecanismos de interatividade + título-mensagem + checklist do Teste do Relance.
\end{SolvedBox}

\subsection*{Semana 11 (11/05--13/05) --- Dashboards que ``forçam'' decisão}
\textbf{O que vai ter na semana:}
\begin{itemize}[leftmargin=*]
	\item Estrutura de dashboard: contexto, diagnóstico, recomendação.
	\item Consistência de métricas e definições (evitar ``cada um calcula de um jeito'').
	\item Preparar ``pitch'' de 3 minutos (história + decisão + próximo passo).
\end{itemize}

\textbf{O aluno aprende a:}
\begin{itemize}[leftmargin=*]
	\item Organizar telas para guiar o olhar e reduzir ruído.
	\item Conectar insight a recomendação (chamada para ação).
\end{itemize}

\subsection*{Semana 12 (18/05--20/05) --- Trabalho 6 (apresentações) + Fechamento do conteúdo}
\textbf{Formato da aula:}
\begin{itemize}[leftmargin=*]
	\item \textbf{Até 1 hora}: apresentações do T6 (pitch + demo do dashboard).
	\item \textbf{Restante}: refinamento, revisão e checklist para Prova II.
\end{itemize}

\textbf{O aluno aprende a:}
\begin{itemize}[leftmargin=*]
	\item Defender um dashboard como produto (para usuário e decisão).
	\item Fechar lacunas técnicas (dados) e de comunicação (mensagem).
\end{itemize}

\begin{SolvedBox}
\textbf{Trabalho 6 (entrega na Semana 12):} dashboard final + roteiro de apresentação (3 min) + checklist completo (dados limpos + Big Idea + interatividade).
\end{SolvedBox}

\section{Resumo das Entregas Quinzenais}
\begin{center}
\renewcommand{\arraystretch}{1.15}
\begin{tabular}{@{}c c p{10.5cm}@{}}
\toprule
\textbf{Entrega} & \textbf{Semana} & \textbf{Tema (artefatos principais)} \\
\midrule
T1 & 02 & ETL básico + 3 visuais + Big Idea \\
T2 & 04 & Posição + dispersão + boxplot/hist + Big Idea \\
T3 & 06 & Missing/outliers/distribuição + 1 relação bivariada \\
T4 & 08 & Storytelling + wireframe + 1 visual no Tableau \\
T5 & 10 & Dashboard interativo (filtro/parâmetro/ação) + título-mensagem \\
T6 & 12 & Dashboard final + pitch (3 min) + checklist \\
\bottomrule
\end{tabular}
\end{center}

% Resumos curtos (até ~800 caracteres por semana) para copiar/colar no site do curso.
% Formato: 1 parágrafo por semana.

\section{Diário de aula (resumos para o site do curso)}

\subsection*{Semana 01 (02/02/2026)}
Diário de aula — Semana 01: apresentação do curso e do plano (dinâmica, avaliações e entregas). Introdução ao que é AED e como a apostila será usada. Tecnologias: Tableau (primeiro contato e visão inicial dos dados/visuais). Encaminhamento: desafio/atividade para a próxima semana.

\subsection*{Semana 02 (09/02--11/02)}
Diário de aula — Semana 02: apresentações do Trabalho 1 (até 1h) e alinhamento de critérios (clareza + Big Idea). Em seguida, fundamentos de dados retangulares (linhas/colunas, dicionário de dados, qualidade: tipos, ausências, duplicadas). Tecnologias: Python/Pandas (Jupyter) e Tableau. Entrega: T1.

\subsection*{Semana 03 (23/02--25/02)}
Diário de aula — Semana 03: estatística descritiva I (média, mediana, moda) e leitura crítica de resumos (assimetria e outliers). Prática com Pandas para descrever dados, contar categorias e comparar grupos. Tecnologias: Python/Pandas (Jupyter). Encaminhamento: revisar resumos e preparar exemplos para a Semana 04.

\subsection*{Semana 04 (02/03--04/03)}
Diário de aula — Semana 04: apresentações do Trabalho 2 (até 1h). Depois, estatística descritiva II (variância, desvio-padrão, amplitude, IQR) e leitura prática com boxplot/histograma para comparar grupos. Tecnologias: Python/Pandas (Jupyter) e Tableau para visualização. Entrega: T2.

\subsection*{Semana 05 (09/03--11/03)}
Diário de aula — Semana 05: distribuições e dados ausentes: como ler histogramas (forma/caudas) e como decidir sobre missing (medir, imputar, remover, sinalizar). Preparação de dados para visualização (tipos, datas e categorias). Tecnologias: Python/Pandas (Jupyter) e Tableau. Encaminhamento: checklist de qualidade para o T3.

\subsection*{Semana 06 (16/03--18/03)}
Diário de aula — Semana 06: apresentações do Trabalho 3 (até 1h). Em seguida, correlação introdutória: scatter, segmentação por categoria e cuidado com “correlação ≠ causalidade”. Tecnologias: Python/Pandas (Jupyter) e Tableau. Entrega: T3.

\subsection*{Semana 07 (23/03--25/03)}
Diário de aula — Semana 07: variáveis categóricas (tabelas de frequência, proporções e gráficos de barras) e revisão guiada da Unidade I (ETL, resumos, distribuições e correlação). Tecnologias: Tableau e Python/Pandas. Encaminhamento: estudo dirigido para a Prova I.

\subsection*{Semana 08 (20/04)}
Diário de aula — Semana 08: aula concentrada (segunda). Apresentações do Trabalho 4 (até 1h) e depois storytelling aplicado: AED vs explanativa, Big Idea, pergunta de decisão e storyboard de dashboard. Tecnologias: Tableau e apresentação/slide. Entrega: T4.

\subsection*{Semana 09 (27/04--29/04)}
Diário de aula — Semana 09: design de informação: Teste do Relance (título-mensagem), hierarquia visual (cor, tamanho, espaço em branco, alinhamento) e padrões práticos de dashboards. Tecnologias: Tableau. Encaminhamento: refinar visuais e preparar interatividade para a Semana 10.

\subsection*{Semana 10 (04/05--06/05)}
Diário de aula — Semana 10: apresentações do Trabalho 5 (até 1h) e depois Tableau interativo (filtros, parâmetros e ações). Observação: semana com feriado, ajuste de ritmo. Tecnologias: Tableau. Entrega: T5.

\subsection*{Semana 11 (11/05--13/05)}
Diário de aula — Semana 11: dashboards que “forçam” decisão: estrutura (contexto, diagnóstico, recomendação), consistência de métricas/definições e preparação de pitch de 3 minutos (história + decisão + próximo passo). Tecnologias: Tableau. Encaminhamento: revisar e consolidar para o trabalho final.

\subsection*{Semana 12 (18/05--20/05)}
Diário de aula — Semana 12: apresentações do Trabalho 6 (até 1h): pitch + demo do dashboard final. Depois, refinamento, revisão e checklist para a Prova II. Tecnologias: Tableau e Python/Pandas (quando necessário para ajustes de dados). Entrega: T6.


\end{document}

