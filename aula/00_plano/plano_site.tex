% Resumo do plano de aula (até ~800 caracteres por semana) para copiar/colar no site do curso.
% Formato: itens numerados por semana.

\section{Plano de aula (resumo para o site do curso)}

\begin{enumerate}[leftmargin=*]
  \item \textbf{Semana 01 (02/02--04/02):} Apresentação do curso e do plano (dinâmica, avaliações e entregas). O que é AED e como usar a apostila. Primeiro contato com o Tableau para ter visão inicial (conectar dados, navegar e criar visuais simples). \textbf{Tecnologias:} Tableau. \textbf{Leitura:} Cap. 01 (O que é AED?) e Cap. 05 (Fundamentos do Tableau e conexão com dados).

  \item \textbf{Semana 02 (09/02--11/02) — T1:} Até 1h de apresentações do Trabalho 1 (infográfico + Big Idea). Depois: dados retangulares, dicionário de dados e qualidade (tipos, ausências, duplicadas), preparando dataset “pronto para viz”. \textbf{Tecnologias:} Python/Pandas (Jupyter) e Tableau. \textbf{Leitura:} Cap. 05 (ingestão, preparo e conexão) + revisar Cap. 01 (fluxo da AED). \textbf{Entrega:} T1.

  \item \textbf{Semana 03 (23/02--25/02):} Estatística descritiva I (média, mediana, moda) e leitura crítica (assimetria/outliers). Prática de resumos e comparações básicas para apoiar decisões. \textbf{Tecnologias:} Python/Pandas (Jupyter). \textbf{Leitura:} Cap. 02 (Análise Univariada — medidas de posição e tipos de variáveis).

  \item \textbf{Semana 04 (02/03--04/03) — T2:} Até 1h de apresentações do Trabalho 2. Depois: variabilidade (variância, desvio-padrão, amplitude, IQR) e gráficos para comparar grupos (boxplot/histograma). \textbf{Tecnologias:} Python/Pandas e Tableau. \textbf{Leitura:} Cap. 02 (dispersão) + Cap. 06 (tipos de gráficos: hist/boxplot). \textbf{Entrega:} T2.

  \item \textbf{Semana 05 (09/03--11/03):} Distribuições e dados ausentes: interpretar histogramas (forma/caudas) e decidir sobre missing (medir, imputar, remover, sinalizar). Preparar dados para visualização (tipos/datas/categorias). \textbf{Tecnologias:} Python/Pandas e Tableau. \textbf{Leitura:} Cap. 02 (univariada/distribuições) + Cap. 06 (histograma/boxplot).

  \item \textbf{Semana 06 (16/03--18/03) — T3:} Até 1h de apresentações do Trabalho 3. Depois: correlação introdutória, scatter e segmentação por categoria; cuidado com “correlação ≠ causalidade”. \textbf{Tecnologias:} Python/Pandas e Tableau. \textbf{Leitura:} Cap. 03 (Análise Bivariada) + Cap. 06 (dispersão). \textbf{Entrega:} T3.

  \item \textbf{Semana 07 (23/03--25/03):} Variáveis categóricas: proporções, tabelas de frequência e gráficos de barras; revisão guiada da Unidade I (ETL, resumos, distribuições e correlação). \textbf{Tecnologias:} Tableau e Python/Pandas. \textbf{Leitura:} Cap. 02 (qualitativas/categóricas) + Cap. 06 (barras).

  \item \textbf{Semana 08 (20/04) — T4:} Aula concentrada. Até 1h de apresentações do Trabalho 4. Depois: storytelling aplicado (AED vs explanativa), Big Idea, pergunta de decisão e storyboard (3–5 telas) para dashboard. \textbf{Tecnologias:} Tableau e slides. \textbf{Leitura:} Cap. 08 (Storytelling e dashboards) + apoio: Cap. 07 (insight). \textbf{Entrega:} T4.

  \item \textbf{Semana 09 (27/04--29/04):} Design de informação: Teste do Relance, hierarquia visual, alinhamento e padrões para dashboards (clareza e consistência). \textbf{Tecnologias:} Tableau. \textbf{Leitura:} Cap. 06 (escolha de gráficos) + Cap. 08 (design de dashboards).

  \item \textbf{Semana 10 (04/05--06/05) — T5:} Até 1h de apresentações do Trabalho 5. Depois: Tableau interativo (filtros, parâmetros e ações) para exploração guiada. Observação: semana com feriado (ajuste de ritmo). \textbf{Tecnologias:} Tableau. \textbf{Leitura:} Cap. 08 (ações) + Cap. 06 (parâmetros/performance). \textbf{Entrega:} T5.

  \item \textbf{Semana 11 (11/05--13/05):} Dashboards que “forçam” decisão: estrutura (contexto, diagnóstico, recomendação), consistência de métricas e preparação de pitch (3 min). \textbf{Tecnologias:} Tableau. \textbf{Leitura:} Cap. 08 (narrativa e publicação) + apoio: Cap. 10 (Tableau e EDA — conceitos de dashboards).

  \item \textbf{Semana 12 (18/05--20/05) — T6:} Até 1h de apresentações do Trabalho 6 (pitch + demo). Depois: refinamento final, revisão e checklist para Prova II. \textbf{Tecnologias:} Tableau e (quando necessário) Python/Pandas. \textbf{Leitura:} revisão Cap. 08 e Cap. 10. \textbf{Entrega:} T6.
\end{enumerate}
