% Resumos curtos (até ~800 caracteres por semana) para copiar/colar no site do curso.
% Formato: 1 parágrafo por semana.

\section{Diário de aula (resumos para o site do curso)}

\subsection*{Semana 01 (02/02/2026)}
Diário de aula — Semana 01: apresentação do curso e do plano (dinâmica, avaliações e entregas). Introdução ao que é AED e como a apostila será usada. Tecnologias: Tableau (primeiro contato e visão inicial dos dados/visuais). Encaminhamento: desafio/atividade para a próxima semana.

\subsection*{Semana 02 (09/02--11/02)}
Diário de aula — Semana 02: apresentações do Trabalho 1 (até 1h) e alinhamento de critérios (clareza + Big Idea). Em seguida, fundamentos de dados retangulares (linhas/colunas, dicionário de dados, qualidade: tipos, ausências, duplicadas). Tecnologias: Python/Pandas (Jupyter) e Tableau. Entrega: T1.

\subsection*{Semana 03 (23/02--25/02)}
Diário de aula — Semana 03: estatística descritiva I (média, mediana, moda) e leitura crítica de resumos (assimetria e outliers). Prática com Pandas para descrever dados, contar categorias e comparar grupos. Tecnologias: Python/Pandas (Jupyter). Encaminhamento: revisar resumos e preparar exemplos para a Semana 04.

\subsection*{Semana 04 (02/03--04/03)}
Diário de aula — Semana 04: apresentações do Trabalho 2 (até 1h). Depois, estatística descritiva II (variância, desvio-padrão, amplitude, IQR) e leitura prática com boxplot/histograma para comparar grupos. Tecnologias: Python/Pandas (Jupyter) e Tableau para visualização. Entrega: T2.

\subsection*{Semana 05 (09/03--11/03)}
Diário de aula — Semana 05: distribuições e dados ausentes: como ler histogramas (forma/caudas) e como decidir sobre missing (medir, imputar, remover, sinalizar). Preparação de dados para visualização (tipos, datas e categorias). Tecnologias: Python/Pandas (Jupyter) e Tableau. Encaminhamento: checklist de qualidade para o T3.

\subsection*{Semana 06 (16/03--18/03)}
Diário de aula — Semana 06: apresentações do Trabalho 3 (até 1h). Em seguida, correlação introdutória: scatter, segmentação por categoria e cuidado com “correlação ≠ causalidade”. Tecnologias: Python/Pandas (Jupyter) e Tableau. Entrega: T3.

\subsection*{Semana 07 (23/03--25/03)}
Diário de aula — Semana 07: variáveis categóricas (tabelas de frequência, proporções e gráficos de barras) e revisão guiada da Unidade I (ETL, resumos, distribuições e correlação). Tecnologias: Tableau e Python/Pandas. Encaminhamento: estudo dirigido para a Prova I.

\subsection*{Semana 08 (20/04)}
Diário de aula — Semana 08: aula concentrada (segunda). Apresentações do Trabalho 4 (até 1h) e depois storytelling aplicado: AED vs explanativa, Big Idea, pergunta de decisão e storyboard de dashboard. Tecnologias: Tableau e apresentação/slide. Entrega: T4.

\subsection*{Semana 09 (27/04--29/04)}
Diário de aula — Semana 09: design de informação: Teste do Relance (título-mensagem), hierarquia visual (cor, tamanho, espaço em branco, alinhamento) e padrões práticos de dashboards. Tecnologias: Tableau. Encaminhamento: refinar visuais e preparar interatividade para a Semana 10.

\subsection*{Semana 10 (04/05--06/05)}
Diário de aula — Semana 10: apresentações do Trabalho 5 (até 1h) e depois Tableau interativo (filtros, parâmetros e ações). Observação: semana com feriado, ajuste de ritmo. Tecnologias: Tableau. Entrega: T5.

\subsection*{Semana 11 (11/05--13/05)}
Diário de aula — Semana 11: dashboards que “forçam” decisão: estrutura (contexto, diagnóstico, recomendação), consistência de métricas/definições e preparação de pitch de 3 minutos (história + decisão + próximo passo). Tecnologias: Tableau. Encaminhamento: revisar e consolidar para o trabalho final.

\subsection*{Semana 12 (18/05--20/05)}
Diário de aula — Semana 12: apresentações do Trabalho 6 (até 1h): pitch + demo do dashboard final. Depois, refinamento, revisão e checklist para a Prova II. Tecnologias: Tableau e Python/Pandas (quando necessário para ajustes de dados). Entrega: T6.
