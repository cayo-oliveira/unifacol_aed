\chapter{Semana 02 (09/02--11/02) --- Trabalho 1 + Dados retangulares}

\section{Objetivo da semana}
Nesta semana eu executo duas coisas em paralelo: (1) \textbf{avaliar} a entrega curta do Trabalho 1 (vale 0,5 ponto na I unidade); (2) consolidar a base de \textbf{dados retangulares} e qualidade de dados.

\section{Roteiro (19h--22h)}
\begin{itemize}[leftmargin=*]
  \item \textbf{19:00--20:00} --- Apresentações do Trabalho 1 (no máximo 1h).
  \item \textbf{20:00--21:20} --- Teoria + prática: dados retangulares, dicionário de dados, tipos, ausências, duplicadas.
  \item \textbf{21:20--22:00} --- Mini-lab: refazer o ETL com checklist e exportar CSV limpo.
\end{itemize}

\section{Entrega (Trabalho 1 --- 0,5 ponto na I unidade)}
\begin{SolvedBox}
\textbf{Formato:} cada aluno tem \textbf{10 minutos} para construir um infográfico/visual (com base no tema escolhido na Semana 01) e explicar para a turma.

\textbf{Artefatos:}
\begin{itemize}[leftmargin=*]
  \item 1 dataset (CSV) escolhido dentre os 20 temas;
  \item 1 ETL básico em Python (notebook) com pelo menos 1 limpeza real;
  \item 1 infográfico no Tableau (ou 1 dashboard simples);
  \item 1 slide/texto com \textbf{Big Idea} + 3 bullets de evidência.
\end{itemize}

\textbf{Critério:} clareza (Teste do Relance), coerência do recorte e conclusão ligada a decisão.
\end{SolvedBox}

\section{Próximo passo}
Depois que os alunos apresentarem, eu conecto o que eles trouxeram com o conceito de \textbf{tabela retangular} e com o que vai cair na Prova I Unidade.
