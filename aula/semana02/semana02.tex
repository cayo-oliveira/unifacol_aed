\chapter{Semana 02 (09/02--11/02) --- Introdução ao Tableau + Dataset de Desempenho Acadêmico}

\section{Objetivo da semana}
Olá pessoal, bom dia! Bem-vindos à Semana 02. Hoje eu vou executar duas coisas em paralelo: (1) avaliar as apresentações do Trabalho 1, que valem 0,5 ponto na I unidade; (2) introduzir o Tableau como ferramenta poderosa para visualização de dados, aplicando tudo ao dataset de desempenho acadêmico dos estudantes. O objetivo é que vocês saiam daqui sabendo criar perguntas estratégicas, montar dashboards interativos e contar histórias com dados (storytelling). Vamos conectar teoria de AED com prática real!

\textbf{Por que AED?} AED investiga datasets para descobrir padrões invisíveis em planilhas; usa estatísticas e gráficos para insights acionáveis. \textbf{Por que Tableau?} Diferente do Excel (estático), Tableau é interativo, drag-and-drop, ideal para exploração rápida. \textbf{Objetivos detalhados:} Dominar ETL em Python (preparação), visualizações no Tableau (análise), storytelling (comunicação). Esperamos vocês aplicando a problemas reais de educação.

Mas antes de começar, vamos contextualizar. AED, ou Análise Exploratória de Dados, é o processo de investigar datasets para descobrir padrões, anomalias e insights. Usamos estatísticas descritivas, visualizações e limpeza de dados. Python é ideal para ETL (Extract, Transform, Load), mas para visualizações interativas, Tableau é superior – permite drag-and-drop, filtros dinâmicos e compartilhamento fácil.

O dataset que vamos usar é sobre desempenho acadêmico de estudantes. Coletado do Kaggle, tem 20 colunas como horas de estudo, frequência, gênero, e a variável alvo: notas dos exames. O objetivo final é responder perguntas como "Quais fatores mais influenciam as notas?" e criar dashboards que comuniquem decisões educacionais.

\textbf{Por que este dataset?} Real, diverso (6607 alunos), cobre fatores sociais/econômicos; ideal para AED aplicada. \textbf{Perguntas estratégicas definidas:} "Frequência afeta notas?" (esperamos correlação alta, decisão: incentivar presença); "Renda impacta?" (esperamos desigualdade, decisão: apoio social). \textbf{O que esperamos das respostas?} Insights quantificáveis para políticas educacionais.

Por que Tableau? Porque combina dados de múltiplas fontes, cria mapas, gráficos avançados e stories narrativas. Hoje, vocês aprenderão desde a instalação até a publicação. Lembrem: visualização não é só bonita, é comunicação de dados para ação.

\begin{BoardBox}
\textbf{Conceitos-chave de AED}
\begin{itemize}
  \item \textbf{ETL}: Extrair dados brutos (ex.: CSV do Kaggle), transformar (limpar nulos, normalizar), carregar para análise. \textbf{Por que?} Dados reais têm erros; ETL prepara para insights confiáveis. \textbf{Objetivo:} Dataset limpo acelera visualizações no Tableau.
  \item \textbf{Variável alvo}: Nota\_Exame – o que queremos prever/explicar. \textbf{Por que?} Foca análise em impacto educacional. \textbf{Objetivo:} Responder "Quais fatores melhoram notas?".
  \item \textbf{Fatores influentes}: Horas\_Estudadas, Frequencia, etc. (correlações positivas/negativas). \textbf{Por que?} Identifica causas de sucesso/falha. \textbf{Objetivo:} Decisões como "Aumentar frequência".
  \item \textbf{Objetivo final}: Dashboards que respondam perguntas estratégicas para melhorar educação. \textbf{Por que?} AED leva a ação baseada em dados.
  \item \textbf{Fórmula z-score}: $z = \frac{x - \mu}{\sigma}$ (normaliza para média 0, desvio 1). \textbf{O que é?} Padroniza valores para comparar escalas diferentes (ex.: horas de estudo vs. notas). \textbf{Por que usar?} Evita viés de unidades; facilita correlações. \textbf{Objetivo:} Features comparáveis em modelos.
  \item \textbf{Correlação Pearson}: $r = \frac{\sum (x_i - \bar{x})(y_i - \bar{y})}{\sqrt{\sum (x_i - \bar{x})^2 \sum (y_i - \bar{y})^2}}$ (mede relação linear, -1 a 1). \textbf{O que é?} Força e direção da relação (1 = perfeita positiva, -1 = negativa). \textbf{Por que?} Quantifica influência (ex.: Frequencia ~0.58 com Nota). \textbf{Objetivo:} Priorizar fatores (alta correlação = foco).
\end{itemize}
\end{BoardBox}

\section{Roteiro Detalhado (19h--22h)}
\begin{itemize}[leftmargin=*]
  \item \textbf{19:00--20:00} --- Apresentações dos alunos (Trabalho 1). \textbf{Por que?} Avalia aplicação prática de AED; esperamos ver ETL real e visualizações.
  \item \textbf{20:00--21:30} --- Introdução ao Tableau: teoria, conexão de dados, visualizações básicas e dashboards. \textbf{Por que?} Core da aula; conecta Python a Tableau.
  \item \textbf{21:30--22:00} --- Atividades práticas e discussão de insights. \textbf{Por que?} Reforça aprendizado; esperamos discussões aplicadas.
\end{itemize}

\subsection{19:00--20:00: Apresentações do Trabalho 1}
Olá pessoal, vamos começar com as apresentações do Trabalho 1. Cada um de vocês tem 10 minutos para apresentar o infográfico ou dashboard que criaram com base no tema escolhido na Semana 01. Lembrem-se: foquem na Big Idea e nos 3 bullets de evidência. Eu vou avaliar clareza, coerência e ligação com decisão.

O Trabalho 1 é uma introdução prática à AED: vocês escolheram um dataset de 20 temas possíveis, fizeram ETL básico em Python (pelo menos uma limpeza, como remover nulos), criaram um infográfico no Tableau e resumiram em Big Idea + 3 bullets. O objetivo é aplicar o ciclo de AED: pergunta → dados → limpeza → visualização → insight → decisão.

Enquanto vocês se preparam, eu vou configurar o ambiente para a parte prática. Abram seus laptops e instalem o Tableau Public (gratuito) se ainda não tiverem. O link é \url{https://public.tableau.com/}. Também, preparem o Python com o ambiente virtual que preparei na pasta \texttt{tableau/dataset/python/}. Ambiente virtual (venv) isola dependências, evitando conflitos. Vamos explicar isso em detalhes na próxima seção.

Durante as apresentações, observem: o infográfico passa no "teste do relance" (entende-se em 5 segundos)? A Big Idea é acionável? Os bullets apoiam com evidências quantitativas? Perguntem: "Alguém teve dificuldade em conectar o dataset no Tableau?" (esperamos respostas sobre erros comuns, para discutir troubleshooting); "Como vocês escolheram as cores?" (esperamos discussões sobre acessibilidade e impacto visual). \textbf{Por que essas perguntas?} Estimulam reflexão sobre design e técnica; definidas para avaliar clareza e aplicabilidade.

\begin{BoardBox}
\textbf{Critérios de Avaliação do Trabalho 1}
\begin{itemize}
  \item \textbf{Clareza}: Visual limpo, cores adequadas, sem clutter (desordem).
  \item \textbf{Coerência}: Recorte lógico do tema, dados relevantes.
  \item \textbf{Decisão}: Big Idea leva a ação (ex.: "Aumentar investimento em X para melhorar Y").
  \item \textbf{Técnico}: ETL real (ex.: fillna, drop duplicates), visualização correta.
  \item \textbf{Perguntas para alunos}: "O que foi mais desafiador no ETL?" "Como o Tableau ajudou na Big Idea?"
\end{itemize}
\end{BoardBox}

\subsection{20:00--21:30: Introdução ao Tableau}
Agora, vamos mergulhar no Tableau! Primeiro, o que é o Tableau? É uma ferramenta de business intelligence para criar visualizações interativas e dashboards sem precisar de muito código. Existem versões: Desktop (paga), Public (gratuita, online) e Server. Hoje usaremos Public.

No Tableau, dados são divididos em \textbf{Dimensions} (categóricos, como Gênero) e \textbf{Measures} (numéricos, como Nota\_Exame). Dimensions criam rótulos/categorias; Measures calculam agregações (soma, média). Por que? Facilita drag-and-drop: arraste Dimension para Rows para categorias, Measure para Columns para valores.

Mas antes, vamos preparar o Python. Ambiente virtual (venv) isola projetos, evitando conflitos de bibliotecas. Sem venv, instalar pandas globalmente pode quebrar outros projetos. O comando \texttt{python -m venv .venv} cria uma pasta com Python isolado. Ativar com \texttt{.venv\Scripts\activate} (Windows) ou \texttt{source .venv/bin/activate} (Mac/Linux). Então, \texttt{pip install -r requirements.txt} instala pandas, numpy, scikit-learn. Objetivo: reprodutibilidade e limpeza.

Python é linguagem de programação para dados: pandas manipula DataFrames (tabelas), numpy arrays numéricos, sklearn machine learning. No notebook, executamos ETL: carregar CSV, limpar nulos (fillna com moda), traduzir categorias, criar features (Categoria\_Desempenho via pd.cut), normalizar (StandardScaler). Por que? Dados brutos têm nulos, tipos errados; transformações facilitam análise.

\subsubsection{20:00--20:15: Conexão de Dados e Exploração}
Olá pessoal, bom trabalho nas apresentações! Agora, vamos abrir o Tableau Public. No menu inicial, cliquem em "Connect to Data" e selecionem "Text file" para conectar ao CSV que preparei: \texttt{tableau/dataset/student\_academic\_performance/StudentPerformanceFactors\_transformado.csv}.

Uma vez conectado, vocês verão a tela de Data Source. Aqui, o Tableau detecta tipos automaticamente. Vejam as colunas: temos numéricas como \texttt{Horas\_Estudadas}, categóricas como \texttt{Genero}, e as novas que criei como \texttt{Categoria\_Desempenho}.

Mas primeiro, vamos executar o notebook de preparação. Abram o Jupyter: \texttt{jupyter notebook preparacao\_tableau.ipynb}. Configurem o kernel (.venv). Execute célula por célula:

1. \textbf{Imports}: \texttt{import pandas as pd} – carrega biblioteca para DataFrames. \texttt{import numpy as np} – arrays eficientes. \texttt{from sklearn.preprocessing import StandardScaler} – normalização z-score (média 0, desvio 1). Objetivo: ter ferramentas para manipulação.

2. \textbf{Load Data}: \texttt{df = pd.read\_csv('../student\_academic\_performance/StudentPerformanceFactors.csv')} – lê CSV em DataFrame. \texttt{print(df.shape)} – mostra (linhas, colunas). \texttt{print(df.head())} – primeiras 5 linhas. Por que? Verificar carregamento correto.

3. \textbf{Data Cleaning}: \texttt{df.isnull().sum()} – conta nulos por coluna. Para colunas com nulos (ex.: Teacher\_Quality), \texttt{mode\_val = df[col].mode()[0]} – moda (valor mais frequente). \texttt{df[col].fillna(mode\_val, inplace=True)} – preenche nulos. Objetivo: dados completos para análise.

4. \textbf{Data Transformations}: Mapeamentos traduzem (ex.: 'Low' → 'Baixo'). \texttt{df['Categoria\_Desempenho'] = pd.cut(df['Exam\_Score'], bins=[0,65,80,100], labels=['Baixo','Médio','Alto'])} – categoriza notas. \texttt{df['Faixa\_Frequencia'] = pd.cut(...)} – agrupa frequência. \texttt{scaler = StandardScaler(); df['Horas\_Estudadas\_Normalizado'] = scaler.fit\_transform(df[['Hours\_Studied']])} – normaliza para comparações. Por que? Features novas revelam padrões (ex.: alunos de alto desempenho).

5. \textbf{Save}: \texttt{df.to\_csv(..., index=False)} – salva sem índice. Objetivo: CSV pronto para Tableau.

Executem e vejam o output. Isso é ETL prático!

\begin{BoardBox}
\textbf{Passos do Notebook (ETL) com Sintaxes}
\begin{enumerate}
  \item \textbf{Imports}: \texttt{import pandas as pd} (DataFrames), \texttt{import numpy as np} (arrays), \texttt{from sklearn.preprocessing import StandardScaler} (z-score). \textbf{Por que?} Ferramentas essenciais para manipulação. \textbf{Objetivo:} Preparar ambiente para ETL eficiente.
  \item \textbf{Load}: \texttt{pd.read\_csv(filepath\_or\_buffer='../student\_academic\_performance/StudentPerformanceFactors.csv', sep=',', header=0)} – lê CSV. Output esperado: shape=(6607, 20), head() mostra primeiras linhas. \textbf{Por que?} Carrega dados brutos. \textbf{Objetivo:} Verificar integridade (linhas/colunas corretas).
  \item \textbf{Clean}: \texttt{df.isnull().sum()} – conta nulos (ex.: Teacher\_Quality: 78). \texttt{df.fillna(df.mode()[0], inplace=True)} – preenche com moda. \textbf{Por que?} Nulos quebram análises. \textbf{Objetivo:} Dataset completo; moda preserva distribuição.
  \item \textbf{Transform}: \texttt{df.map(\{'Low': 'Baixo'\})} – traduz. \texttt{pd.cut(df['Exam\_Score'], bins=[0,65,80,100], labels=['Baixo','Médio','Alto'])} – categoriza. \texttt{StandardScaler().fit\_transform(df[['col']])} – normaliza. \textbf{Por que?} Padroniza para Tableau (ex.: z-score evita escalas diferentes). \textbf{Objetivo:} Features novas (Categoria\_Desempenho) revelam padrões; normalização facilita comparações.
  \item \textbf{Save}: \texttt{df.to\_csv('file.csv', index=False)} – exporta limpo. \textbf{Por que?} Tableau precisa CSV limpo. \textbf{Objetivo:} Arquivo pronto para importação sem erros.
\end{enumerate}
\textbf{Objetivo final}: Dataset otimizado para visualizações. Se erro (ex.: ModuleNotFoundError), instale com pip. \textbf{Por que ETL?} Dados brutos têm inconsistências; transformação acelera AED.
\end{BoardBox}

Para explorar, cliquem em "Sheet 1". Arrastem \texttt{Nota\_Exame} para Rows e vejam o histograma automático. Isso é AED visual: distribuição mostra concentração em 65-75, assimetria esquerda (notas baixas raras).

\subsubsection{20:15--20:45: Visualizações Básicas}
Agora, vamos criar gráficos guiados por perguntas estratégicas. A lógica é sempre a mesma: (1) definir a pergunta, (2) escolher o gráfico certo, (3) selecionar as colunas, (4) montar passo a passo no Tableau, (5) ler o resultado, (6) responder à pergunta com uma decisão. Vamos aplicar isso a 4 visualizações.

\bigskip
\textbf{--- Visualização 1: Scatter Plot ---}

\begin{SolvedBox}
\textbf{Pergunta estratégica 1:} ``Estudar mais horas realmente melhora a nota do exame?''

\textbf{Por que essa pergunta?} Se houver correlação positiva, a decisão é incentivar horas de estudo; se não houver, devemos investigar outros fatores.

\textbf{Tipo de gráfico:} Scatter Plot (dispersão) --- ideal para ver relação entre duas variáveis numéricas.

\textbf{Colunas usadas:} \texttt{Horas\_Estudadas} (eixo X), \texttt{Nota\_Exame} (eixo Y), \texttt{Genero} (cor).
\end{SolvedBox}

Passo a passo no Tableau:
\begin{enumerate}
  \item Cliquem em \textbf{Sheet 1} (aba inferior). Renomeiem para ``Scatter\_Horas\_vs\_Nota'' (duplo clique na aba).
  \item No painel esquerdo, encontrem \texttt{Horas\_Estudadas} (em Measures). \textbf{Arrastem para Columns} (prateleira superior).
  \item Encontrem \texttt{Nota\_Exame} (em Measures). \textbf{Arrastem para Rows} (prateleira lateral).
  \item Resultado parcial: vocês verão um único ponto (Tableau agrega por padrão). Para ver todos os pontos, arrastem \texttt{Genero} para \textbf{Detail} no cartão Marks (painel central).
  \item Agora arrastem \texttt{Genero} também para \textbf{Color} no cartão Marks. Os pontos ficarão coloridos por gênero.
  \item Para adicionar linha de tendência: menu \textbf{Analysis > Trend Lines > Show Trend Lines}. Selecionem ``Linear''.
  \item Opcional: cliquem com botão direito no eixo X > \textbf{Edit Axis} > Title: ``Horas Estudadas''. Façam o mesmo no eixo Y.
\end{enumerate}

\begin{SolvedBox}
\textbf{Leitura do resultado:} Olhem o gráfico. Os pontos sobem da esquerda para a direita? Se sim, a correlação é positiva. A linha de tendência confirma: inclinação para cima = mais horas $\to$ melhores notas. A correlação esperada é ~0.45 (moderada positiva). Notem que os pontos masculinos e femininos seguem padrão semelhante.

\textbf{Resposta à pergunta:} \textbf{SIM}, estudar mais horas melhora a nota, mas a relação é moderada (não perfeita). Há outros fatores em jogo. \textbf{Decisão:} Incentivar horas de estudo, mas investigar fatores complementares.
\end{SolvedBox}

\bigskip
\textbf{--- Visualização 2: Bar Chart ---}

\begin{SolvedBox}
\textbf{Pergunta estratégica 2:} ``O envolvimento dos pais faz diferença nas notas?''

\textbf{Por que essa pergunta?} Se pais mais envolvidos = notas maiores, a decisão é criar programas de engajamento parental na escola.

\textbf{Tipo de gráfico:} Bar Chart (barras) --- ideal para comparar médias entre categorias.

\textbf{Colunas usadas:} \texttt{Envolvimento\_Parental} (eixo X, Dimension), \texttt{Nota\_Exame} (eixo Y, Measure, agregação: Average).
\end{SolvedBox}

Passo a passo no Tableau:
\begin{enumerate}
  \item Criem nova sheet: cliquem no ícone ``+'' (aba inferior). Renomeiem para ``Bar\_Envolvimento''.
  \item Arrastem \texttt{Envolvimento\_Parental} (Dimensions) para \textbf{Columns}.
  \item Arrastem \texttt{Nota\_Exame} (Measures) para \textbf{Rows}. O Tableau calcula SUM por padrão.
  \item \textbf{Mudem a agregação:} cliquem com botão direito em \texttt{Nota\_Exame} na prateleira Rows > \textbf{Measure} > \textbf{Average}. Agora mostra a média.
  \item \textbf{Ordenem:} cliquem no ícone de ordenação decrescente na toolbar (seta para baixo). As barras ficarão do maior para o menor.
  \item \textbf{Adicionem rótulos:} arrastem \texttt{Nota\_Exame} para \textbf{Label} no cartão Marks. Agora cada barra mostra o valor.
  \item \textbf{Cor:} arrastem \texttt{Envolvimento\_Parental} para \textbf{Color} para diferenciar visualmente.
\end{enumerate}

\begin{SolvedBox}
\textbf{Leitura do resultado:} Comparem as barras. ``Alto'' mostra média ~75, ``Médio'' ~70, ``Baixo'' ~65. Diferença de 10 pontos entre extremos. Isso é significativo: 10 pontos pode ser a diferença entre aprovação e reprovação.

\textbf{Resposta à pergunta:} \textbf{SIM}, envolvimento parental alto está associado a notas 15\% maiores. \textbf{Decisão:} Criar programas de reuniões escola-família e acompanhamento parental.
\end{SolvedBox}

\bigskip
\textbf{--- Visualização 3: Histograma ---}

\begin{SolvedBox}
\textbf{Pergunta estratégica 3:} ``Como as notas estão distribuídas? Há muitos alunos reprovando?''

\textbf{Por que essa pergunta?} Se a distribuição mostra concentração em notas baixas, a decisão é reforço escolar urgente.

\textbf{Tipo de gráfico:} Histograma --- ideal para ver distribuição de uma variável numérica.

\textbf{Colunas usadas:} \texttt{Nota\_Exame} (eixo X, bins automáticos), contagem (eixo Y).
\end{SolvedBox}

Passo a passo no Tableau:
\begin{enumerate}
  \item Nova sheet > renomeiem para ``Histograma\_Notas''.
  \item Arrastem \texttt{Nota\_Exame} para \textbf{Columns}.
  \item No cartão Marks, selecionem o menu ``Automatic'' e troquem para \textbf{Bar} (ícone de barras).
  \item Cliquem com botão direito em \texttt{Nota\_Exame} na prateleira Columns > \textbf{Create Bins...} > Tamanho: \textbf{5}. Cliquem OK.
  \item Agora arrastem a nova pill ``Nota\_Exame (bin)'' para substituir a original em Columns.
  \item O eixo Y (contagem) aparece automaticamente como \texttt{CNT(Nota\_Exame)}.
  \item \textbf{Cor por desempenho:} arrastem \texttt{Categoria\_Desempenho} para \textbf{Color}. Vermelhos (Baixo), amarelos (Médio), verdes (Alto).
\end{enumerate}

\begin{SolvedBox}
\textbf{Leitura do resultado:} A distribuição concentra-se entre 65--75 (maioria dos alunos). A cauda esquerda (notas <55) é pequena mas preocupante. A cor mostra que ``Baixo'' (vermelho) domina abaixo de 65.

\textbf{Resposta à pergunta:} A maioria está na faixa média. Cerca de 25\% estão abaixo de 65 (Baixo). \textbf{Decisão:} Reforço escolar direcionado ao quartil inferior.
\end{SolvedBox}

\bigskip
\textbf{--- Visualização 4: Box Plot ---}

\begin{SolvedBox}
\textbf{Pergunta estratégica 4:} ``A frequência às aulas impacta as notas? Há outliers?''

\textbf{Por que essa pergunta?} Se alunos com alta frequência têm notas consistentemente maiores, a decisão é política de presença obrigatória.

\textbf{Tipo de gráfico:} Box Plot --- ideal para ver mediana, quartis e outliers por categoria.

\textbf{Colunas usadas:} \texttt{Faixa\_Frequencia} (eixo X, Dimension), \texttt{Nota\_Exame} (eixo Y, desagregado).
\end{SolvedBox}

Passo a passo no Tableau:
\begin{enumerate}
  \item Nova sheet > renomeiem para ``BoxPlot\_Frequencia''.
  \item Arrastem \texttt{Faixa\_Frequencia} para \textbf{Columns}.
  \item Arrastem \texttt{Nota\_Exame} para \textbf{Rows}.
  \item No cartão Marks, troquem ``Automatic'' para \textbf{Circle} (círculos).
  \item Menu \textbf{Analysis > Add Reference Line > Box Plot}. Configurem: mediana, quartis, whiskers 1.5×IQR.
  \item Arrastem \texttt{Faixa\_Frequencia} para \textbf{Color} para diferenciar.
  \item Observem os outliers (pontos fora dos whiskers): alunos com frequência alta mas notas baixas --- o que está acontecendo com eles?
\end{enumerate}

\begin{SolvedBox}
\textbf{Leitura do resultado:} A mediana sobe conforme a frequência aumenta: ``Baixa'' ~60, ``Média'' ~68, ``Alta'' ~75. Os outliers em ``Alta'' (alunos presentes mas com notas baixas) podem indicar dificuldades de aprendizagem que presença sozinha não resolve.

\textbf{Resposta à pergunta:} \textbf{SIM}, frequência impacta fortemente (correlação ~0.58). Mas outliers revelam que presença é necessária, não suficiente. \textbf{Decisão:} Política de presença + tutoria para alunos com notas abaixo da mediana.
\end{SolvedBox}

Perguntem aos alunos: ``Qual dos 4 gráficos responde melhor à pergunta 'O que fazer para melhorar notas?'{}''. Esperamos debate: scatter mostra relação individual, bar mostra categorias, histograma mostra distribuição, box plot mostra dispersão. Juntos, contam a história completa.

\begin{BoardBox}
\textbf{Resumo: Pergunta $\to$ Gráfico $\to$ Colunas $\to$ Resposta}
\begin{enumerate}
  \item ``Mais horas = melhores notas?'' $\to$ Scatter Plot $\to$ \texttt{Horas\_Estudadas} × \texttt{Nota\_Exame}, cor \texttt{Genero} $\to$ SIM, correlação ~0.45. Decisão: incentivar estudo.
  \item ``Pais envolvidos = notas maiores?'' $\to$ Bar Chart $\to$ \texttt{Envolvimento\_Parental} × avg(\texttt{Nota\_Exame}) $\to$ SIM, +10 pontos. Decisão: engajamento parental.
  \item ``Quantos alunos reprovam?'' $\to$ Histograma $\to$ \texttt{Nota\_Exame} (bins=5), cor \texttt{Categoria\_Desempenho} $\to$ ~25\% abaixo de 65. Decisão: reforço escolar.
  \item ``Frequência impacta?'' $\to$ Box Plot $\to$ \texttt{Faixa\_Frequencia} × \texttt{Nota\_Exame} $\to$ SIM, +15 pontos mediana. Decisão: política de presença + tutoria.
\end{enumerate}
\end{BoardBox}

\subsubsection{20:45--21:15: Dashboards Interativos}

\begin{SolvedBox}
\textbf{Pergunta estratégica do dashboard:} ``Quais subgrupos de alunos precisam de intervenção urgente, e quais fatores priorizar?''

\textbf{Por que essa pergunta?} Um dashboard integra as 4 visualizações que fizemos. Ao filtrar por subgrupo (renda, gênero, escola), o educador identifica QUEM precisa de ajuda e POR QUÊ. Isso é AED aplicada: de dados a decisão.

\textbf{Objetivo:} Criar um painel único que responda: qual a média geral? quem está abaixo? o que influencia? como agir?
\end{SolvedBox}

Vamos montar o dashboard em 5 etapas. Cada etapa cria uma peça do painel final.

\bigskip
\textbf{Etapa 1 --- Sheet ``KPIs'' (indicadores-chave)}

\textbf{O que mostrar:} Média geral de Nota\_Exame, total de alunos, taxa de aprovação (>70).

\textbf{Colunas:} \texttt{Nota\_Exame} (Average, Count), campo calculado para aprovação.

Passo a passo:
\begin{enumerate}
  \item Nova sheet > renomeiem para ``KPIs''.
  \item Arrastem \texttt{Measure Names} para \textbf{Columns}.
  \item Arrastem \texttt{Measure Values} para \textbf{Rows} (ou Text).
  \item Filtrem Measure Names: mantenham apenas \texttt{Nota\_Exame}.
  \item Cliquem em \texttt{Nota\_Exame} no cartão Measure Values > \textbf{Average}. Resultado esperado: \textbf{67.7}.
  \item Para total de alunos: arrastem \texttt{Nota\_Exame} novamente para Measure Values > \textbf{Count}. Resultado: \textbf{6607}.
  \item Para taxa de aprovação: menu \textbf{Analysis > Create Calculated Field}: nome ``Taxa\_Aprovacao'', fórmula: \texttt{COUNTD(IF [Nota\_Exame]>70 THEN [Nota\_Exame] END) / COUNTD([Nota\_Exame])}. Arrastem para Measure Values, formatem como porcentagem.
  \item No cartão Marks, troquem para \textbf{Text}. Aumentem a fonte para 36pt (formatem como ``Big Numbers'').
\end{enumerate}

\begin{SolvedBox}
\textbf{O que os KPIs dizem:} Média 67.7 (abaixo de 70 = alerta). 6607 alunos na amostra. ~60\% aprovam (>70). Ou seja: \textbf{40\% dos alunos estão em risco.} Essa é a motivação do dashboard: encontrar o que ajuda esses 40\%.
\end{SolvedBox}

\bigskip
\textbf{Etapa 2 --- Reusar as sheets anteriores}

Já temos 4 sheets criadas na seção anterior:
\begin{itemize}
  \item ``Scatter\_Horas\_vs\_Nota'' --- responde: horas de estudo melhoram notas?
  \item ``Bar\_Envolvimento'' --- responde: envolvimento parental faz diferença?
  \item ``Histograma\_Notas'' --- responde: como as notas estão distribuídas?
  \item ``BoxPlot\_Frequencia'' --- responde: frequência impacta notas?
\end{itemize}

Não precisamos refazê-las. Vamos usá-las diretamente no dashboard.

\bigskip
\textbf{Etapa 3 --- Montar o Dashboard (layout)}

Passo a passo:
\begin{enumerate}
  \item Cliquem no ícone \textbf{``New Dashboard''} (aba inferior, ícone de grade).
  \item Renomeiem para ``Dashboard\_Desempenho\_Academico''.
  \item No painel esquerdo, vejam as sheets disponíveis. Configurem o tamanho: \textbf{Size > Automatic} (se adapta à tela).
  \item \textbf{Topo:} Arrastem ``KPIs'' para a parte superior. Redimensionem para ocupar ~15\% da altura.
  \item \textbf{Centro-esquerda:} Arrastem ``Scatter\_Horas\_vs\_Nota'' abaixo dos KPIs, lado esquerdo (~50\% da largura).
  \item \textbf{Centro-direita:} Arrastem ``Bar\_Envolvimento'' ao lado do scatter (~50\% da largura).
  \item \textbf{Base-esquerda:} Arrastem ``Histograma\_Notas'' abaixo do scatter.
  \item \textbf{Base-direita:} Arrastem ``BoxPlot\_Frequencia'' abaixo do bar chart.
  \item \textbf{Título:} Duplo clique no topo > ``Painel de Desempenho Acadêmico -- AED Unifacol''.
\end{enumerate}

\bigskip
\textbf{Etapa 4 --- Adicionar filtros interativos}

Agora, o mais poderoso: filtros que atualizam TODOS os gráficos simultaneamente.

Passo a passo:
\begin{enumerate}
  \item No painel esquerdo do dashboard, arrastem \texttt{Genero} para uma área vazia > \textbf{Add as Filter}.
  \item Repitam com \texttt{Renda\_Familiar} e \texttt{Tipo\_Escola}.
  \item Para que o filtro afete todas as sheets: cliquem na seta do filtro > \textbf{Apply to Worksheets > All Using This Data Source}.
  \item Posicionem os filtros no canto direito ou topo do dashboard.
\end{enumerate}

\begin{SolvedBox}
\textbf{Interação guiada --- Experimentem agora (façam junto comigo):}

\textbf{Filtro 1 --- Renda\_Familiar = ``Baixo'':} Selecionem apenas ``Baixo'' no filtro. O que muda?
\begin{itemize}
  \item KPIs: média cai de 67.7 para ~62. Taxa de aprovação cai para ~45\%.
  \item Scatter: pontos se concentram mais embaixo.
  \item Bar: envolvimento parental ``Alto'' fica raro neste grupo.
  \item \textbf{Insight:} Alunos de renda baixa têm média 8\% menor. \textbf{Decisão:} Programas de apoio financeiro + tutoria.
\end{itemize}

\textbf{Filtro 2 --- Genero = ``Feminino'':} Limpem o filtro anterior. Selecionem ``Feminino''.
\begin{itemize}
  \item KPIs: média sobe ligeiramente (~69).
  \item Box Plot: mediana feminina mais alta em todas as faixas de frequência.
  \item \textbf{Insight:} Meninas têm desempenho marginalmente melhor. \textbf{Decisão:} Investigar fatores como motivação e hábitos de estudo.
\end{itemize}

\textbf{Filtro 3 --- Tipo\_Escola = ``Pública'' + Renda = ``Baixo'':} Combinem dois filtros.
\begin{itemize}
  \item KPIs: média cai para ~58. Aprovação ~35\%.
  \item \textbf{Insight:} Este é o subgrupo mais vulnerável. \textbf{Decisão:} Prioridade máxima para intervenção.
\end{itemize}
\end{SolvedBox}

\bigskip
\textbf{Etapa 5 --- Ações de drill-down (opcional avançado)}

Para tornar o dashboard ainda mais interativo:
\begin{enumerate}
  \item Menu \textbf{Dashboard > Actions > Add Action > Filter}.
  \item Source: ``Bar\_Envolvimento''. Target: todas as sheets.
  \item Agora, ao clicar em uma barra (ex.: ``Alto''), todos os gráficos filtram automaticamente para mostrar apenas alunos com envolvimento alto.
\end{enumerate}

\begin{BoardBox}
\textbf{Estrutura final do dashboard}
\begin{itemize}
  \item \textbf{Topo:} KPIs --- Média (67.7), Total (6607), Aprovação (60\%). \textbf{Função:} Overview instantâneo para educador.
  \item \textbf{Centro-esquerda:} Scatter Plot (Horas × Nota). \textbf{Função:} Responde ``estudo ajuda?''.
  \item \textbf{Centro-direita:} Bar Chart (Envolvimento × Nota). \textbf{Função:} Responde ``pais importam?''.
  \item \textbf{Base-esquerda:} Histograma (distribuição de notas). \textbf{Função:} Responde ``quantos reprovam?''.
  \item \textbf{Base-direita:} Box Plot (Frequência × Nota). \textbf{Função:} Responde ``presença importa?''.
  \item \textbf{Filtros:} Gênero, Renda, Tipo\_Escola. \textbf{Função:} Segmenta para encontrar subgrupos vulneráveis.
  \item \textbf{Actions:} Clicar em barra filtra tudo. \textbf{Função:} Drill-down sem sair do painel.
\end{itemize}
\textbf{Objetivo final:} Educador abre o dashboard, filtra por escola dele, identifica subgrupo em risco, toma decisão (tutoria, apoio parental, presença).
\end{BoardBox}

\subsubsection{21:15--21:30: Storytelling}

\begin{SolvedBox}
\textbf{Pergunta estratégica do storytelling:} ``Como comunicar a um diretor de escola que frequência é o fator mais importante para melhorar notas --- e convencê-lo a agir?''

\textbf{Por que storytelling?} Dados sozinhos não convencem. Um dashboard mostra informação, mas uma story guia o público por uma narrativa: contexto $\to$ problema $\to$ evidência $\to$ decisão. É assim que dados viram política educacional.

\textbf{Big Idea da nossa story:} ``Frequência é a chave: alunos com presença acima de 80\% têm notas 15\% maiores --- investir em controle de presença é a ação mais eficaz.''
\end{SolvedBox}

Vamos criar uma Story completa com 5 Story Points. Cada ponto responde a uma pergunta e avança a narrativa.

Passo a passo inicial:
\begin{enumerate}
  \item Cliquem no ícone \textbf{``New Story''} (aba inferior, ícone de livro).
  \item Renomeiem para ``Da Frequência à Excelência''.
  \item No painel esquerdo, vocês verão as sheets e dashboards disponíveis.
\end{enumerate}

\bigskip
\textbf{--- Story Point 1: ``O Panorama'' ---}

\textbf{Pergunta:} ``Como estão as notas dos nossos alunos?''

Passo a passo:
\begin{enumerate}
  \item Arrastem a sheet ``Histograma\_Notas'' para a área central do Story.
  \item Na caixa de texto (caption) acima, escrevam: \textbf{``Panorama: A maioria dos 6607 alunos tem notas entre 65-75. Cerca de 25\% estão abaixo de 65 (em risco).''}
  \item Cliquem em \textbf{Add Annotation > Area} e destaquem a região vermelha (<65) com texto: ``Zona de risco: 1 em cada 4 alunos.''
\end{enumerate}

\begin{SolvedBox}
\textbf{Narrativa para o diretor:} ``Diretor, veja esta distribuição. A média é 67.7, mas o problema real está aqui (apontar região vermelha): 25\% dos alunos estão abaixo do mínimo. Precisamos investigar por quê.''
\end{SolvedBox}

\bigskip
\textbf{--- Story Point 2: ``O Fator Determinante'' ---}

\textbf{Pergunta:} ``O que mais influencia as notas?''

Passo a passo:
\begin{enumerate}
  \item Cliquem em \textbf{``Blank''} (novo Story Point).
  \item Arrastem a sheet ``BoxPlot\_Frequencia''.
  \item Caption: \textbf{``A frequência é o fator mais forte: correlação de 0.58 com a nota. Alunos com presença alta têm mediana 15 pontos acima.''}
  \item Annotation com seta: apontem para a diferença entre medianas de ``Baixa'' e ``Alta''.
\end{enumerate}

\begin{SolvedBox}
\textbf{Narrativa:} ``Investigamos todos os fatores. E olhe o que encontramos: frequência é o mais forte. Veja a diferença: alunos que vêm às aulas regularmente (>80\%) têm mediana de 75. Os que faltam muito? 60. São 15 pontos de diferença.''
\end{SolvedBox}

\bigskip
\textbf{--- Story Point 3: ``Confirmando com Outra Perspectiva'' ---}

\textbf{Pergunta:} ``Essa relação se mantém quando olhamos horas de estudo?''

Passo a passo:
\begin{enumerate}
  \item Novo Story Point > arrastem ``Scatter\_Horas\_vs\_Nota''.
  \item Caption: \textbf{``Horas de estudo também importam (correlação 0.45), mas a frequência é mais forte. Alunos frequentes que estudam mais têm os melhores resultados.''}
  \item Annotation: destaquem o cluster superior-direito (alta hora + alta nota).
\end{enumerate}

\begin{SolvedBox}
\textbf{Narrativa:} ``Horas de estudo ajudam, sim, mas sozinhas não bastam. Vejam que a dispersão é grande. Agora, quando combinamos presença alta com estudo, aí temos os melhores resultados (apontar cluster). A presença é a base.''
\end{SolvedBox}

\bigskip
\textbf{--- Story Point 4: ``Os Grupos Vulneráveis'' ---}

\textbf{Pergunta:} ``Quem são os alunos que mais precisam de ajuda?''

Passo a passo:
\begin{enumerate}
  \item Novo Story Point > arrastem o ``Dashboard\_Desempenho\_Academico''.
  \item \textbf{Apliquem filtro:} Renda\_Familiar = ``Baixo'', Tipo\_Escola = ``Pública''.
  \item Caption: \textbf{``O subgrupo mais vulnerável: alunos de escola pública e renda baixa. Média de 58, aprovação de apenas 35\%. Frequência é ainda mais baixa neste grupo.''}
\end{enumerate}

\begin{SolvedBox}
\textbf{Narrativa:} ``Agora, quem são esses alunos em risco? Filtramos por escola pública e renda baixa. Olhe os números: média 58, só 35\% aprovam. E a frequência deles é a mais baixa. É um ciclo: renda baixa $\to$ dificuldade de transporte $\to$ faltam mais $\to$ notas caem.''
\end{SolvedBox}

\bigskip
\textbf{--- Story Point 5: ``A Decisão'' ---}

\textbf{Pergunta:} ``O que fazer com esses dados?''

Passo a passo:
\begin{enumerate}
  \item Novo Story Point > arrastem ``Bar\_Envolvimento'' com filtro Renda=Baixo.
  \item Caption: \textbf{``Recomendação: (1) Programa de auxílio transporte para aumentar frequência em escolas públicas; (2) Engajamento parental para famílias de baixa renda; (3) Tutoria focada no quartil inferior.''}
  \item Adicionem caixa de texto (\textbf{Drag > Text}) com a Big Idea final em destaque.
\end{enumerate}

\begin{SolvedBox}
\textbf{Narrativa final:} ``Diretor, os dados mostram claramente: frequência é a alavanca. Para os alunos de renda baixa, a barreira é prática --- transporte, alimentação. Minha recomendação: (1) auxílio transporte, que pode aumentar presença em 20\%; (2) reuniões trimestrais com pais, que associamos a +10 pontos na nota; (3) tutoria após aula para o quartil inferior. Com essas 3 ações, projetamos melhoria de 15\% na taxa de aprovação.''
\end{SolvedBox}

Perguntem aos alunos: ``Vocês perceberam como a story guiou o diretor de 'temos um problema' até 'eis a solução'? Cada Story Point respondeu uma pergunta e avançou a narrativa. Isso é storytelling com dados.''

\begin{BoardBox}
\textbf{Estrutura da Story: 5 Story Points}
\begin{enumerate}
  \item \textbf{Panorama:} Histograma --- ``Como estão as notas?'' $\to$ 25\% em risco.
  \item \textbf{Fator principal:} Box Plot --- ``O que mais influencia?'' $\to$ Frequência (r=0.58).
  \item \textbf{Confirmação:} Scatter Plot --- ``Horas de estudo confirmam?'' $\to$ Sim, mas frequência é mais forte.
  \item \textbf{Vulneráveis:} Dashboard filtrado --- ``Quem precisa de ajuda?'' $\to$ Escola pública + renda baixa.
  \item \textbf{Decisão:} Bar Chart + texto --- ``O que fazer?'' $\to$ Transporte + pais + tutoria.
\end{enumerate}
\textbf{Big Idea:} ``Frequência é a chave --- investir em presença é a ação mais eficaz para melhorar desempenho acadêmico.''
\end{BoardBox}

\subsection{21:30--22:00: Atividades Práticas e Discussão}
Agora, pratiquem! Criem uma visualização própria. Discutam: O que o dataset revela sobre educação? Como aplicar em projetos reais?

Exercícios: 1. Scatter plot com filtro por Tipo\_Escola. 2. Dashboard com 3 gráficos. 3. Story curta.

Discussão: Correlação alta (Frequencia 0.58) sugere foco em presença. Aplicações: dashboards para escolas monitorarem alunos em risco.

Objetivo final: autonomia em Tableau para AED.

\begin{BoardBox}
\textbf{Exercícios Práticos}
\begin{enumerate}
  \item Criar scatter plot: Horas\_Sono vs. Nota\_Exame, cor por Motivação. \textbf{Por que?} Testa relação numérica; cor revela interações. \textbf{Alternativas:} Bubble chart (tamanho por frequência), dual axis (linhas). \textbf{O que esperamos:} Discussão sobre sono vs. motivação; correlação baixa sugere foco em outros fatores.
  \item Dashboard: Incluir heatmap de correlações (usar table calc). \textbf{Por que ideal?} Mostra múltiplas relações simultaneamente. \textbf{Alternativas:} Correlation matrix (tabela), network graphs (dependências). \textbf{O que esperamos:} Identificação de fatores chave (ex.: Frequencia alta correlação).
  \item Story: "Impacto da renda na educação". \textbf{Por que?} Explora desigualdade social. \textbf{Alternativas:} Infográficos (estáticos), presentations (PowerPoint). \textbf{O que esperamos:} Narrativa com decisão (ex.: programas de apoio a baixa renda).
\end{enumerate}
\textbf{Discussão}: Insights aplicáveis (ex.: tutoria para baixa renda). \textbf{Objetivo:} Autonomia; esperamos alunos aplicando AED a problemas reais.
\end{BoardBox}

\section{Materiais Preparados}
\begin{itemize}[leftmargin=*]
  \item \textbf{Dataset:} \texttt{tableau/dataset/student\_academic\_performance/StudentPerformanceFactors\_transformado.csv} (preparado com transformações em Python: traduções, features novas, normalização).
  \item \textbf{Contexto:} \texttt{tableau/dataset/student\_academic\_performance/context.md} (guia completo com metadados, perguntas estratégicas, protótipo de dashboard – leiam para entender o dataset).
  \item \textbf{Notebook de Preparação:} \texttt{tableau/dataset/python/preparacao\_tableau.ipynb} (ETL em Python com comentários: imports, load, clean, transform, save – execute para ver passos).
  \item \textbf{Roteiro da Aula:} \texttt{tableau/roteiro\_aula\_tableau.md} (passos detalhados para a sessão prática – use como checklist).
  \item \textbf{Perguntas:} \texttt{tableau/perguntas\_tableau.md} (lista para guiar análises – inspirem-se para suas visualizações).
  \item \textbf{Ambiente:} Pasta \texttt{tableau/dataset/python/} com README (instalação venv), requirements.txt (pandas, numpy, scikit-learn).
\end{itemize}

Por que esses materiais? Dataset limpo acelera aula; contexto explica fundo; notebook ensina ETL; roteiro guia prática; perguntas estimulam criatividade.

\section{Entrega (Trabalho 1 --- 0,5 ponto na I unidade)}
\begin{SolvedBox}
\textbf{Formato:} cada aluno tem \textbf{10 minutos} para construir um infográfico/visual (com base no tema escolhido na Semana 01) e explicar para a turma.

\textbf{Artefatos:}
\begin{itemize}[leftmargin=*]
  \item 1 dataset (CSV) escolhido dentre os 20 temas;
  \item 1 ETL básico em Python (notebook) com pelo menos 1 limpeza real;
  \item 1 infográfico no Tableau (ou 1 dashboard simples);
  \item 1 slide/texto com \textbf{Big Idea} + 3 bullets de evidência.
\end{itemize}

\textbf{Critério:} clareza (Teste do Relance), coerência do recorte e conclusão ligada a decisão.
\end{SolvedBox}

\section{Apresentações e Interações (30 min)}
Durante as apresentações, observem: o infográfico passa no "teste do relance" (entende-se em 5 segundos)? A Big Idea é acionável? Os bullets apoiam com evidências quantitativas? Perguntem: "Alguém teve dificuldade em conectar o dataset no Tableau?" ou "Como vocês escolheram as cores?"

\begin{BoardBox}
\textbf{Critérios de Avaliação do Trabalho 1}
\begin{itemize}
  \item \textbf{Clareza}: Visual limpo, cores adequadas, sem clutter (desordem).
  \item \textbf{Coerência}: Recorte lógico do tema, dados relevantes.
  \item \textbf{Decisão}: Big Idea leva a ação (ex.: "Aumentar investimento em X para melhorar Y").
  \item \textbf{Técnico}: ETL real (ex.: fillna, drop duplicates), visualização correta.
  \item \textbf{Perguntas para alunos}: "O que foi mais desafiador no ETL?" "Como o Tableau ajudou na Big Idea?"
\end{itemize}
\end{BoardBox}

Se houver tempo, discutam: "Como o Tableau se compara ao Excel para visualização?" ou "Quais visualizações vocês acham mais impactantes?"

\begin{BoardBox}
\textbf{Troubleshooting Comum}
\begin{itemize}
  \item \textbf{Erro de conexão}: "Tableau não conecta ao CSV" $\to$ Verificar caminho absoluto, encoding='utf-8'.
  \item \textbf{Dados não aparecem}: "Colunas em branco" $\to$ Usar df.info() no Jupyter para verificar tipos.
  \item \textbf{Visualização errada}: "Gráfico não faz sentido" $\to$ Verificar Measures vs Dimensions (ex.: Idade como Dimension, Nota como Measure).
  \item \textbf{Performance}: "Tableau lento" $\to$ Filtrar dados no ETL (ex.: df = df[df['Ano'] > 2020]).
  \item \textbf{Cores}: "Paleta não combina" $\to$ Usar paletas acessíveis (ex.: colorblind-friendly).
\end{itemize}
\end{BoardBox}

\textbf{Fechamento (10 min):} Revisem os aprendizados: ETL com Python (preparação de dados), conexão no Tableau (importação), visualizações básicas (gráficos interativos), storytelling (comunicação). Anunciem: "Próxima aula: Dashboards avançados e filtros." Peçam feedback: "O que gostaram mais?" (esperamos menções a interatividade); "O que melhorar?" (esperamos sugestões para mais prática). \textbf{Por que feedback?} Melhora aulas futuras; definido para avaliar engajamento e compreensão.

\section{Próximo passo}
Excelente aula! Agora conectem visualização com dados retangulares. Reflitam sobre storytelling para a Prova I Unidade. Até semana que vem!

Na próxima aula, vamos aprofundar dados retangulares: tipos, ausências, duplicadas. Usem o que aprenderam hoje (ETL em Python, visualizações no Tableau) para preparar Trabalhos futuros. O objetivo final é dominar AED de ponta a ponta: da pergunta à decisão baseada em dados.
