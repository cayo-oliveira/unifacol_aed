\documentclass[12pt,openany]{book}


\usepackage[a4paper,margin=2.5cm]{geometry}
\usepackage[T1]{fontenc}
\usepackage[utf8]{inputenc}
\usepackage[brazil]{babel}
\usepackage{lmodern}
\usepackage{microtype}
\usepackage{amsmath, amssymb, amsthm}
\usepackage{booktabs}
\usepackage{graphicx}
\usepackage{xcolor}
\usepackage{hyperref}
\usepackage{enumitem}
\usepackage{array}
\usepackage{float}
\usepackage{caption}
\usepackage{listings}
\usepackage{siunitx}
\usepackage[most]{tcolorbox}

\usepackage{appendix}


\hypersetup{colorlinks=true,linkcolor=blue,urlcolor=blue,citecolor=teal}

\usepackage{tikz}
\usetikzlibrary{shapes.geometric, arrows}

\tikzstyle{startstop} = [rectangle, rounded corners, minimum width=3cm, minimum height=1cm,text centered, draw=black, fill=pastelblue]
\tikzstyle{arrow} = [thick,->,>=stealth]

% ---------- Colors (pastel palette) ----------
\definecolor{pastelblue}{HTML}{E6F0FF}    % light blue
\definecolor{pastelgray}{HTML}{F2F2F2}    % very light gray
\definecolor{pastelorange}{HTML}{FFEBD6}  % light orange
\definecolor{pastelteal}{HTML}{DFF7F2}    % accent (optional)
% added to match Tópicos palette
\definecolor{pastelgreen}{HTML}{E8F5E9}   % green - atenção software
\definecolor{pastelyellow}{HTML}{FFF9E6}  % yellow - dicas

% ---------- tcolorbox styles ----------
\tcbset{sharp corners, boxrule=0pt, coltitle=black, fonttitle=\bfseries, left=8pt, right=8pt, top=6pt, bottom=6pt}

\newtcolorbox{FormulaBox}[1][]{colback=pastelblue, coltitle=white, title=Fórmula, #1}
\newtcolorbox{ProofBox}[1][]{colback=pastelgray, coltitle=white, title=Demonstração (Resumo), #1}
\newtcolorbox{SolvedBox}[1][]{colback=pastelorange, coltitle=white, title=Questão ENADE Resolvida, #1}
\newtcolorbox{NoteBox}[1][]{colback=pastelteal, coltitle=white, title=Nota, #1}

% Attention and Tip boxes (aligned with Tópicos)
\newtcolorbox{AttentionBox}[1][]{colback=pastelgreen, coltitle=white, title=Atenção para o Software, #1}
\newtcolorbox{TipBox}[1][]{colback=pastelyellow, coltitle=white, title=Dica, #1}

% Common box separator macro
\newcommand{\BoxSeparator}{\vspace{0.5em}\hrule\vspace{0.5em}}


% ---------- Listings (Python & SQL) ----------
\lstdefinestyle{codeblock}{
  basicstyle=\ttfamily\small,
  numbers=left,
  numberstyle=\tiny,
  stepnumber=1,
  numbersep=6pt,
  showstringspaces=false,
  breaklines=true,
  frame=single,
  framerule=0.2pt,
  framesep=6pt,
  rulecolor=\color{pastelgray},
  xleftmargin=1em,
  xrightmargin=1em
}
\lstdefinelanguage{SQL}{morekeywords={SELECT,FROM,WHERE,AND,OR,JOIN,LEFT,RIGHT,INNER,OUTER,ON,AS,GROUP,BY,ORDER,INSERT,INTO,VALUES,UPDATE,DELETE,CREATE,TABLE,VIEW,DROP,ALTER,PRIMARY,KEY,NOT,NULL,DEFAULT,DISTINCT}, sensitive=false}
\lstset{style=codeblock}

% ---------- Common macros ----------
\newcommand{\mean}{\bar{x}}
\newcommand{\var}{\mathrm{Var}}
\newcommand{\sd}{\mathrm{DP}}
\newcommand{\cov}{\mathrm{Cov}}
\newcommand{\corr}{\mathrm{Corr}}
\newcommand{\given}{\;\middle|\;}

\title{\Huge \textbf{Análise Exploratória de Dados}\\ \Large Um Livro-Guia Prático com Exercícios e Soluções}
\author{Cayo Oliveira (org.)}
\date{2025}

\begin{document}
\maketitle

\tableofcontents



\input{chapters/00_plano.tex}
\chapter{O que é AED?}

A \textbf{Análise Exploratória de Dados (AED)} surgiu como um marco na estatística moderna graças a John W. Tukey, em seu artigo clássico \textit{Exploratory Data Analysis} publicado em 1977. Tukey defendia que, antes de qualquer modelagem estatística ou preditiva, o pesquisador deveria “deixar os dados falarem”, ou seja, explorar padrões, tendências e anomalias sem pressupor um modelo rígido. Essa mudança de paradigma mostrou que a estatística não era apenas confirmatória (testar hipóteses), mas também exploratória, voltada à descoberta.

\section{Importância no Mercado de Trabalho e na Pesquisa}

No contexto atual de ciência de dados e inteligência artificial, a AED tornou-se uma etapa obrigatória em pipelines de análise. Empresas de todos os setores — tecnologia, finanças, saúde, varejo e governo — reconhecem que análises mal preparadas levam a modelos enviesados e decisões equivocadas. Uma AED bem feita garante qualidade, transparência e confiabilidade. 

Além do mercado corporativo, a AED é igualmente fundamental em \textbf{pesquisas empresariais, acadêmicas e científicas}. Pesquisadores utilizam AED para:
\begin{itemize}[leftmargin=*]
  \item \textbf{Gerar bases de dados para estudos}, a partir de coleta em questionários, sensores, bases públicas ou experimentos.
  \item \textbf{Verificar hipóteses preliminares}, testando padrões antes da aplicação de métodos confirmatórios.
  \item \textbf{Construir reprodutibilidade}, documentando os passos de preparação, limpeza e análise para garantir que outros pesquisadores possam validar os resultados.
\end{itemize}

\textbf{Quem pede e quem faz a AED?}
\begin{itemize}[leftmargin=*]
  \item \textbf{Solicitantes}: 
    \begin{itemize}
      \item No setor corporativo: gestores de negócio, áreas de produto, marketing, finanças ou líderes técnicos, que precisam de respostas claras para apoiar decisões estratégicas.
      \item No meio acadêmico: orientadores de pesquisa, grupos de estudo e órgãos de fomento que demandam análises preliminares de dados coletados.
    \end{itemize}
  \item \textbf{Executores}: 
    \begin{itemize}
      \item No setor corporativo: analistas de dados, cientistas de dados, engenheiros de dados e engenheiros de analytics, que possuem habilidades estatísticas, de programação e de comunicação.
      \item No meio acadêmico: estudantes de graduação e pós-graduação, professores e pesquisadores com formação quantitativa.
    \end{itemize}
\end{itemize}

Assim, a AED funciona como elo entre \textbf{quem define a pergunta} (negócio ou pesquisa) e \textbf{quem manipula os dados} (técnicos ou acadêmicos), garantindo que a resposta seja confiável, interpretável e útil.


\section{Fluxo Principal da AED}
O processo da AED pode ser descrito em quatro grandes etapas interligadas:

\begin{figure}[H]
\centering
\begin{tikzpicture}[node distance=2cm]

\node (A) [startstop] {Leitura da Base};
\node (B) [startstop, below of=A] {Transformação de Dados};
\node (C) [startstop, below of=B] {Geração de Informação};
\node (D) [startstop, below of=C] {Insight};

\draw [arrow] (A) -- (B);
\draw [arrow] (B) -- (C);
\draw [arrow] (C) -- (D);

\end{tikzpicture}
\caption{Fluxo principal da Análise Exploratória de Dados (AED).}
\end{figure}

\subsection*{1. Leitura da base}
Consiste em importar os dados de suas fontes originais:
\begin{itemize}[leftmargin=*]
  \item Bancos de dados relacionais (SQL, PostgreSQL, MySQL).
  \item Arquivos tabulares (\texttt{CSV}, Excel).
  \item Estruturas semi-estruturadas (\texttt{JSON}, XML).
  \item APIs, logs de sistemas ou planilhas corporativas.
\end{itemize}
Nessa fase, o analista garante que a base foi carregada corretamente e que a estrutura de linhas e colunas é consistente.

\subsection*{2. Transformação de dados}
É a etapa de preparar a base para análise, também chamada de \textit{data wrangling}. Inclui:
\begin{itemize}[leftmargin=*]
  \item Tratamento de valores nulos e ausentes (remoção, substituição, imputação).
  \item Eliminação de duplicidades.
  \item Padronização de formatos (datas, moedas, códigos).
  \item Criação de variáveis derivadas (ex.: idade a partir de data de nascimento).
  \item Normalização e escalonamento de variáveis numéricas.
\end{itemize}
O objetivo é obter um conjunto de dados limpo, coerente e pronto para a geração de informação.

\subsection*{3. Geração de informação}
Nessa fase, aplicam-se métodos estatísticos e gráficos para compreender os dados:
\begin{itemize}[leftmargin=*]
  \item \textbf{Análise univariada}: estudo de uma variável isolada (médias, medianas, histogramas, boxplots).
  \item \textbf{Análise bivariada}: relação entre duas variáveis (correlação, covariância, gráficos de dispersão, tabelas de contingência).
  \item \textbf{Análise multivariada}: estudo conjunto de várias variáveis (\textit{odds ratio}, matriz de correlação, regressão múltipla, análise de clusters, PCA).
\end{itemize}
Essa etapa transforma dados brutos em informações interpretáveis.

\subsection*{4. Insight}

O \textbf{insight} é a etapa final da Análise Exploratória de Dados, onde os achados estatísticos e visuais são traduzidos em mensagens de valor. Ele representa a ponte entre os \textit{números} e as \textit{decisões}. 

Um insight é mais do que apresentar uma métrica ou gráfico: é a capacidade de interpretar o que aquela informação significa em um contexto prático e de sugerir caminhos de ação. Essa etapa exige não apenas conhecimento técnico, mas também compreensão do domínio do problema (negócios, saúde, educação, finanças, etc.).

\paragraph{Características de um bom insight:}
\begin{itemize}[leftmargin=*]
  \item \textbf{Clareza}: deve ser simples e compreensível para públicos não técnicos.
  \item \textbf{Relevância}: deve responder a uma pergunta real do negócio ou da pesquisa.
  \item \textbf{Acionabilidade}: precisa apontar um caminho prático, uma ação possível.
  \item \textbf{Evidência}: deve estar embasado em dados e análises confiáveis.
\end{itemize}

\paragraph{Insight em contextos diferentes:}
\begin{itemize}[leftmargin=*]
  \item \textbf{Negócios}: orientar campanhas de marketing, prever comportamento de clientes ou reduzir custos.
  \item \textbf{Educação}: identificar padrões de desempenho e sugerir estratégias de ensino personalizadas.
  \item \textbf{Saúde}: apontar fatores de risco em pacientes e apoiar protocolos de prevenção.
  \item \textbf{Pesquisa científica}: validar hipóteses preliminares e direcionar futuras investigações.
\end{itemize}

\begin{NoteBox}
\textbf{Exemplo de Insight:}  
Em uma empresa de educação, a AED mostrou que alunos com mais de 7 horas de estudo semanal tinham o dobro de chances de alcançar nota acima de 8. Esse resultado pode orientar políticas de incentivo, criação de trilhas de estudo e comunicação personalizada para melhorar o desempenho médio da turma.

\medskip

\textbf{Exemplo em Negócios:}  
Uma rede de supermercados identificou, por meio da AED, que clientes que compravam frutas frescas semanalmente também tinham alta probabilidade de adquirir produtos de panificação. O insight gerou a decisão de posicionar pães e bolos próximos ao setor de hortifrúti, aumentando as vendas cruzadas em 15\%.
\end{NoteBox}

O insight é o momento em que os dados “ganham voz”. É quando a AED deixa de ser apenas uma análise técnica e passa a influenciar diretamente decisões estratégicas, acadêmicas ou operacionais.


\section{Visão Geral da AED}

A Análise Exploratória de Dados pode ser entendida como um conjunto de práticas que conectam três dimensões principais:  
(i) \textit{estatística descritiva}, para resumir e organizar os dados;  
(ii) \textit{geração de insights}, para apoiar decisões práticas;  
(iii) \textit{preparação para modelos preditivos e inteligência artificial}, garantindo que os algoritmos recebam dados consistentes e representativos.

\subsection*{Análises e Escopos}
De forma geral, podemos dividir a geração de informação da AED em três níveis de análise, que serão explorados em detalhe nos próximos capítulos, mas que já merecem uma introdução:

\begin{itemize}[leftmargin=*]
  \item \textbf{Análise Univariada}: estuda uma variável isoladamente. Exemplos de gráficos: histogramas, boxplots e barras. Principais medidas: média, mediana, moda, variância e desvio-padrão.
  
  \item \textbf{Análise Bivariada}: avalia a relação entre duas variáveis, que podem ser quantitativas ou qualitativas. Exemplos de gráficos: dispersão (scatterplot), gráficos de barras agrupadas e tabelas de contingência. Principais medidas: covariância e correlação de Pearson.
  
  \item \textbf{Análise Multivariada}: considera três ou mais variáveis em conjunto, explorando interações mais complexas. Exemplos: \textit{odds ratio}, matriz de correlação, análise de clusters, regressão múltipla, PCA (análise de componentes principais). Medidas comuns incluem correlações múltiplas e medidas de associação.
\end{itemize}

\begin{FormulaBox}
\textbf{Resumo de medidas-chave (amostra):}\\[4pt]
Média: $\bar{x}=\frac{1}{n}\sum_{i=1}^{n}x_i$;\quad
Mediana: valor central dos dados ordenados;\\[4pt]
Variância: $s^2=\frac{1}{n-1}\sum (x_i-\bar{x})^2$;\quad
Desvio-padrão: $s=\sqrt{s^2}$.\\[4pt]
Covariância: $\mathrm{Cov}(X,Y)=\frac{1}{n-1}\sum (x_i-\bar{x})(y_i-\bar{y})$;\\[4pt]
Correlação de Pearson: $r=\dfrac{\mathrm{Cov}(X,Y)}{s_X s_Y}\in[-1,1]$.\\[4pt]
Odds: $\mathrm{Odds}(A)=\frac{P(A)}{1-P(A)}$;\quad
Odds Ratio (A vs. B): $\mathrm{OR}=\frac{\mathrm{Odds}_A}{\mathrm{Odds}_B}$.
\end{FormulaBox}

Nesta visão geral, o objetivo não é aprofundar cálculos, mas apresentar o \textbf{vocabulário essencial da AED} — medidas, fórmulas e tipos de gráfico mais comuns. Nos próximos capítulos, cada tipo de análise será detalhado com exemplos, interpretações e problemas resolvidos.

\section{Ferramentas Modernas}

A prática da Análise Exploratória de Dados conta hoje com um ecossistema robusto de ferramentas, que vão desde linguagens de programação até softwares de visualização interativa. Cada escolha depende do perfil da equipe, do volume de dados e do objetivo da análise. 

\subsection*{Linguagens e Bibliotecas}
\begin{itemize}[leftmargin=*]
  \item \textbf{Python}: linguagem mais difundida em ciência de dados. As bibliotecas \texttt{pandas} (manipulação de tabelas), \texttt{matplotlib} e \texttt{plotly} (visualização estática e interativa) tornam o fluxo da AED ágil e reprodutível.
  \item \textbf{R}: muito usada no meio acadêmico, com pacotes como \texttt{ggplot2} (visualização) e \texttt{dplyr} (transformação de dados).
  \item \textbf{SQL}: essencial para extrair dados diretamente de bancos relacionais (MySQL, PostgreSQL, SQL Server), sendo a primeira etapa de muitas AEDs corporativas.
\end{itemize}

\subsection*{Ferramentas de Visualização e BI}
\begin{itemize}[leftmargin=*]
  \item \textbf{Tableau} e \textbf{Power BI}: permitem criar dashboards interativos, explorando univariada, bivariada e multivariada de forma visual.
  \item \textbf{Amazon QuickSight}: solução de BI em nuvem, integrada a serviços AWS, usada em contextos corporativos de larga escala.
  \item \textbf{Google Looker Studio}: alternativa gratuita para integração rápida de dados e relatórios.
\end{itemize}

\subsection*{Contextos de Uso}
\begin{itemize}[leftmargin=*]
  \item \textbf{Empresas}: Python/SQL para limpeza, BI para comunicação rápida com gestores.
  \item \textbf{Academia}: R/Python para geração de bases e análises reprodutíveis.
  \item \textbf{Mercado de dados em nuvem}: integração com BigQuery, Redshift ou Snowflake, conectados a ferramentas de BI.
\end{itemize}

\begin{NoteBox}
\textbf{Boas práticas:}  
\begin{enumerate}[leftmargin=*]
  \item Versionar notebooks e scripts (Git/GitHub ou GitLab).  
  \item Separar dados brutos, processados e finais em diretórios distintos.  
  \item Documentar cada transformação aplicada (log ou README).  
  \item Preferir visualizações interativas sempre que possível para explorar diferentes perspectivas.  
\end{enumerate}
\end{NoteBox}



\section{Problema Resolvido — Questão 1 (conceitos)}
\begin{SolvedBox}
\textbf{Solicitação (resumo)}: definir AED, análise univariada, bivariada e multivariada.\\[4pt]
\textbf{Resposta curta}:\\
AED: etapa exploratória para entender dados antes de modelar.\\
Univariada: uma variável por vez (médias, histogramas, boxplots).\\
Bivariada: relação entre duas variáveis (scatter, tabela de contingência, correlação).\\
Multivariada: três ou mais variáveis (matriz de correlação, PCA, regressão múltipla, clusters).
\end{SolvedBox}

\section*{Problemas Sugeridos}
\begin{enumerate}[leftmargin=*]
  \item \textbf{Problema 1:} Dê três exemplos práticos (do mercado ou da academia) onde a Análise Univariada é suficiente para gerar um insight útil. Explique por que não seria necessário avançar para análises bivariadas ou multivariadas nesses casos.
  
  \item \textbf{Problema 2:} Imagine que você recebeu uma base de dados de clientes contendo idade, renda e gasto mensal. Descreva como faria uma análise univariada, depois bivariada e, por fim, multivariada desses dados. Indique quais gráficos e medidas utilizaria em cada etapa.
\end{enumerate}




\begin{appendices}

\chapter*{Apêndice do Capítulo 1 — Mini-Guia de Python e SQL para AED}
\addcontentsline{toc}{chapter}{Apêndice do Capítulo 1 — Mini-Guia de Python e SQL para AED}

Este apêndice funciona como um guia rápido das principais operações usadas em Análise Exploratória de Dados (AED).  
Ele reúne exemplos práticos em \textbf{Python (pandas)} e \textbf{SQL}, focados em tarefas recorrentes de leitura, transformação, limpeza e preparação dos dados.

\section*{Conceito Estatístico Importante}
\begin{ProofBox}
\textbf{Por que usar $n-1$ na variância amostral?}\\[4pt]
O uso de $n-1$ corrige o viés do estimador, tornando $s^2$ um estimador não-viesado de $\sigma^2$ sob hipóteses clássicas (correção de Bessel).
\end{ProofBox}

\section{Mini-Guia em Python (pandas)}

\subsection*{Leitura e Escrita}
\begin{lstlisting}[language=Python,caption={Carregar e salvar dados}]
import pandas as pd

# Ler arquivos
df_csv = pd.read_csv("dados.csv")
df_excel = pd.read_excel("dados.xlsx")
df_json = pd.read_json("dados.json")

# Salvar dados
df_csv.to_csv("saida.csv", index=False)
df_excel.to_excel("saida.xlsx", index=False)
\end{lstlisting}

\subsection*{Exploração Inicial}
\begin{lstlisting}[language=Python,caption={Inspeção rápida de dados}]
df.head()        # primeiras linhas
df.info()        # tipos de dados e nulos
df.describe()    # estatísticas descritivas
df.shape         # linhas x colunas
df.columns       # nomes das colunas
\end{lstlisting}

\subsection*{Limpeza e Transformação}
\begin{lstlisting}[language=Python,caption={Tratamento básico de dados}]
# Remover duplicatas
df = df.drop_duplicates()

# Tratar valores nulos
df["idade"].fillna(df["idade"].median(), inplace=True)

# Renomear colunas
df.rename(columns={"nome_antigo": "nome_novo"}, inplace=True)

# Alterar tipos
df = df.astype({"idade": "Int64"})

# Criar coluna derivada
df["idade_categoria"] = pd.cut(df["idade"], bins=[0,18,30,60,100],
                               labels=["jovem","adulto","meia-idade","idoso"])
\end{lstlisting}

\subsection*{Filtros e Seleção}
\begin{lstlisting}[language=Python,caption={Selecionar e filtrar dados}]
# Seleção de colunas
df[["nome", "idade"]]

# Filtro por condição
df_maiores = df[df["idade"] >= 18]

# Query estilo SQL
df.query("salario > 5000 and cidade == 'Recife'")
\end{lstlisting}

\subsection*{Agrupamento e Resumos}
\begin{lstlisting}[language=Python,caption={Agrupar e sumarizar}]
# Média de idade por cidade
df.groupby("cidade")["idade"].mean()

# Contagem por categoria
df["cidade"].value_counts()
\end{lstlisting}

---

\section{Mini-Guia em SQL (ANSI)}

\subsection*{Leitura e Criação}
\begin{lstlisting}[language=SQL,caption={Criar e importar tabelas}]
-- Criar tabela simples
CREATE TABLE alunos (
  id INT PRIMARY KEY,
  nome VARCHAR(100),
  idade INT,
  cidade VARCHAR(50)
);

-- Inserir dados
INSERT INTO alunos (id, nome, idade, cidade)
VALUES (1, 'Maria', 22, 'Recife');
\end{lstlisting}

\subsection*{Seleção e Filtros}
\begin{lstlisting}[language=SQL,caption={Consultas básicas}]
-- Selecionar colunas
SELECT nome, idade FROM alunos;

-- Filtro com condições
SELECT * FROM alunos
WHERE idade >= 18 AND cidade = 'Recife';

-- Ordenação
SELECT * FROM alunos
ORDER BY idade DESC;
\end{lstlisting}

\subsection*{Transformação e Tratamento}
\begin{lstlisting}[language=SQL,caption={Limpeza e transformação}]
-- Remover duplicatas usando CTE
WITH cte AS (
  SELECT *, ROW_NUMBER() OVER (PARTITION BY nome ORDER BY id) AS rn
  FROM alunos
)
DELETE FROM cte WHERE rn > 1;

-- Substituir valores nulos
UPDATE alunos SET cidade = COALESCE(cidade, 'Indefinida');
\end{lstlisting}

\subsection*{Agregações e Agrupamentos}
\begin{lstlisting}[language=SQL,caption={Estatísticas descritivas}]
-- Média de idade por cidade
SELECT cidade, AVG(idade) AS media_idade
FROM alunos
GROUP BY cidade;

-- Contagem por categoria
SELECT cidade, COUNT(*) AS qtd
FROM alunos
GROUP BY cidade;
\end{lstlisting}

\subsection*{Integração e Visualização}
\begin{lstlisting}[language=SQL,caption={Junções entre tabelas}]
-- Join entre duas tabelas
SELECT a.nome, a.idade, c.curso
FROM alunos a
JOIN cursos c ON a.id = c.aluno_id;
\end{lstlisting}
\end{appendices}



\chapter{Análise Univariada}

\section{Introdução}
A \textbf{análise univariada} é o estudo de uma variável por vez, com o objetivo de compreender sua distribuição, valores típicos e dispersão.  
Ela é a base da \textbf{estatística descritiva}, responsável por organizar, resumir e apresentar dados de forma clara.  

De modo geral, a \textbf{estatística} se divide em duas grandes áreas:  
\begin{itemize}[leftmargin=*]
  \item \textbf{Estatística descritiva}: coleta, organização, resumo e apresentação de dados;  
  \item \textbf{Estatística inferencial}: técnicas que permitem generalizar conclusões de uma amostra para a população (ex.: testes de hipóteses, intervalos de confiança).  
\end{itemize}

Na AED, a estatística descritiva é essencial para “deixar os dados falarem” antes da aplicação de modelos mais sofisticados.

\section{Conjuntos e Tipos de Variáveis}
Antes de calcular medidas, é preciso reconhecer o tipo de variável:  
\begin{itemize}[leftmargin=*]
  \item \textbf{Qualitativas}: categóricas, não numéricas (ex.: sexo, estado civil, cidade);  
  \item \textbf{Quantitativas discretas}: assumem valores inteiros contáveis (ex.: número de filhos, horas de estudo);  
  \item \textbf{Quantitativas contínuas}: assumem valores em intervalos reais (ex.: altura, renda, temperatura).  
\end{itemize}

Exemplo de conjunto de dados (adaptado da prova):  

\begin{center}
\begin{tabular}{@{}lccccc@{}}\toprule
Aluno & A & B & C & D & E \\\midrule
Horas de Estudo & 8 & 4 & 2 & 1 & 10 \\
Nota Final      & 8 & 6 & 2 & 1 & 10 \\\bottomrule
\end{tabular}
\end{center}

\section{Medidas de Posição}

\subsection*{Média}
\begin{FormulaBox}
Média aritmética: $\quad \bar{x} = \frac{1}{n}\sum_{i=1}^{n}x_i$
\end{FormulaBox}

Exemplo (Notas):  
\[
\bar{y} = \frac{8+6+2+1+10}{5} = \frac{27}{5} = 5{,}4
\]

\subsection*{Mediana}
\begin{FormulaBox}
Mediana ($\tilde{x}$): valor central dos dados ordenados.  
Se $n$ é ímpar, $\tilde{x}$ é o valor central; se $n$ é par, é a média dos dois centrais.
\end{FormulaBox}

Exemplo (Horas): $\{1,2,4,8,10\}$ → $\tilde{x}=4$.

\subsection*{Moda}
\begin{FormulaBox}
Moda: valor que mais se repete no conjunto.  
Pode não existir (sem repetições) ou haver mais de uma moda (bimodal, multimodal).
\end{FormulaBox}

Exemplo: $\{2,2,4,5,6\}$ → moda = 2.

\paragraph{Média vs Mediana}
A média é sensível a valores extremos, enquanto a mediana é robusta a outliers.  OU seja, se a diferença entre média e mediana for grande, é um alto indício de existir outliers na base.
\begin{itemize}
  \item Dados: $\{10,12,14,15,500\}$  
  \item Média = 110,2 (influenciada pelo 500)  
  \item Mediana = 14 (representa melhor o centro dos dados)  
\end{itemize}

\section{Medidas de Dispersão}

\subsection*{Variância}
\begin{FormulaBox}
Variância amostral: $\quad s^2 = \frac{1}{n-1}\sum_{i=1}^{n}(x_i - \bar{x})^2$
\end{FormulaBox}

\subsection*{Desvio Padrão}
\begin{FormulaBox}
Desvio-padrão: $\quad s = \sqrt{s^2}$
\end{FormulaBox}

Exemplo (Notas):  
$s_y^2 = 14{,}8$; \quad $s_y \approx 3{,}85$

\section{Boxplot e Identificação de Outliers}

O \textbf{boxplot} é um gráfico que resume a distribuição de uma variável numérica utilizando \textbf{quartis} e limites de dispersão.  
Ele é muito útil para identificar a mediana, a variabilidade e a presença de \textbf{outliers}.

\begin{figure}[H]
\centering
\begin{tikzpicture}
  % Caixa
  \draw[thick] (0,0) rectangle (4,2); % Q1 até Q3
  % Mediana
  \draw[thick] (2,0) -- (2,2);
  % Whiskers
  \draw[thick] (0,1) -- (-1,1);
  \draw[thick] (4,1) -- (5,1);
  \draw[-] (-1,1) -- (0,1);
  \draw[-] (4,1) -- (5,1);
  % Legendas
  \node[below] at (0,0) {$Q_1$};
  \node[below] at (2,0) {$Q_2$};
  \node[below] at (4,0) {$Q_3$};
  \node[below] at (-1,0.8) {Mínimo};
  \node[below] at (5,0.8) {Máximo};
\end{tikzpicture}
\caption{Representação esquemática de um boxplot.}
\end{figure}

\subsection*{Passo a Passo para Construir um Boxplot}
\begin{enumerate}[leftmargin=*]
  \item \textbf{Calcular $Q_1$, $Q_2$ e $Q_3$}:  
  - $Q_1$ (1º quartil) é o valor abaixo do qual estão 25\% dos dados.  
  - $Q_2$ (mediana) é o valor central (50\%).  
  - $Q_3$ (3º quartil) é o valor abaixo do qual estão 75\% dos dados.  
  *(Os detalhes de cálculo dos percentis estarão no Apêndice do capítulo).*
  
  \item \textbf{Encontrar os valores mínimo e máximo gerais da amostra}.  

  \item \textbf{Calcular o intervalo interquartil (IQR)}:
  \begin{FormulaBox}
  $IQR = Q_3 - Q_1$
  \end{FormulaBox}

  \item \textbf{Determinar limites inferior e superior usando o IQR}:
  \begin{FormulaBox}
  $\text{Mín\_IQR} = Q_1 - 1,5 \times IQR$ \\[4pt]
  $\text{Máx\_IQR} = Q_3 + 1,5 \times IQR$
  \end{FormulaBox}

  \item \textbf{Calcular os whiskers}:  
  - O \textbf{mínimo do boxplot} será o maior valor entre o mínimo geral e $\text{Mín\_IQR}$.  
  - O \textbf{máximo do boxplot} será o menor valor entre o máximo geral e $\text{Máx\_IQR}$.  
  Valores além desses limites são representados como \textbf{outliers}.
\end{enumerate}

Esse processo garante que a caixa represente a maior parte dos dados (50\% entre $Q_1$ e $Q_3$), mas que valores muito extremos não distorçam a visualização.  
Assim, o boxplot é uma ferramenta robusta contra outliers, diferentemente da média e do desvio-padrão.


\section{Problemas Resolvidos — Questões 2 e 3}

\subsection*{Questão 2 — Cálculo de média, mediana, variância e desvio-padrão}

\begin{SolvedBox}
\textbf{Contexto.} Uma equipe coletou \emph{Horas de estudo} ($x$) e \emph{Notas finais} ($y$) de 5 participantes. Queremos: (i) calcular as \textbf{médias} de $x$ e $y$; (ii) as \textbf{medianas}; (iii) a \textbf{variância amostral} e o \textbf{desvio-padrão} de cada variável.

\textbf{Passo 1 — Tabela de dados}\\[-6pt]
\begin{center}
\begin{tabular}{@{}lccccc@{}}\toprule
Registro & R1 & R2 & R3 & R4 & R5 \\\midrule
Horas ($x$) & 8 & 4 & 2 & 1 & 10 \\
Notas ($y$) & 8 & 6 & 2 & 1 & 10 \\\bottomrule
\end{tabular}
\end{center}

\textbf{Passo 2 — Cálculo das médias}

\begin{FormulaBox}
Média: $\quad \bar{x} = \dfrac{\sum x_i}{n}, \qquad \bar{y} = \dfrac{\sum y_i}{n}$
\end{FormulaBox}

\[
\bar{x} = \frac{8+4+2+1+10}{5} = \frac{25}{5} = 5,
\qquad
\bar{y} = \frac{8+6+2+1+10}{5} = \frac{27}{5} = 5{,}4
\]

\textbf{Passo 3 — Medianas (dados ordenados)}\\
Horas ordenadas: $\{1,2,4,8,10\} \Rightarrow \tilde{x}=4$. \quad
Notas ordenadas: $\{1,2,6,8,10\} \Rightarrow \tilde{y}=6$.

\textbf{Passo 4 — Tabela para variância e desvio-padrão (Horas)}

\begin{center}
\begin{tabular}{@{}c|c|c@{}}\toprule
$\,x_i\,$ & $x_i - \bar{x}$ & $(x_i - \bar{x})^2$ \\\midrule
8  & 3   & 9 \\
4  & -1  & 1 \\
2  & -3  & 9 \\
1  & -4  & 16 \\
10 & 5   & 25 \\\midrule
\textbf{Soma} & --- & \textbf{60} \\\bottomrule
\end{tabular}
\end{center}

\begin{FormulaBox}
Variância amostral: $\quad s_x^2 = \dfrac{\sum (x_i - \bar{x})^2}{n-1}$ \qquad
Desvio-padrão: $\quad s_x = \sqrt{s_x^2}$
\end{FormulaBox}

\[
s_x^2 = \frac{60}{5-1} = \frac{60}{4} = 15,
\qquad
s_x = \sqrt{15} \approx 3{,}873
\]
\end{SolvedBox}

\begin{SolvedBox}
\textbf{Passo 5 — Tabela para variância e desvio-padrão (Notas)}



\begin{center}
\begin{tabular}{@{}c|c|c@{}}\toprule
$\,y_i\,$ & $y_i - \bar{y}$ & $(y_i - \bar{y})^2$ \\\midrule
8  & 2{,}6  & 6{,}76 \\
6  & 0{,}6  & 0{,}36 \\
2  & -3{,}4 & 11{,}56 \\
1  & -4{,}4 & 19{,}36 \\
10 & 4{,}6  & 21{,}16 \\\midrule
\textbf{Soma} & --- & \textbf{59{,}2} \\\bottomrule
\end{tabular}
\end{center}



\begin{FormulaBox}
Variância amostral: $\quad s_y^2 = \dfrac{\sum (y_i - \bar{y})^2}{n-1}$ \qquad
Desvio-padrão: $\quad s_y = \sqrt{s_y^2}$
\end{FormulaBox}

\[
s_y^2 = \frac{59{,}2}{4} = 14{,}8,
\qquad
s_y = \sqrt{14{,}8} \approx 3{,}85
\]

\textbf{Resumo (como eu apresentaria na resposta):}
\begin{itemize}[leftmargin=*]
  \item $\bar{x}=5$, $\tilde{x}=4$, $s_x^2=15$, $s_x\approx 3{,}873$;
  \item $\bar{y}=5{,}4$, $\tilde{y}=6$, $s_y^2=14{,}8$, $s_y\approx 3{,}85$.
\end{itemize}
\end{SolvedBox}

% ------------------------------------------------------------

\subsection*{Questão 3 — Construção do boxplot das notas}

\begin{SolvedBox}
\textbf{Contexto.} Usando as mesmas \emph{Notas} do conjunto acima, deseja-se construir o \textbf{boxplot}. Os valores de $Q_1$, $Q_2$ e $Q_3$ são informados pelo professor.

\textbf{Passo 1 — Dados ordenados e quartis dados}\\
Notas ordenadas: $\{1,2,6,8,10\}$ \quad (fornecido) $Q_1=2,\; Q_2=6,\; Q_3=8$.

\textbf{Passo 2 — Intervalo interquartil (IQR)}

\begin{FormulaBox}
$IQR = Q_3 - Q_1$
\end{FormulaBox}

\[
IQR = 8 - 2 = 6
\]

\textbf{Passo 3 — Limites por IQR (para outliers)}

\begin{FormulaBox}
\text{Mín\_IQR} = Q_1 - 1{,}5 \times IQR \qquad
\text{Máx\_IQR} = Q_3 + 1{,}5 \times IQR
\end{FormulaBox}

\[
\text{Mín\_IQR} = 2 - 1{,}5\times 6 = -7
\qquad
\text{Máx\_IQR} = 8 + 1{,}5\times 6 = 17
\]

\textbf{Passo 4 — Whiskers (mín e máx do boxplot)}\\
Mínimo geral = $1$, Máximo geral = $10$.\\
\emph{Regra:} 
\begin{itemize}[leftmargin=*]
  \item \textbf{Mínimo do boxplot} = $\max(\text{mínimo geral}, \text{Mín\_IQR}) = \max(1,-7)=1$;
  \item \textbf{Máximo do boxplot} = $\min(\text{máximo geral}, \text{Máx\_IQR}) = \min(10,17)=10$.
\end{itemize}
Como $1$ e $10$ estão \emph{dentro} de $[-7,17]$, \textbf{não há outliers}.

\textbf{Passo 5 — Desenho esquemático (valores indicados)}\\[-4pt]
\begin{figure}[H]
\centering
\begin{tikzpicture}[x=0.6cm,y=0.6cm]
  % Eixo base (0 a 11 para visual)
  \draw[gray!60] (-1,0) -- (11,0);
  \foreach \x in {0,...,10} \draw[gray!60] (\x,0.1) -- (\x,-0.1) node[below] {\scriptsize \x};

  % Caixa Q1 a Q3
  \filldraw[fill=pastelblue, draw=black] (2,1) rectangle (8,3); % Q1=2, Q3=8
  % Mediana
  \draw[very thick] (6,1) -- (6,3); % Q2=6
  % Whiskers
  \draw[thick] (1,2) -- (2,2); % min whisker
  \draw[thick] (8,2) -- (10,2); % max whisker
  % Pontos rotulados
  \node[below] at (2,1) {\scriptsize $Q_1=2$};
  \node[below] at (6,1) {\scriptsize $Q_2=6$};
  \node[below] at (8,1) {\scriptsize $Q_3=8$};
  \node[below] at (1,0) {\scriptsize min=1};
  \node[below] at (10,0) {\scriptsize max=10};
\end{tikzpicture}
\caption{Boxplot das notas com $Q_1=2$, $Q_2=6$, $Q_3=8$, whiskers em 1 e 10.}
\end{figure}

\textbf{Como eu escreveria a interpretação:} a metade central dos dados está entre $2$ e $8$ (IQR=6), a mediana é $6$, e não há outliers.
\end{SolvedBox}

\clearpage
\section*{Problemas Sugeridos — Praticando Análise Univariada}

\begin{enumerate}[leftmargin=*]

  \item Considere as idades $\{18, 20, 21, 23, 30\}$.  
  Calcule média, mediana e moda. Interprete se a média ou a mediana representa melhor o grupo.

  \item Para os salários mensais $\{2000, 2200, 2300, 2400, 15000\}$, calcule média, mediana, variância e desvio-padrão.  
  Verifique se há outlier com base no boxplot.

  \item Dada a série de notas $\{5, 7, 8, 9, 10\}$:  
  Calcule o intervalo interquartil (IQR) e os limites inferior e superior (Mín\_IQR e Máx\_IQR).

  \item Considere os tempos de atendimento em minutos $\{3, 4, 5, 6, 25\}$.  
  Construa o boxplot (informe $Q_1, Q_2, Q_3$) e interprete o impacto do valor 25.

  \item Uma pesquisa registrou a quantidade de livros lidos em um mês por 6 pessoas: $\{0, 1, 2, 2, 3, 10\}$.  
  Calcule média, mediana, variância e desvio-padrão. Interprete.

  \item Para os dados de temperatura $\{21, 22, 22, 23, 24, 25, 40\}$, calcule a média, o desvio-padrão e verifique se há outlier pelo método do IQR.

  \item Uma base contém os valores de vendas (em milhares de reais): $\{15, 16, 16, 17, 18, 100\}$.  
  Construa o boxplot e indique se 100 é ou não um outlier.

  \item Dada a sequência de alturas (cm) $\{150, 155, 160, 165, 170, 175, 180\}$, calcule média, mediana, variância e desvio-padrão.  
  Interprete a simetria dos dados.

  \item Um conjunto de dados registra a quantidade de horas de sono por noite: $\{5, 6, 7, 7, 8, 9, 10\}$.  
  Calcule a média e o desvio-padrão. Qual a interpretação prática de $s$ neste contexto?

  \item Considerando os valores de consumo de energia (kWh): $\{120, 125, 130, 135, 200\}$, construa o boxplot.  
  Explique como a presença do valor 200 altera a interpretação dos dados.

\end{enumerate}







\section*{Apêndice — Cálculo de Percentis e Quartis}
\begin{ProofBox}
\textbf{O que são percentis?}  
Percentis são medidas de posição que dividem um conjunto de dados ordenados em 100 partes iguais.  
- O percentil $p$ indica o valor abaixo do qual está $p\%$ das observações.  
Exemplo: o percentil 90 ($P_{90}$) é o valor abaixo do qual estão 90\% dos dados.  

\medskip
\textbf{Como calcular percentis (método clássico):}  
\[
\text{Posição}(p) = \frac{p}{100} \times (n+1)
\]
onde $n$ é o número de observações e $p$ é o percentil desejado.  

- Se a posição for um número inteiro, o percentil corresponde exatamente ao valor nessa posição dos dados ordenados.  
- Se a posição não for inteira, o valor é obtido por \textbf{interpolação linear} entre os dois valores mais próximos.  

\medskip
\textbf{Exemplo:}  
Para $n=10$ observações e $p=25$:  
\[
\text{Posição}(25) = \frac{25}{100}(10+1) = 2,75
\]  
Isso significa que $Q_1$ (percentil 25) está entre o 2º e o 3º valor da lista ordenada, a 75\% da distância entre eles.

\medskip
\textbf{Quartis como casos especiais de percentis:}  
- $Q_1 = P_{25}$: 25\% dos dados estão abaixo.  
- $Q_2 = P_{50}$: mediana (50\% dos dados abaixo e 50\% acima).  
- $Q_3 = P_{75}$: 75\% dos dados estão abaixo.  

\medskip
\textbf{Observação prática:}  
Na estatística aplicada e em softwares (R, Python, Excel, SQL), existem diferentes métodos para cálculo de percentis, que podem variar no uso de $n$ ou $n+1$ na fórmula.  
Na AED, o importante é manter a consistência no método escolhido.
\end{ProofBox}


\chapter{Análise Bivariada}

\section{Introdução}
A \textbf{análise bivariada} é o estudo de duas variáveis ao mesmo tempo, com o objetivo de verificar a existência de associação, tendência ou dependência entre elas.  

Perguntas típicas que a análise bivariada ajuda a responder:
\begin{itemize}[leftmargin=*]
  \item Existe relação entre as horas de estudo e o desempenho em provas?
  \item A renda de um consumidor influencia seu gasto mensal?
  \item A temperatura está relacionada às vendas de sorvete?
\end{itemize}

\section{Tabelas de Contingência e Gráficos de Dispersão}
Antes das medidas numéricas, a exploração inicial pode ser feita por:
\begin{itemize}[leftmargin=*]
  \item \textbf{Tabelas de contingência}: aplicáveis quando as variáveis são qualitativas (ex.: gênero $\times$ preferência de produto).
  \item \textbf{Gráficos de dispersão}: aplicáveis quando as variáveis são quantitativas, permitindo observar padrões de associação (linearidade, tendência positiva/negativa, presença de outliers).
\end{itemize}

\section{Covariância}
\subsection*{Definição}
A \textbf{covariância} mede o grau em que duas variáveis variam conjuntamente.  
- Se valores altos de $X$ estão associados a valores altos de $Y$, a covariância tende a ser positiva.  
- Se valores altos de $X$ estão associados a valores baixos de $Y$, a covariância tende a ser negativa.  
- Se não há relação linear clara, a covariância tende a valores próximos de zero.

\begin{FormulaBox}
\[
\mathrm{Cov}(X,Y)=\dfrac{1}{n-1}\sum_{i=1}^n (x_i-\bar{x})(y_i-\bar{y})
\]
\end{FormulaBox}

\subsection*{Interpretação}
- $\mathrm{Cov}(X,Y)>0$: associação linear positiva.  
- $\mathrm{Cov}(X,Y)<0$: associação linear negativa.  
- $\mathrm{Cov}(X,Y)\approx 0$: ausência de relação linear clara.

\subsection*{Exemplo}
Considere as variáveis \emph{Horas de estudo} ($X$) e \emph{Nota final} ($Y$):

\begin{center}
\begin{tabular}{@{}c|c|c|c|c@{}}\toprule
$x_i$ & $y_i$ & $(x_i-\bar{x})$ & $(y_i-\bar{y})$ & Produto $(x_i-\bar{x})(y_i-\bar{y})$ \\\midrule
8 & 8  & 3   & 2,6  & 7,8 \\
4 & 6  & -1  & 0,6  & -0,6 \\
2 & 2  & -3  & -3,4 & 10,2 \\
1 & 1  & -4  & -4,4 & 17,6 \\
10& 10 & 5   & 4,6  & 23,0 \\\midrule
\multicolumn{4}{r|}{Soma} & \textbf{58,0} \\\bottomrule
\end{tabular}
\end{center}

\[
\mathrm{Cov}(X,Y)=\frac{58}{5-1}=14,5
\]

A covariância positiva indica que quanto mais horas de estudo, maior tende a ser a nota.

\subsection*{Limitações}
A covariância não é normalizada: o valor depende da unidade de medida das variáveis.  
Por exemplo, medir horas em minutos aumentaria numericamente a covariância sem mudar a relação.  
Por isso, usamos a \textbf{correlação} como medida padronizada.

\section{Correlação de Pearson}
\subsection*{Definição}
A \textbf{correlação de Pearson} é a medida mais usada para quantificar a força e a direção da relação linear entre duas variáveis quantitativas.  
Ela é obtida pela normalização da covariância.

\begin{FormulaBox}
\[
r = \dfrac{\mathrm{Cov}(X,Y)}{s_X s_Y}, \quad r\in[-1,1]
\]
\end{FormulaBox}

\subsection*{Interpretação dos valores de $r$}
\begin{itemize}[leftmargin=*]
  \item $r \approx 1$: correlação linear positiva muito forte.
  \item $r \approx -1$: correlação linear negativa muito forte.
  \item $r \approx 0$: ausência de relação linear clara.
\end{itemize}

\subsection*{Exemplo (continuação do anterior)}
Sabendo que $s_X=\sqrt{15}\approx 3,873$ e $s_Y=\sqrt{14,8}\approx 3,847$:

\[
r = \frac{14,5}{(3,873)(3,847)} \approx 0,973
\]

Isso indica uma \textbf{forte correlação positiva}: mais horas de estudo estão fortemente associadas a notas mais altas.

\subsection*{Observações importantes}
- Correlação mede \emph{associação}, não \emph{causalidade}.  
- Relações não lineares podem ter correlação próxima de zero, mesmo havendo forte associação (ex.: curva em forma de U).  
- A presença de outliers pode distorcer o valor de $r$.

\section{Resumo Comparativo}
\begin{itemize}[leftmargin=*]
  \item \textbf{Covariância}: indica direção da relação, mas depende da escala.  
  \item \textbf{Correlação}: padroniza a covariância, facilitando a interpretação ($-1 \leq r \leq 1$).  
\end{itemize}


\section{Problema Resolvido — Relação entre horas de estudo e notas}

\begin{SolvedBox}
\textbf{Contexto.} Foram registradas \emph{Horas de estudo} ($x$) e \emph{Notas finais} ($y$) de 5 estudantes. Queremos calcular a \textbf{covariância} e a \textbf{correlação de Pearson} entre as variáveis.

\textbf{Passo 1 — Tabela de dados}\\[-6pt]
\begin{center}
\begin{tabular}{@{}lccccc@{}}\toprule
Registro & R1 & R2 & R3 & R4 & R5 \\\midrule
Horas ($x$) & 8 & 4 & 2 & 1 & 10 \\
Notas ($y$) & 8 & 6 & 2 & 1 & 10 \\\bottomrule
\end{tabular}
\end{center}

\textbf{Passo 2 — Médias}\\
\[
\bar{x} = \frac{25}{5} = 5, 
\qquad 
\bar{y} = \frac{27}{5} = 5{,}4
\]

\textbf{Passo 3 — Tabela para produtos centrados}

\begin{center}
\begin{tabular}{@{}c|c|c|c|c@{}}\toprule
$x_i$ & $y_i$ & $(x_i-\bar{x})$ & $(y_i-\bar{y})$ & Produto $(x_i-\bar{x})(y_i-\bar{y})$ \\\midrule
8  & 8  & 3   & 2,6  & 7,8  \\
4  & 6  & -1  & 0,6  & -0,6 \\
2  & 2  & -3  & -3,4 & 10,2 \\
1  & 1  & -4  & -4,4 & 17,6 \\
10 & 10 & 5   & 4,6  & 23,0 \\\midrule
\textbf{Totais} & --- & --- & --- & \textbf{58,0} \\\bottomrule
\end{tabular}
\end{center}

\textbf{Passo 4 — Covariância}

\begin{FormulaBox}
$\mathrm{Cov}(X,Y) = \dfrac{\sum (x_i-\bar{x})(y_i-\bar{y})}{n-1}$
\end{FormulaBox}

\[
\mathrm{Cov}(X,Y) = \frac{58}{4} = 14,5
\]

\textbf{Passo 5 — Correlação de Pearson}

\begin{FormulaBox}
$r = \dfrac{\mathrm{Cov}(X,Y)}{s_X s_Y}$
\end{FormulaBox}

Sabemos: $s_X=\sqrt{15}\approx 3,873$, \quad $s_Y=\sqrt{14,8}\approx 3,847$.

\[
r = \frac{14,5}{(3,873)(3,847)} \approx \frac{14,5}{14,92} \approx 0,973
\]

\textbf{Conclusão:} $r \approx 0,97$, indicando \textbf{forte correlação linear positiva} entre horas de estudo e nota.
\end{SolvedBox}

\section*{Problemas Sugeridos — Análise Bivariada}

\begin{enumerate}[leftmargin=*]
  \item Calcule a covariância e a correlação entre os pares $(x,y)=\{(1,2),(2,3),(3,5),(4,4)\}$.  
  \item Um conjunto de dados registra altura (cm) e peso (kg) de 5 pessoas. Estime a correlação.  
  \item Construa um gráfico de dispersão para as horas de sono $\{5,6,7,8,9\}$ e desempenho $\{60,65,70,80,85\}$. Descreva a tendência.  
  \item Em uma empresa, vendas (mil R\$) e gastos em propaganda (mil R\$) foram:
  $$\{(10,3),(15,5),(20,7),(30,9)\}$$. Calcule $r$.  
  \item Mostre que se todas as observações de $y$ forem iguais, a correlação é indefinida.  
  \item Para as notas de matemática e português de 6 alunos, construa a tabela de produtos centrados e estime $r$.  
  \item Explique com exemplo quando $r$ próximo de zero não significa ausência de relação.  
  \item Compare os gráficos de dispersão de correlações $r=0,9$, $r=0$, e $r=-0,9$. Explique a diferença.  
  \item Uma base tem pares $(x,y)=\{(2,4),(4,8),(6,12),(8,16)\}$. Calcule $r$ e interprete.  
  \item Explique, com um exemplo de mercado, por que “correlação não implica causalidade”.
\end{enumerate}

\section*{Apêndice do Capítulo 3 — Demonstrações}

\begin{ProofBox}
\textbf{Por que a correlação é adimensional?}\\[4pt]
A covariância depende das unidades de medida de $X$ e $Y$.  
Ao dividir por $s_X s_Y$, normalizamos essa medida, garantindo $r \in [-1,1]$, sem unidade.  

\textbf{Valores de referência:}\\
$r \approx 1$: relação linear positiva forte.\\
$r \approx -1$: relação linear negativa forte.\\
$r \approx 0$: ausência de relação linear (mas podem existir relações não lineares).
\end{ProofBox}

\chapter{Análise Multivariada}

\section{Visão Geral}
A análise multivariada considera três ou mais variáveis simultaneamente, permitindo estudar relações complexas entre fatores.  
Exemplos práticos incluem: \textbf{matriz de correlação}, \textbf{regressão múltipla}, \textbf{análise de componentes principais (PCA)}, \textbf{clusterização} e \textbf{árvores de decisão}.  

Neste capítulo, vamos focar nos conceitos fundamentais de \textbf{probabilidade, probabilidade condicional, odds e odds ratio}, essenciais para responder perguntas do tipo:  
- Quais fatores mais influenciam a chance de um aluno alcançar nota alta?  
- Qual grupo apresenta maior risco ou chance relativa?  

---

\section{Probabilidade}
\subsection*{Definição}
A probabilidade mede a chance de ocorrência de um evento em um espaço de possibilidades.

\begin{FormulaBox}
Probabilidade: 
\[
P(A) = \frac{\text{número de casos favoráveis}}{\text{número total de casos}}
\]
\end{FormulaBox}

\subsection*{Exemplo}
Em uma turma de 20 alunos, 8 obtiveram nota $\ge 8$.  
\[
P(\text{Alta}) = \frac{8}{20} = 0,4 = 40\%
\]

---

\section{Probabilidade Condicional}
\subsection*{Definição}
A probabilidade condicional mede a chance de um evento ocorrer dado que outro já ocorreu.

\begin{FormulaBox}
Probabilidade condicional:
\[
P(A \mid B) = \frac{P(A \cap B)}{P(B)}
\]
\end{FormulaBox}

\subsection*{Exemplo}
Se entre 10 alunos que estudaram mais de 5 horas, 7 tiveram nota $\ge 8$:  
\[
P(\text{Alta} \mid \text{Horas}>5) = \frac{7}{10} = 70\%
\]

---

\section{Odds}
\subsection*{Definição}
Enquanto a probabilidade compara sucessos com o total de casos, as \textbf{odds} comparam sucessos com fracassos.

\begin{FormulaBox}
\[
\mathrm{Odds}(A) = \frac{P(A)}{1-P(A)}
\]
\end{FormulaBox}

\subsection*{Exemplo}
Se $P(\text{Alta})=0,4$:  
\[
\mathrm{Odds}(\text{Alta}) = \frac{0,4}{1-0,4} = \frac{0,4}{0,6} \approx 0,67
\]  
Ou seja, para cada 10 alunos, a razão esperada é $\approx 4$ com Alta contra $\approx 6$ sem Alta.

---

\section{Odds Ratio (OR)}
\subsection*{Definição}
O Odds Ratio compara duas razões de chances, mostrando quanto mais provável é um evento em um grupo em relação a outro.

\begin{FormulaBox}
\[
\mathrm{OR}_{A:B} = \frac{\mathrm{Odds}(A)}{\mathrm{Odds}(B)}
\]
\end{FormulaBox}

\subsection*{Exemplo}
- Grupo A: $P(\text{Alta})=0,7 \Rightarrow \mathrm{Odds}_A=\tfrac{0,7}{0,3}=2,33$  
- Grupo B: $P(\text{Alta})=0,4 \Rightarrow \mathrm{Odds}_B=\tfrac{0,4}{0,6}=0,67$  

\[
\mathrm{OR}_{A:B} = \frac{2,33}{0,67} \approx 3,5
\]

Interpretação: alunos do grupo A têm cerca de 3,5 vezes mais chance de Alta em comparação ao grupo B.

---

\section{Como identificar características que mais influenciam uma variável-alvo com Odds Ratio}
\begin{itemize}[leftmargin=*]
  \item Passo 1: Defina a variável-alvo binária (ex.: Nota Alta = sim/não).  
  \item Passo 2: Separe os dados em grupos (ex.: faixas de horas de estudo).  
  \item Passo 3: Calcule as probabilidades de sucesso em cada grupo.  
  \item Passo 4: Converta em Odds.  
  \item Passo 5: Compare com um grupo de referência via OR.  
\end{itemize}

\textbf{Exemplo prático:}  
Se para $\ge 8$ horas de estudo o OR = 4, e para $< 4$ horas o OR = 0,5, concluímos que estudar mais tempo aumenta significativamente a chance de nota Alta.

---
\clearpage
\section{Problemas Resolvidos — Odds e Odds Ratio}

\subsection*{Exemplo 1 — Horas de estudo e notas}


\begin{SolvedBox}
\textbf{Contexto:} A classificação \emph{Nota Alta/Baixa} segue as seguintes regras fornecidas: \\
\begin{tabular}{@{}c|c|c@{}}\toprule
Horas de estudo & Nota & Nota Alta/Baixa \\\midrule
$<4$   & $<6$   & Baixa \\
$<4$   & $\ge 6$& Baixa \\
$5$--$7$ & $<6$   & Alta \\
$5$--$7$ & $\ge 6$& Alta \\
$\ge 8$  & $<6$   & Alta \\
$\ge 8$  & $\ge 6$& Alta \\\bottomrule
\end{tabular}

\medskip
Para tornar o cálculo concreto, suponha \textbf{10 observações em cada linha} (total 60). Agregando por \emph{faixa de horas}:

\begin{center}
\begin{tabular}{@{}c|c|c|c@{}}\toprule
Grupo de Horas & Alta & Baixa & Total \\\midrule
$<4$     & 0  & 20 & 20 \\
$5$--$7$ & 20 & 0  & 20 \\
$\ge 8$  & 20 & 0  & 20 \\\bottomrule
\end{tabular}
\end{center}

\textbf{Passo 1 — Probabilidades de Alta por grupo}
\[
P(\text{Alta}\mid <4)=\tfrac{0}{20}=0,\quad
P(\text{Alta}\mid 5\text{--}7)=\tfrac{20}{20}=1,\quad
P(\text{Alta}\mid \ge 8)=\tfrac{20}{20}=1.
\]

\textbf{Passo 2 — \emph{Odds} de Alta por grupo (com e sem correção)}
\begin{FormulaBox}
\textbf{Odds (sem correção):}\;\; \mathrm{Odds}=\dfrac{\text{Alta}}{\text{Baixa}}
\end{FormulaBox}

\begin{FormulaBox}
\textbf{Odds (Haldane–Anscombe):}\;\; \mathrm{Odds}_{0{,}5}=\dfrac{\text{Alta}+0{,}5}{\text{Baixa}+0{,}5}
\end{FormulaBox}

\begin{itemize}[leftmargin=*]
\item $<4$: $\mathrm{Odds}=\tfrac{0}{20}=0$ (indefinida para OR); \quad
      $\mathrm{Odds}_{0{,}5}=\tfrac{0{,}5}{20{,}5}\approx 0{,}0244$.
\item $5$--$7$: $\mathrm{Odds}=\tfrac{20}{0}=\infty$; \quad
      $\mathrm{Odds}_{0{,}5}=\tfrac{20{,}5}{0{,}5}=41{,}0$.
\item $\ge 8$: $\mathrm{Odds}=\tfrac{20}{0}=\infty$; \quad
      $\mathrm{Odds}_{0{,}5}=41{,}0$.
\end{itemize}
\end{SolvedBox}



\begin{SolvedBox}
\textbf{Passo 3 — Odds Ratio (OR) usando o grupo $\ge 8$ como referência}

\begin{FormulaBox}
\textbf{Odds Ratio:}\;\; \mathrm{OR}_{A:\text{ref}}=\dfrac{\mathrm{Odds}(A)}{\mathrm{Odds}(\text{ref})}
\end{FormulaBox}

\[
\mathrm{OR}_{<4:\ge 8}=\frac{0{,}0244}{41{,}0}\approx 5{,}95\times 10^{-4},
\qquad
\mathrm{OR}_{5\text{--}7:\ge 8}=\frac{41{,}0}{41{,}0}=1{,}00.
\]

\textbf{Interpretação:}
\begin{itemize}[leftmargin=*]
\item Estudar \textbf{$<4$ horas} está associado a \emph{odds} de Alta \textbf{quase nulas} comparadas a $\ge 8$ horas.
\item Os grupos \textbf{$5$--$7$} e \textbf{$\ge 8$} têm \emph{odds} praticamente idênticas (OR $\approx 1$) neste cenário.
\item O uso da correção $+0{,}5$ é fundamental quando há \textbf{zeros} nas células, evitando divisões por zero e permitindo comparar grupos via OR.
\end{itemize}
\end{SolvedBox}



% -----------------------------------------------------

\subsection*{Exemplo 2 — Exercício físico e pressão arterial}

\begin{SolvedBox}
\textbf{Contexto:} Pesquisadores dividiram pessoas em dois grupos: praticam exercício físico regularmente (Sim) e não praticam (Não). O desfecho observado foi \emph{pressão arterial controlada (Sim ou Não)}.

\textbf{Tabela de frequências:}  

\begin{center}
\begin{tabular}{@{}c|c|c@{}}\toprule
Grupo & Pressão Controlada & Não Controlada \\\midrule
Exercício (Sim) & 18 & 12 \\
Exercício (Não) & 10 & 20 \\\bottomrule
\end{tabular}
\end{center}

\textbf{Passo 1 — Probabilidades}  
- Exercício (Sim): $P(\text{Controlada}) = 18/30 = 0,6$  
- Exercício (Não): $P(\text{Controlada}) = 10/30 \approx 0,33$

\textbf{Passo 2 — Odds}  
- Exercício (Sim): $\mathrm{Odds} = 18/12 = 1,5$  
- Exercício (Não): $\mathrm{Odds} = 10/20 = 0,5$

\textbf{Passo 3 — OR (referência = Não)}  
\[
\mathrm{OR} = \frac{1,5}{0,5} = 3
\]

\textbf{Conclusão:} Pessoas que praticam exercício têm 3 vezes mais chance de controlar a pressão arterial do que as sedentárias.
\end{SolvedBox}

% -----------------------------------------------------

\subsection*{Exemplo 3 — Uso de aplicativo de estudo e aprovação}

\begin{SolvedBox}
\textbf{Contexto:} Estudantes foram separados pelo uso de um aplicativo de estudo (Sim/Não). O desfecho foi aprovação na disciplina.

\textbf{Tabela de frequências:}  

\begin{center}
\begin{tabular}{@{}c|c|c@{}}\toprule
Grupo & Aprovado & Reprovado \\\midrule
App (Sim)  & 45 & 15 \\
App (Não)  & 20 & 25 \\\bottomrule
\end{tabular}
\end{center}

\textbf{Passo 1 — Probabilidades}  
- App (Sim): $P(\text{Aprovado}) = 45/60 = 0,75$  
- App (Não): $P(\text{Aprovado}) = 20/45 \approx 0,44$

\textbf{Passo 2 — Odds}  
- App (Sim): $\mathrm{Odds} = 45/15 = 3,0$  
- App (Não): $\mathrm{Odds} = 20/25 = 0,8$

\textbf{Passo 3 — OR (referência = Não)}  
\[
\mathrm{OR} = \frac{3,0}{0,8} = 3,75
\]

\textbf{Conclusão:} Estudantes que usam o aplicativo têm quase 4 vezes mais chance de aprovação do que os que não usam.
\end{SolvedBox}

% -----------------------------------------------------

\subsection*{Exemplo 4 — Consumo de frutas e obesidade}

\begin{SolvedBox}
\textbf{Contexto:} Em um estudo, indivíduos foram separados em dois grupos: consumo diário de frutas (Sim/Não). O desfecho observado foi obesidade (Sim/Não).

\textbf{Tabela de frequências:}  

\begin{center}
\begin{tabular}{@{}c|c|c@{}}\toprule
Grupo & Obeso & Não Obeso \\\midrule
Frutas (Sim)  & 12 & 48 \\
Frutas (Não)  & 30 & 30 \\\bottomrule
\end{tabular}
\end{center}

\textbf{Passo 1 — Probabilidades}  
- Frutas (Sim): $P(\text{Obeso}) = 12/60 = 0,2$  
- Frutas (Não): $P(\text{Obeso}) = 30/60 = 0,5$

\textbf{Passo 2 — Odds}  
- Frutas (Sim): $\mathrm{Odds} = 12/48 = 0,25$  
- Frutas (Não): $\mathrm{Odds} = 30/30 = 1,0$

\textbf{Passo 3 — OR (referência = Não)}  
\[
\mathrm{OR} = \frac{0,25}{1,0} = 0,25
\]

\textbf{Conclusão:} Consumir frutas diariamente reduz a chance de obesidade (odds 4 vezes menores).
\end{SolvedBox}

% -----------------------------------------------------

\subsection*{Exemplo 5 — Tabagismo e doença respiratória}

\begin{SolvedBox}
\textbf{Contexto:} Uma pesquisa analisou fumantes e não fumantes, verificando a ocorrência de doenças respiratórias.

\textbf{Tabela de frequências:}  

\begin{center}
\begin{tabular}{@{}c|c|c@{}}\toprule
Grupo & Doente & Saudável \\\midrule
Fumante     & 40 & 60 \\
Não Fumante & 20 & 80 \\\bottomrule
\end{tabular}
\end{center}

\textbf{Passo 1 — Probabilidades}  
- Fumante: $P(\text{Doente}) = 40/100 = 0,4$  
- Não Fumante: $P(\text{Doente}) = 20/100 = 0,2$

\textbf{Passo 2 — Odds}  
- Fumante: $\mathrm{Odds} = 40/60 = 0,67$  
- Não Fumante: $\mathrm{Odds} = 20/80 = 0,25$

\textbf{Passo 3 — OR (referência = Não)}  
\[
\mathrm{OR} = \frac{0,67}{0,25} \approx 2,68
\]

\textbf{Conclusão:} Fumantes têm quase 3 vezes mais chance de desenvolver doença respiratória em comparação com não fumantes.
\end{SolvedBox}

% -----------------------------------------------------

\subsection*{Exemplo 6 — Horas e notas}

\begin{SolvedBox}

\textbf{Contexto:} Uma turma foi dividida em 3 grupos segundo as horas de estudo: $<4$, $5$–$7$ e $\ge 8$. Foi registrada a nota final (Alta $\ge 8$ ou Baixa $<8$).  

\textbf{Tabela de frequências (dados simulados)}  

\begin{center}
\begin{tabular}{@{}c|c|c@{}}\toprule
Grupo de Horas & Nota Alta & Nota Baixa \\\midrule
$<4$       & 2 & 8 \\
$5$–$7$    & 5 & 5 \\
$\ge 8$    & 7 & 3 \\\bottomrule
\end{tabular}
\end{center}

\textbf{Passo 1 — Probabilidade condicional (ex.: Alta dado $5$–$7$)}  
\[
P(\text{Alta}\mid 5\text{–}7) = \frac{5}{10} = 0,5
\]

\textbf{Passo 2 — Cálculo das Odds}  
- Grupo $<4$: $\mathrm{Odds}=\frac{2}{8}=0,25$  
- Grupo $5$–$7$: $\mathrm{Odds}=\frac{5}{5}=1,0$  
- Grupo $\ge 8$: $\mathrm{Odds}=\frac{7}{3}\approx 2,33$

\textbf{Passo 3 — Cálculo dos OR (referência: $\ge 8$)}  
\[
\mathrm{OR}_{<4:\ge 8} = \frac{0,25}{2,33}\approx 0,11
\]
\[
\mathrm{OR}_{5\text{–}7:\ge 8} = \frac{1,0}{2,33}\approx 0,43
\]

\textbf{Conclusão:}  
- O grupo $\ge 8$ é o que mais favorece nota Alta.  
- Quanto menor o tempo de estudo, menor a chance de nota Alta em relação ao grupo de referência.
\end{SolvedBox}


---

\section*{Problemas Sugeridos — Análise Multivariada}
\begin{enumerate}[leftmargin=*]
  \item Em uma pesquisa com 100 pacientes, 40 melhoraram após tratamento (Alta) e 60 não. Calcule as probabilidades, odds e OR entre os grupos Tratamento e Placebo.  
  \item Considere 3 grupos de estudantes (pouco, médio, muito estudo). Monte tabelas $2\times 2$ e calcule os OR relativos ao grupo de muito estudo.  
  \item Em um jogo de basquete, 30 de 50 arremessos foram convertidos. Calcule $P$, Odds e interprete.  
  \item Explique a diferença entre $P(\text{Vitória}\mid \text{Casa})$ e $P(\text{Casa}\mid \text{Vitória})$ em uma temporada esportiva.  
  \item Se em uma pesquisa eleitoral 55\% votariam no candidato A, calcule a probabilidade, odds e interprete.  
  \item Monte um exemplo em que a probabilidade de sucesso é a mesma em dois grupos, mas o OR é diferente (discuta por quê).  
  \item Explique por que o OR é usado em regressão logística como medida de efeito.  
  \item Em um estudo de saúde, fumo $\times$ doença respiratória apresentaram OR=4. Interprete.  
  \item Calcule e compare Odds e OR em um campeonato de futebol entre times de alto e baixo investimento.  
  \item Discuta o impacto de outliers (jogadores extremos) nos cálculos de probabilidade e OR.
\end{enumerate}

---

\section*{Apêndice do Capítulo 4 — Odds e Apostas Online}

\begin{ProofBox}
\textbf{Odds nas apostas esportivas:}  
Nas casas de apostas, as \emph{odds} representam a chance atribuída a um evento.  

- Se $P(\text{Vitória})=0,5$, a odd justa seria $2,0$ (pagamento de 2 para cada 1 apostado).  
- Odds baixas (ex.: 1,2) indicam favorito; odds altas (ex.: 5,0) indicam azarão.

\textbf{Odds Ratio nas apostas:}  
Pode-se comparar o risco relativo entre dois resultados:  
- Ex.: OR entre “Time A vencer” e “Time B vencer” mostra qual lado tem mais chance relativa.  

\textbf{Prática:}  
Casas de apostas usam margens (ou “taxa da casa”) que reduzem a odd real para garantir lucro.  
Na AED e na estatística, o OR serve como medida análoga, mas aplicada em contextos de saúde, educação, marketing e ciências sociais.
\end{ProofBox}

\chapter{Fundamentos do Tableau e Conexão com Dados} % Capítulo 5

% =========================================================
% 5.0 — Contexto: AED, Visualização e a escolha de ferramentas
% =========================================================

\section{Onde estamos no processo de AED?}

A \textbf{Análise Exploratória de Dados (AED)} organiza-se em quatro macroetapas interdependentes, que formam o ciclo contínuo de exploração, entendimento e comunicação dos dados. 

\begin{enumerate}[leftmargin=*]
  \item \textbf{Ingestão e Leitura de Dados} — etapa inicial de coleta e importação das informações a partir de diferentes fontes (arquivos CSV, bancos de dados relacionais, planilhas, APIs, entre outros). O foco é garantir que os dados estejam acessíveis, íntegros e compreendidos em sua estrutura original (linhas, colunas, tipos, dicionário de variáveis).

  \item \textbf{Preparo e Transformação (\textit{Data Wrangling})} — fase em que os dados são tratados e modelados para análise: remoção de duplicatas, tratamento de nulos, padronização de formatos, criação de variáveis derivadas e normalização. É aqui que o analista transforma dados brutos em uma base coerente, limpa e pronta para exploração.

  \item \textbf{Análise Descritiva e Diagnóstica} — aplicação de métodos estatísticos e visuais para compreender padrões, tendências e relações entre variáveis. Envolve análises univariadas, bivariadas e multivariadas, apoiadas por gráficos e medidas resumo. Nesta fase, a visualização tem papel central na \textit{descoberta} de comportamentos ocultos, outliers e correlações.

  \item \textbf{Comunicação de Achados e Geração de Insights} — etapa final, responsável por traduzir os resultados analíticos em informações compreensíveis e acionáveis. Envolve a construção de \textit{dashboards}, \textit{stories} e relatórios interativos, nos quais a visualização serve para \textit{explicar} e \textit{convencer}, conectando análise e decisão.
\end{enumerate}

Este capítulo inaugura a \textbf{Parte II do livro}, onde a visualização passa de apoio pontual para \textbf{protagonista} do processo: aprenderemos a escolher gráficos adequados, construir painéis, elaborar narrativas e publicar resultados de forma profissional.




\begin{NoteBox}
\textbf{Crítica}: muitos projetos pulam da coleta para um \textit{dashboard} final. 
Sem exploração visual \textit{para descobrir} (prototipagem, testes de hipóteses), a narrativa fica frágil. 
A boa prática é iterar: explorar \(\rightarrow\) formular \(\rightarrow\) refinar \(\rightarrow\) comunicar.
\end{NoteBox}

\section{O que é uma ferramenta de visualização?}

Uma \textbf{ferramenta de visualização} é um software que transforma dados em representações gráficas, como gráficos de barras, linhas, mapas ou diagramas. O objetivo é permitir que o usuário \textbf{veja padrões, tendências e relações} que seriam difíceis de perceber apenas observando números em uma tabela. Essas ferramentas fazem a ponte entre a \textit{camada de dados} e a \textit{percepção humana}, traduzindo valores em formas visuais compreensíveis e interativas.

Em essência, ferramentas de visualização realizam um processo de \textbf{mapeamento visual}: elas pegam atributos dos dados (valores, categorias, tempo, localização) e os associam a propriedades gráficas (posição, cor, tamanho, forma, textura). Assim, o usuário pode explorar os dados visualmente e tomar decisões baseadas em evidências.

Uma ferramenta adequada para Análise Exploratória de Dados (AED) deve apresentar as seguintes capacidades:

\begin{itemize}[leftmargin=*]

  \item \textbf{Conectar múltiplas fontes de dados:}  
  As boas ferramentas conseguem se conectar a diferentes origens — arquivos locais (\texttt{.csv}, \texttt{.xlsx}), bancos de dados relacionais (MySQL, PostgreSQL, SQL Server), ou até APIs e serviços em nuvem. Essa capacidade de integração permite unir dados dispersos em um único ambiente analítico, sem necessidade de codificação complexa.

  \item \textbf{Modelar e agregar corretamente:}  
  Antes da visualização, os dados precisam ser modelados. Isso inclui:
  \begin{itemize}
    \item \textbf{Relacionamentos:} conectar tabelas por chaves comuns (ex.: \texttt{cliente\_id});
    \item \textbf{Joins:} combinar dados de diferentes fontes, como uma tabela de vendas com outra de produtos;
    \item \textbf{Camada semântica:} é a camada que organiza os dados de forma lógica e compreensível para o usuário final.  
    Em vez de enxergar apenas nomes de colunas técnicas, o analista visualiza conceitos do negócio, como “Faturamento”, “Quantidade Vendida” ou “Lucro”.  
    Essa camada também define medidas (métricas numéricas que podem ser somadas ou calculadas) e dimensões (categorias usadas para agrupar, como “Região” ou “Mês”).
  \end{itemize}
  O objetivo dessa etapa é garantir que o dado mostrado em um gráfico realmente represente uma agregação coerente (por exemplo, soma de vendas por mês ou média de nota por curso).

  \item \textbf{Expressar encodings visuais:}  
  O termo \textit{encoding} (codificação visual) se refere à forma como o dado é traduzido para a tela. Por exemplo:
  \begin{itemize}
    \item Posição no eixo X e Y para representar variáveis;
    \item Tamanho para indicar magnitude;
    \item Cor e forma para diferenciar categorias.
  \end{itemize}
  Aqui entram dois conceitos fundamentais:
  \begin{itemize}
    \item \textbf{Escala:} é a relação entre o valor do dado e sua representação visual.  
    Uma escala pode ser linear (0 a 100 em uma linha contínua), logarítmica (quando há grandes variações de magnitude) ou categórica (para valores discretos como “Masculino/Feminino”).  
    Escalas bem definidas garantem proporcionalidade e precisão na interpretação dos gráficos.
    \item \textbf{Paleta:} é o conjunto de cores usado para representar valores.  
    Uma boa paleta respeita princípios de contraste e acessibilidade, permitindo que todas as categorias sejam distinguíveis — inclusive por pessoas com daltonismo.  
    Ferramentas como Tableau e Power BI oferecem paletas contínuas (do claro ao escuro) e categóricas (cores distintas para grupos diferentes), que ajudam a reforçar a narrativa visual.
  \end{itemize}
  Além disso, os encodings visuais são complementados por \textit{tooltips} — pequenas janelas que aparecem ao passar o mouse sobre um ponto, exibindo detalhes numéricos ou contextuais do dado.

  \item \textbf{Interagir com o usuário:}  
  Uma visualização moderna não é estática: ela responde à interação.  
  Isso inclui:
  \begin{itemize}
    \item Filtros (selecionar apenas determinados períodos, regiões ou produtos);
    \item Destaques (realçar pontos específicos ao clicar ou passar o cursor);
    \item Parâmetros (valores dinâmicos que alteram cálculos ou visualizações);
    \item Ações entre painéis (clicar em um gráfico e ver outro se atualizar com base na seleção).  
  \end{itemize}
  Essa interatividade transforma dashboards em ferramentas exploratórias, onde o usuário conduz sua própria investigação.

  \item \textbf{Publicar e compartilhar com governança:}  
  Uma boa ferramenta de visualização permite publicar painéis e compartilhá-los de forma controlada.  
  Isso envolve versionamento (guardar histórico de alterações), permissões de acesso e integração com portais ou sistemas corporativos.  
  No contexto acadêmico, é o equivalente a permitir que outros pesquisadores repliquem e validem suas análises.

  \item \textbf{Escalar o desempenho:}  
  Visualizações sobre grandes volumes de dados exigem otimização.  
  Algumas ferramentas utilizam:
  \begin{itemize}
    \item \textit{Cache}: guarda resultados de consultas recentes para exibição mais rápida;
    \item \textit{Extracts} (como o formato \texttt{.hyper} do Tableau): criam cópias otimizadas dos dados para análise offline;
    \item Processamento \textit{in-memory}: carrega os dados na memória RAM, reduzindo o tempo de resposta de consultas complexas.
  \end{itemize}
  Essas estratégias são fundamentais para garantir que dashboards e análises funcionem bem mesmo com milhões de registros.
\end{itemize}

Em resumo, uma ferramenta de visualização ideal é aquela que \textbf{integra dados, modela significados, traduz números em formas visuais compreensíveis e possibilita a exploração interativa}.  
Ela atua como o elo entre o dado bruto e o insight — a descoberta significativa que guia decisões, pesquisas ou ações.


\section{Uma breve história crítica da visualização de dados}

A visualização de dados, como campo de estudo e prática, combina arte, ciência e tecnologia. Seu propósito central é transformar números e observações em formas visuais capazes de revelar padrões e significados. Embora hoje pareça natural construir gráficos interativos em poucos cliques, essa trajetória tem mais de dois séculos de evolução conceitual e técnica.

\begin{itemize}[leftmargin=*]

  \item \textbf{Séculos XVIII–XIX — As origens gráficas:}  
  A história da visualização de dados começa com o escocês \textbf{William Playfair} (1759–1823), considerado o “pai da estatística gráfica”. Ele introduziu os primeiros \textit{gráficos de barras}, \textit{gráficos de linhas} e \textit{gráficos de setores}, defendendo que a visão humana compreende relações numéricas de forma mais rápida quando representadas visualmente.  
  Poucas décadas depois, \textbf{Florence Nightingale} (1820–1910) utilizou diagramas em forma de roseta para demonstrar o impacto das más condições sanitárias na mortalidade hospitalar durante a Guerra da Crimeia. Seu trabalho foi pioneiro ao unir \textbf{visualização e argumentação científica}, utilizando gráficos como ferramenta de persuasão e política pública.  
  Nessa fase, o objetivo principal era \textbf{comunicar evidências} a públicos leigos e influenciar decisões — uma prática que antecipou o atual papel do \textit{data storytelling}.

  \item \textbf{Século XX — A formalização e o pensamento estatístico:}  
  No século XX, a visualização se torna um campo científico autônomo. O francês \textbf{Jacques Bertin} (1918–2010), em sua obra \textit{Semiologia Gráfica} (1967), definiu as \textbf{variáveis visuais fundamentais} — posição, tamanho, forma, cor, orientação e textura — que até hoje sustentam as bases do design de gráficos.  
  Paralelamente, o estatístico norte-americano \textbf{John Tukey} (1915–2000) propôs a \textbf{Análise Exploratória de Dados (AED)} em 1977, defendendo o uso intensivo de gráficos como ferramenta para compreender os dados antes de aplicar testes formais. Tukey introduziu o conceito de gráficos exploratórios, como boxplots e stem-and-leaf, focando na \textbf{descoberta de anomalias, padrões e hipóteses}.  
  Já \textbf{Edward Tufte}, com obras como \textit{The Visual Display of Quantitative Information} (1983), estabeleceu princípios de clareza, precisão e integridade visual, enfatizando que “\textit{gráficos ruins podem distorcer tanto quanto dados errados}”.  
  Essa fase marcou a transição da visualização como arte ilustrativa para uma \textbf{ciência da comunicação quantitativa}.

  \item \textbf{Final do século XX / início do XXI — O código e a gramática dos gráficos:}  
  Com o advento dos computadores, a visualização entrou na era digital. Linguagens como \textbf{S}, \textbf{R} e \textbf{Python} introduziram bibliotecas de geração de gráficos programáveis (\texttt{matplotlib}, \texttt{ggplot2}, \texttt{seaborn}), tornando possível criar visualizações complexas de forma reprodutível e automatizada.  
  Nesse contexto, o estatístico \textbf{Leland Wilkinson} publicou em 1999 o livro \textit{The Grammar of Graphics}, que propôs uma estrutura formal para construir gráficos a partir de uma \textbf{gramática visual}. Essa ideia inspirou diretamente o pacote \texttt{ggplot2} (de Hadley Wickham, 2005) e influenciou profundamente as ferramentas modernas de visualização, como Tableau e Power BI.  
  Ao mesmo tempo, a web trouxe a visualização interativa, com bibliotecas como \texttt{D3.js} permitindo manipular elementos gráficos dinamicamente.  
  O foco passou da representação estática para a \textbf{exploração dinâmica e interativa}, permitindo ao usuário manipular dados, aplicar filtros e observar padrões em tempo real.

  \item \textbf{Anos 2010+ — A era do \textit{Business Intelligence (BI)} e do \textit{self-service analytics}:}  
  Na década de 2010, a visualização de dados se consolidou como pilar do ecossistema de BI (Business Intelligence). Ferramentas como \textbf{Tableau}, \textbf{Power BI}, \textbf{Qlik Sense}, \textbf{Looker} e \textbf{Amazon QuickSight} democratizaram o acesso à análise visual, permitindo que usuários de negócio explorassem dados sem precisar programar.  
  Cada ferramenta trouxe uma inovação técnica:
  \begin{itemize}
    \item \textbf{Tableau:} introduziu o motor \textit{VizQL} (Visual Query Language), que traduz automaticamente interações visuais em consultas SQL, tornando o processo de análise intuitivo e rápido.
    \item \textbf{Power BI:} incorporou o modelo tabular do Excel com a linguagem DAX e o Power Query, criando uma ponte entre planilhas e modelagem analítica.
    \item \textbf{Qlik:} inovou com seu \textbf{motor associativo}, que permite descobrir relações entre dados de forma não linear, oferecendo liberdade exploratória.
    \item \textbf{Looker:} estruturou a camada semântica de dados via \textit{LookML}, aproximando desenvolvedores e analistas.
    \item \textbf{Amazon QuickSight:} trouxe o conceito de \textit{cloud-native analytics}, integrando visualizações diretamente com fontes em nuvem (S3, Athena, Redshift) e reduzindo custos de infraestrutura.
  \end{itemize}
  Essa geração marcou a passagem do gráfico isolado para o \textbf{produto de dados completo}: painéis interativos, narrativas visuais e publicação colaborativa com governança.  
  O analista moderno deixa de ser apenas um “criador de gráficos” e passa a ser um \textbf{curador de percepções}, utilizando ferramentas visuais como meio de comunicação estratégica.

  \item \textbf{A era atual — Inteligência aumentada e automação visual:}  
  Hoje, a visualização se integra a algoritmos de \textit{machine learning}, sistemas de recomendação e assistentes inteligentes. Ferramentas modernas sugerem automaticamente tipos de gráficos adequados, detectam outliers e geram narrativas visuais.  
  Plataformas como Power BI Copilot, Tableau Pulse e Looker AI incorporam \textbf{IA generativa} para traduzir perguntas em linguagem natural (“quais produtos venderam mais em 2024?”) em dashboards prontos.  
  Assim, a fronteira entre análise exploratória e automação analítica se torna cada vez mais tênue — e o papel humano passa a ser interpretar e contextualizar o insight.
\end{itemize}

\bigskip

Em resumo, a visualização de dados evoluiu da arte manual de Playfair e Nightingale para a ciência computacional e interativa dos dias atuais.  
O que antes era feito com régua e tinta, hoje é construído sobre bases estatísticas, linguagens formais e sistemas de BI em nuvem.  
Essa trajetória reflete uma constante: a busca por \textbf{ver para compreender}, tornando os dados acessíveis, confiáveis e visualmente significativos.


% =========================================================
% 5.1 — Panorama de ferramentas (comparativo legível)
% =========================================================

\section{Ferramentas principais antes do Tableau}

Antes do surgimento de plataformas modernas de visualização como o Tableau, Power BI e Looker, a análise de dados era conduzida principalmente por meio de \textbf{planilhas eletrônicas} e \textbf{bibliotecas estatísticas programáveis}. Essas ferramentas foram — e ainda são — a base da alfabetização analítica de milhões de profissionais no mundo.

\subsection*{Planilhas e bibliotecas}

\begin{itemize}[leftmargin=*]

  \item \textbf{Excel: o clássico que se recusa a morrer}  
  Lançado pela Microsoft em 1985, o Excel se tornou a ferramenta mais difundida da história da computação empresarial. Ele é onipresente porque une simplicidade e poder: com poucas fórmulas, o usuário pode organizar dados, calcular métricas e gerar gráficos em minutos.  

  O Excel evoluiu muito além de uma simples planilha. Suas \textbf{tabelas dinâmicas} (pivot tables) permitem resumir grandes volumes de dados de forma interativa, enquanto os gráficos incorporados possibilitam análises visuais imediatas. Além disso, o recurso de \textit{Power Query} trouxe capacidade de integração com diversas fontes externas (CSV, SQL, APIs, etc.), aproximando o Excel das ferramentas de BI.  

  Embora tenha limitações — como a ausência de controle de versão robusto, problemas de desempenho com grandes bases e dificuldade de governança —, o Excel se mantém relevante por três motivos principais:
  \begin{enumerate}
    \item \textbf{Baixa barreira de entrada:} qualquer pessoa com noções básicas consegue aprender rapidamente.
    \item \textbf{Flexibilidade:} pode ser usado tanto para cálculos simples quanto para modelos financeiros complexos.
    \item \textbf{Evolução inteligente:} a chegada do \textbf{Copilot} (IA da Microsoft) e da integração com o \textbf{ChatGPT} trouxeram uma nova camada de automação. Agora é possível criar gráficos, fórmulas e análises descritivas em linguagem natural — por exemplo: “explique a variação das vendas por mês” ou “crie um gráfico com os produtos mais vendidos”.  
  \end{enumerate}

  Essa fusão entre IA generativa e planilhas tradicionais revitalizou o Excel, tornando-o um \textbf{ambiente híbrido de análise e explicação}. Ele continua sendo o ponto de partida de inúmeros profissionais antes de migrarem para plataformas mais robustas de visualização.

  \item \textbf{Google Sheets + Looker Studio: colaboração e nuvem}  
  O Google Sheets popularizou a ideia de planilhas colaborativas, permitindo que várias pessoas trabalhassem no mesmo arquivo simultaneamente. Embora possua recursos mais limitados de modelagem e visualização do que o Excel, ele se destaca pela \textbf{integração direta com o ecossistema Google} (Forms, Drive, BigQuery).  
  Em conjunto com o Looker Studio (antigo Data Studio), o Sheets permite criar relatórios e dashboards simples para publicação web. É uma solução leve e acessível, ideal para contextos educacionais e análises rápidas em grupo.

  \item \textbf{Python e R (Plotly, Matplotlib, ggplot2): o poder do código}  
  Para usuários mais técnicos, as linguagens Python e R representaram um avanço enorme na visualização científica.  
  Com bibliotecas como \texttt{Matplotlib}, \texttt{Plotly} e \texttt{ggplot2}, tornou-se possível criar gráficos personalizados, interativos e reprodutíveis — algo essencial na pesquisa acadêmica e no desenvolvimento de produtos de dados.  
  O ponto forte é o \textbf{controle total sobre o visual e a lógica dos gráficos}. O ponto fraco é que o compartilhamento dessas visualizações exige alguma infraestrutura (por exemplo, o uso de Streamlit, Shiny ou Jupyter Notebook), o que ainda restringe o acesso de usuários não técnicos.
\end{itemize}

\subsection*{Plataformas de BI}

A evolução das planilhas e linguagens analíticas levou ao surgimento das ferramentas de \textbf{Business Intelligence (BI)} — plataformas que unem modelagem de dados, visualização interativa e governança em um único ambiente. Essas soluções começaram a se popularizar entre 2010 e 2015, acompanhando o crescimento do \textit{Big Data} e da computação em nuvem.

\begin{itemize}[leftmargin=*]

  \item \textbf{Power BI: democratização da análise corporativa}  
  Desenvolvido pela Microsoft, o Power BI trouxe o modelo tabular e a linguagem \textbf{DAX (Data Analysis Expressions)}, permitindo a criação de cálculos complexos e relacionamentos entre tabelas.  
  Ele é amplamente utilizado por empresas de todos os portes graças à sua integração com o Excel, Azure e SharePoint, além do excelente custo-benefício.  
  Seu desafio principal é a curva de aprendizado do DAX, que exige certa familiaridade com modelagem analítica.

  \item \textbf{Amazon QuickSight: BI nativo em nuvem}  
  Criado pela Amazon Web Services, o QuickSight é uma ferramenta \textit{cloud-native} que processa dados usando o mecanismo \textbf{SPICE} (\textit{Super-fast, Parallel, In-memory Calculation Engine}).  
  Ele é ideal para quem já utiliza a infraestrutura AWS (como S3, Athena, Redshift), mas oferece menos opções de customização visual em comparação a Tableau ou Power BI.  
  Seu foco está em escalabilidade, custo reduzido e integração com pipelines automatizados.

  \item \textbf{Qlik Sense: o motor associativo}  
  O Qlik introduziu um conceito inovador: o \textbf{motor associativo in-memory}, que permite explorar relações entre dados de maneira livre, sem depender de hierarquias pré-definidas.  
  Isso o torna uma ferramenta poderosa para descobertas exploratórias. Entretanto, seu licenciamento e curva de aprendizado são mais complexos, o que limita a adoção em ambientes educacionais.

  \item \textbf{Looker (camada semântica centralizada)}  
  O Looker, adquirido pelo Google, trabalha sobre o conceito de \textbf{LookML} — uma linguagem para descrever modelos de dados e métricas centralizadas.  
  Essa abordagem garante consistência e governança: todos os dashboards utilizam a mesma definição de “lucro”, “cliente ativo” ou “taxa de conversão”.  
  O ponto fraco é a necessidade de uma equipe técnica para manter e evoluir os modelos.

  \item \textbf{Apache Superset e Metabase: o código aberto do BI}  
  Ambas são ferramentas \textit{open-source} que trazem a filosofia “SQL-first” — isto é, permitem criar dashboards diretamente a partir de consultas SQL.  
  O \textbf{Superset}, mantido pela Apache Foundation, é mais completo, mas requer configuração e infraestrutura (DevOps). Já o \textbf{Metabase} oferece simplicidade e boa experiência de uso, sendo indicado para times menores ou ambientes acadêmicos.  
  Em contrapartida, essas soluções podem ter limitações em performance e em recursos de governança.

\end{itemize}

\begin{NoteBox}
\textbf{Critério didático:}  
Para iniciantes, a ferramenta ideal deve equilibrar \textbf{facilidade de construção}, \textbf{repertório visual amplo} e \textbf{potencial narrativo}.  
Por esse motivo, este curso utiliza o \textbf{Tableau Public} — uma plataforma gratuita, intuitiva e rica em recursos visuais, capaz de introduzir os princípios fundamentais da Análise Exploratória de Dados (AED) sem exigir conhecimento prévio em programação.
\end{NoteBox}



\begin{table}[H]
\centering
\caption{Comparativo entre ferramentas de visualização/BI quanto à descrição, pontos fortes e pontos fracos.}
\label{tab:comparativo_ferramentas_bi}
\begin{tabular}{p{2.8cm} p{4cm} p{4cm} p{4cm}}
\hline
\textbf{Ferramenta} & \textbf{Descrição} & \textbf{Pontos Fortes} & \textbf{Pontos Fracos} \\ \hline

\textbf{Excel} &
Planilha tradicional desktop/cloud voltada para análises rápidas e relatórios. &
Onipresente; fácil uso; boa para protótipos e análises locais. &
Baixa escalabilidade; pouca governança; difícil automação. \\

\textbf{Google Sheets} &
Planilha colaborativa online com integração ao ecossistema Google. &
Colaboração em tempo real; integra com BigQuery e Apps Script. &
Limite de linhas; recursos analíticos limitados; depende da internet. \\

\textbf{Power BI} &
Ferramenta da Microsoft para modelagem tabular, dashboards e relatórios corporativos. &
Alta integração com Excel e Azure; visualizações ricas; automação via DAX. &
Curva de aprendizado do DAX; limitações de compartilhamento gratuito. \\

\textbf{Tableau} &
Plataforma de visualização interativa voltada para exploração e storytelling. &
Visualizações sofisticadas; interface intuitiva; recursos de história (story points). &
Licença cara; modelagem de dados limitada; curva de aprendizado técnica. \\

\textbf{Looker Studio (Data Studio)} &
Ferramenta gratuita do Google para dashboards conectados a múltiplas fontes. &
Simples de usar; integra com Sheets e BigQuery; ideal para relatórios rápidos. &
Baixa performance com grandes volumes; governança limitada. \\

\textbf{Python (Pandas + Plotly)} &
Ecossistema em código para análise exploratória e automação de dashboards. &
Altamente flexível; reprodutível; ideal para análises avançadas. &
Requer conhecimento técnico; setup complexo; pouca interface nativa. \\

\textbf{R (tidyverse + ggplot2/Shiny)} &
Linguagem estatística voltada à análise e comunicação de dados. &
Visualizações consistentes; poderosa para estatística e prototipagem. &
Curva de aprendizado íngreme; menor uso em ambientes corporativos. \\

\textbf{Apache Superset} &
Ferramenta open-source para exploração SQL e dashboards interativos. &
Gratuita; escalável; integra com múltiplos bancos de dados. &
Exige infraestrutura própria; menos polida que opções pagas. \\

\textbf{Metabase} &
Plataforma open-source de BI com interface intuitiva e perguntas guiadas. &
Interface amigável; fácil configuração; ideal para análises rápidas. &
Limites em modelagens complexas; recursos avançados na versão paga. \\

\textbf{Qlik Sense} &
Plataforma corporativa de BI com motor associativo para descoberta de dados. &
Rápida exploração de grandes volumes; recursos corporativos maduros. &
Licenciamento complexo; curva de aprendizado alta; custo elevado. \\ \hline
\end{tabular}

\vspace{0.2cm}
\footnotesize
\textit{Legenda:} Comparativo das principais ferramentas utilizadas em Análise Exploratória de Dados (AED) e Business Intelligence (BI) quanto à sua aplicabilidade prática.

\end{table}

\begin{NoteBox}
\textbf{Importante}: não traduzimos termos técnicos como \textbf{extract}, \textbf{VizQL}, \textbf{LOD}. 
Eles aparecem como tais nas interfaces e na documentação oficial — preservar o termo aumenta a precisão didática.
\end{NoteBox}

% =========================================================
% 5.2 — Por que Tableau neste curso (com honestidade)
% =========================================================

\section{Por que o Tableau Public para este curso?}
\textbf{Prós didáticos}:
\begin{itemize}[leftmargin=*]
  \item Construção visual por \textbf{arrastar e soltar} com \textbf{alto repertório} (mapas, dual-axis, densidade, \textit{story points});
  \item \textbf{VizQL}: o gráfico responde \textit{em tempo real} ao gesto do analista, favorecendo a exploração;
  \item \textbf{LOD} e cálculos de tabela que resolvem KPIs estáveis (sem recorrer a código externo);
  \item \textbf{Publicação} simples no \textbf{Tableau Public} (portfólios dos alunos).
\end{itemize}
\textbf{Cuidados}:
\begin{itemize}[leftmargin=*]
  \item Modelagem descuidada \(\Rightarrow\) duplicações e métricas inconsistentes;
  \item Mau uso de filtros/ordem de operações \(\Rightarrow\) números \textit{aparentemente} diferentes;
  \item \textbf{Extract} é preferível ao \textit{live} para estabilidade no Public.
\end{itemize}

\begin{figure}[H]
\centering
\fbox{\rule{0pt}{120pt}\rule{0.92\linewidth}{0pt}}
\caption{[Espaço reservado] Capturas da interface do Tableau: \textit{Data Source}, \textit{Sheet}, \textit{Dashboard}, \textit{Story}.}
\end{figure}

% =========================================================
% 5.3 — Arquitetura do Tableau (detalhada, sem performance ainda)
% =========================================================

\section{Arquitetura do Tableau (visão detalhada e crítica)}

Compreender a arquitetura do Tableau é fundamental para projetar análises escaláveis e de alto desempenho.  
O Tableau não é apenas um criador de gráficos — ele é um \textbf{motor de consultas visuais} (\textbf{VizQL}) que traduz ações de interface (como arrastar um campo) em comandos SQL otimizados, executados sobre um modelo de dados em duas camadas.  
A seguir, detalhamos como esse fluxo ocorre \textbf{por trás da interface}, quais são os pontos de atenção e como projetar conexões eficientes.

\subsection*{1) Modelo de Dados em Duas Camadas: Lógica e Física}
O Tableau separa sua modelagem em dois níveis, o que explica por que às vezes ele parece “decidir sozinho” como unir tabelas.

\begin{itemize}[leftmargin=*]
  \item \textbf{Camada Lógica (Relationships)} — descreve como tabelas se relacionam sem fixar o tipo de \textit{join} antecipadamente.  
  Essa camada é mais \textbf{semântica}: o Tableau entende as chaves e decide, em tempo de execução, como consultar cada tabela, respeitando sua granularidade.  
  Ideal quando as tabelas têm níveis distintos de detalhe (ex.: Vendas diárias e Metas mensais).

  \item \textbf{Camada Física (Joins)} — define as junções reais de dados (\textit{inner, left, right, full}).  
  Aqui há maior controle, mas também maior risco de \textbf{duplicação de registros} e perda de performance, principalmente em bases grandes.  
  O Tableau executa esses joins diretamente na fonte de dados, o que significa que a modelagem física ruim pode gerar queries lentas e pesadas.
\end{itemize}

\begin{NoteBox}
\textbf{Atenção:} o Tableau nunca “guarda” seus dados — ele apenas os interpreta.  
A qualidade e granularidade do modelo de origem são o que definem a fluidez das suas visualizações.
\end{NoteBox}

\subsection*{2) Hyper Data Engine: o cérebro colunar do Tableau}
O \textbf{Hyper} é o mecanismo interno do Tableau, um formato colunar (\texttt{.hyper}) otimizado para leitura analítica e compressão.  
Ele é usado em \textbf{Extracts}, que são snapshots da base de dados.

\begin{itemize}[leftmargin=*]
  \item \textbf{Extract (\texttt{.hyper})}: reduz a dependência de consultas diretas à fonte, aumentando a performance.  
  Cada \textit{extract} pode ser atualizado manualmente ou via agendamento (no Tableau Cloud/Server).  
  \item \textbf{Leitura colunar}: o Tableau lê apenas as colunas necessárias à visualização, não a tabela inteira — isso explica por que gráficos simples são instantâneos mesmo em grandes bases.
  \item \textbf{Semântica de tipos}: suporta campos numéricos, categóricos, datas e geográficos (\textit{geo-roles}), garantindo coerência nos cálculos e mapas.
\end{itemize}

\begin{FormulaBox}
\textbf{Boas práticas de conexão performática:}\\[4pt]
\(\text{Fonte de dados limpa} \;\rightarrow\; \text{Extract (.hyper)} \;\rightarrow\; \text{Filtros de contexto} \;\rightarrow\; \text{Visualização otimizada}\)
\end{FormulaBox}

\subsection*{3) VizQL — A linguagem visual de consulta}
O \textbf{VizQL (Visual Query Language)} é o componente que transforma ações gráficas em comandos de consulta SQL.  
Cada arrastar de campo aciona uma query invisível que:
\begin{enumerate}[leftmargin=*]
  \item Consulta a fonte de dados (ou o extract);
  \item Agrega e agrupa resultados segundo dimensões e medidas;
  \item Retorna o resultado e o converte em \textit{marcas} visuais (pontos, linhas, barras, áreas);
  \item Renderiza os canais visuais (posição, cor, tamanho, forma).
\end{enumerate}

\begin{NoteBox}
\textbf{Interpretação prática:} quando o usuário arrasta \textit{Vendas} para \textit{Linhas} e \textit{Região} para \textit{Colunas},  
o Tableau gera algo conceitualmente próximo de:  
\texttt{SELECT Região, SUM(Vendas) FROM base GROUP BY Região;}
\end{NoteBox}

\subsection*{4) A camada semântica: onde o Tableau entende o seu negócio}
O Tableau organiza o modelo semântico por meio de \textbf{Dimensões}, \textbf{Medidas} e \textbf{Hierarquias},  
permitindo navegar entre diferentes granularidades sem reescrever a consulta.

\begin{itemize}[leftmargin=*]
  \item \textbf{Dimensões} — campos categóricos (Produto, Região, Data); definem agrupamentos e eixos.  
  \item \textbf{Medidas} — campos numéricos agregáveis (Vendas, Lucro, Receita).  
  \item \textbf{Hierarquias} — permitem o \textit{drill-down} (Ano → Trimestre → Mês; País → Estado → Cidade).  
  \item \textbf{Funções de papel} — controlam o comportamento visual (discrete vs. continuous, \textit{geo-role}, \textit{date-part} vs. \textit{date-value}).
\end{itemize}

\begin{NoteBox}
\textbf{Atenção ao contexto:} toda agregação no Tableau depende das dimensões em uso na \textit{view}.  
Mudar um campo de lugar altera a consulta SQL gerada.
\end{NoteBox}

\subsection*{5) Estratégias de conexão e escolha da tabela ideal}
Nem toda tabela é adequada para ser conectada diretamente ao Tableau.  
O tipo de tabela (e sua granularidade) determina o desempenho e a clareza analítica.

\begin{itemize}[leftmargin=*]
  \item \textbf{Tabelas Especializadas Analíticas (grandes)} — contêm dados detalhados, linha a linha, com milhões de registros.  
  São ideais para exploração, mas exigem cuidado: use \textbf{Extracts} e \textbf{filtros de contexto}.  
  Evite cálculos linha a linha no Tableau; faça-os previamente no Python ou SQL.

  \item \textbf{Tabelas Especializadas Sumarizadas} — já agregam as informações (por mês, por região, por produto).  
  Reduzem volume e aumentam performance, mas limitam o nível de detalhe.  
  São ideais para dashboards e análises gerenciais.

  \item \textbf{Tabelas Visão Insight} — derivadas de análises ou modelos; contêm KPIs prontos, métricas preditivas e classificações.  
  Excelentes para storytelling e publicação, mas não para exploração — o dado já vem “interpretado”.
\end{itemize}

\begin{NoteBox}
\textbf{Resumo prático:}\\[4pt]
\begin{itemize}[leftmargin=*]
  \item \textbf{Exploração →} use bases analíticas grandes (com Extract e filtros);\\
  \item \textbf{Dashboards →} use tabelas sumarizadas (pré-agregadas);\\
  \item \textbf{Histórias e KPIs →} use visões insight (curadas e leves).
\end{itemize}
\end{NoteBox}

\subsection*{6) Cálculos e LOD (Level of Detail)}
Os cálculos no Tableau atuam em níveis diferentes:
\begin{itemize}[leftmargin=*]
  \item \textbf{Campos calculados} — criam métricas derivadas: \texttt{Lucro = Vendas - Custo}.
  \item \textbf{LOD Expressions} — controlam o nível de agregação:  
    \texttt{\{FIXED Região: SUM(Vendas)\}} mantém a soma fixa por região, mesmo com filtros.  
  \item \textbf{Cálculos de Tabela} — operam sobre a visualização (percentuais, rankings, somas cumulativas).
\end{itemize}

\begin{FormulaBox}
\textbf{Regra geral:} use LOD para \textbf{garantir consistência de KPIs} e cálculos de tabela para \textbf{comparações dinâmicas na visualização}.
\end{FormulaBox}

\subsection*{7) Dashboards, Stories e Metadados}
No Tableau, o produto final não é um gráfico, mas uma \textbf{interface interativa}.  
\textbf{Dashboards} combinam múltiplas visualizações com filtros, ações e parâmetros.  
\textbf{Stories} conectam visualizações em sequência narrativa, guiando o público até o insight.

\begin{NoteBox}
\textbf{Resumo crítico:} dominar a arquitetura do Tableau é entender o que acontece entre o clique e o gráfico.  
Ao conhecer a camada lógica, o Hyper, o VizQL e as granularidades, o analista deixa de ser apenas um “usuário de painéis” e passa a ser um \textbf{projetista de performance e significado}.
\end{NoteBox}

\begin{figure}[H]
\centering
\fbox{\rule{0pt}{120pt}\rule{0.92\linewidth}{0pt}}
\caption{[Espaço reservado] Arquitetura conceitual do Tableau: Relationships (lógica) → Joins (física) → VizQL → Marcas/Canais → Renderização; Hyper (\texttt{.hyper}) como camada colunar de alto desempenho.}
\end{figure}


% =========================================================
% 5.4 — (Transição) A partir daqui, entramos na prática com Tableau
% =========================================================
\section{Visão Geral do Tableau Public}

O \textbf{Tableau Public} é a porta de entrada para o ecossistema Tableau e um excelente ambiente de aprendizagem.  
Ele permite \textbf{criar, publicar e compartilhar visualizações interativas gratuitamente}, sem exigir licenças corporativas.  
Tudo é salvo na nuvem do Tableau (perfil público), o que facilita a criação de portfólios e a divulgação de projetos analíticos.

\begin{NoteBox}
\textbf{Por que o Tableau Public neste curso?}  
Porque ele une três pilares importantes da aprendizagem em AED:  
(i) experimentação prática e visual;  
(ii) narrativa e comunicação de dados;  
(iii) publicação e compartilhamento dos resultados com a comunidade.
\end{NoteBox}

\subsection*{Fluxo de trabalho no Tableau Public}
A construção de uma análise completa no Tableau segue um \textbf{ciclo lógico}, do dado ao insight:

\begin{enumerate}[leftmargin=*]
  \item \textbf{Conectar e modelar os dados} — importar arquivos (\texttt{.csv}, \texttt{.xlsx}, \texttt{.json}) ou bancos de dados; definir \textit{relationships} (camada lógica) e \textit{joins} (camada física); escolher se a conexão será \textbf{live} ou via \textbf{extract (.hyper)}.
  
  \item \textbf{Construir visualizações} — selecionar campos (dimensões e medidas), aplicar filtros, criar campos calculados e explorar diferentes tipos de gráfico conforme o tipo de variável e o objetivo analítico.
  
  \item \textbf{Montar dashboards e stories} — organizar múltiplas visualizações em painéis interativos; criar narrativas visuais (\textit{stories}) com contexto, conflito e conclusão.
  
  \item \textbf{Publicar e compartilhar} — enviar o projeto ao \textbf{Tableau Public}, tornando-o acessível por link ou incorporável em sites e portfólios.
\end{enumerate}

\begin{FormulaBox}
\textbf{Fluxo essencial do trabalho:}\\[3pt]
\texttt{Dados → Modelagem → Visualização → Interação → Storytelling → Publicação}
\end{FormulaBox}

\begin{figure}[H]
\centering
\fbox{\rule{0pt}{110pt}\rule{0.9\linewidth}{0pt}}
\caption{[Espaço reservado] Interface principal do Tableau Public: Data Source → Sheet → Dashboard → Story.}
\end{figure}

\subsection*{O ecossistema Tableau e seus componentes}
O Tableau é formado por um conjunto de produtos que compartilham a mesma base tecnológica (\textbf{VizQL} e \textbf{Hyper}).  
A diferença está na finalidade, no público e na forma de armazenamento dos dados.

\begin{itemize}[leftmargin=*]
  \item \textbf{Tableau Desktop} — versão completa, usada em contextos corporativos; conecta-se a múltiplas fontes (SQL, AWS, BigQuery, etc.) e permite automações complexas.
  
  \item \textbf{Tableau Public} — versão gratuita e em nuvem; trabalha principalmente com \textbf{extracts} locais de arquivos e publica automaticamente no perfil do usuário; ideal para aprendizado, prototipagem e portfólios públicos.
  
  \item \textbf{Tableau Cloud / Server} — ambientes corporativos de publicação privada; controlam segurança, acesso e agendamento de atualizações de \textbf{extracts}.
\end{itemize}

\begin{NoteBox}
\textbf{Diferença prática:}  
No \textbf{Public}, o dado é sempre publicado junto com a visualização, por isso não é indicado para informações sigilosas.  
No \textbf{Server/Cloud}, o dado pode permanecer protegido e as permissões são definidas por perfis de acesso.
\end{NoteBox}

\subsection*{Boas práticas iniciais}
Antes de começar a construir gráficos, o analista deve preparar o ambiente para evitar inconsistências e manter reprodutibilidade.

\begin{itemize}[leftmargin=*]
  \item \textbf{Padronize os nomes dos campos} — use convenções claras (sem acentos, espaços ou duplicidades).  
  \item \textbf{Mantenha um dicionário de dados} — descreva o significado e a unidade de cada variável.  
  \item \textbf{Organize as pastas do projeto} — separe dados brutos, processados e finais.  
  \item \textbf{Versione os arquivos} — salve as etapas de construção do dashboard para evitar perda de progresso.  
  \item \textbf{Salve periodicamente como \texttt{.twbx}} — formato compactado que inclui os dados e o layout (ideal para backup ou portfólio offline).
\end{itemize}

\begin{NoteBox}
\textbf{Dica didática:} pensar o Tableau como um \textbf{laboratório visual}.  
Cada gráfico é um experimento: testamos hipóteses, validamos padrões e ajustamos a narrativa.  
A força da ferramenta está na iteração rápida — criar, testar, refinar e comunicar.
\end{NoteBox}

% [A sessão de performance virá posteriormente em seção própria, conforme o plano do capítulo.]


\section{Get Started with Tableau Prep}

O \textbf{Tableau Prep} é o componente do ecossistema Tableau voltado à preparação e limpeza de dados antes da análise.  
Ele permite criar fluxos (\textit{flows}) que conectam, transformam, combinam e exportam conjuntos de dados limpos e prontos para o \textbf{Tableau Desktop} ou o \textbf{Tableau Public}.  
A seguir, veremos um guia completo baseado no tutorial oficial da Salesforce\footnote{Fonte: \url{https://help.tableau.com/current/prep/en-us/prep_get_started.htm?_gl=1*2dbq0u*_gcl_au*MTE2NDcwMjI0OS4xNzYwMDAwNDIy*_ga*MTU1MDY5NDQ1MS4xNzYwMDAwNDIy*_ga_8YLN0SNXVS*czE3NjAzNzU5OTkkbzIkZzEkdDE3NjAzNzYwODEkajU1JGwwJGgw}}.

\subsection*{Objetivo do Tableau Prep}

A função do Tableau Prep é transformar dados brutos, incompletos ou despadronizados em uma estrutura analítica limpa e confiável.  
Ele atua no início do processo da AED — na etapa de \textbf{ingestão, transformação e organização dos dados} — automatizando tarefas que normalmente exigiriam SQL, Python ou planilhas complexas.

\begin{FormulaBox}
\textbf{Fluxo conceitual da preparação de dados:}\\[4pt]
\(\text{Fonte bruta} \Rightarrow \text{Conexão} \Rightarrow \text{Limpeza e padronização} \Rightarrow \text{Integração (joins/unions)} \Rightarrow \text{Exportação para análise}\)
\end{FormulaBox}

\begin{NoteBox}
\textbf{Importância prática:} Um bom analista deve dominar tanto o Tableau Desktop (análise) quanto o Tableau Prep (preparação).  
Sem uma base de dados limpa, nenhuma visualização é confiável — mesmo que o gráfico seja bonito.
\end{NoteBox}

---

\subsection*{1. Conectando-se aos dados}

Ao abrir o Tableau Prep Builder, a tela inicial apresenta o painel \textbf{Connections}, semelhante ao Tableau Desktop.  
É nele que se inicia o fluxo de trabalho:

\begin{enumerate}[leftmargin=*]
  \item Clique em \textbf{Add Connection} e selecione o tipo de arquivo (ex.: \texttt{.csv}, \texttt{.xlsx}, \texttt{.json}).
  \item Navegue até o diretório desejado e selecione os arquivos de entrada (\textit{Input Step}).
  \item Se houver múltiplos arquivos semelhantes (ex.: \texttt{Orders\_2015.csv}, \texttt{Orders\_2016.csv}), use a opção \textbf{Union multiple tables}.
  \item Cada Input Step cria automaticamente uma amostra dos dados para melhorar o desempenho e exibir prévias dos campos e tipos.
\end{enumerate}

\begin{NoteBox}
\textbf{Dica:} Utilize nomes consistentes e mantenha seus arquivos em pastas organizadas.  
Isso permite que o Tableau Prep combine automaticamente arquivos com estrutura idêntica (como séries anuais ou regionais).
\end{NoteBox}

---

\subsection*{2. Explorando e avaliando a estrutura dos dados}

Após conectar-se às fontes, cada etapa é representada como um \textbf{nó visual} no \textit{Flow Pane}.  
O Tableau exibe os campos, tipos de dados e amostras, permitindo que você:

\begin{itemize}[leftmargin=*]
  \item visualize campos duplicados (ex.: colunas com prefixo \texttt{Right\_});
  \item filtre colunas indesejadas diretamente na entrada;
  \item corrija tipos de dados (ex.: converter “String” para “Decimal”);
  \item e avalie a consistência de chaves e granularidades.
\end{itemize}

\begin{FormulaBox}
\textbf{Dica técnica:} o Tableau Prep processa amostras parciais para preservar a performance,  
mas é possível ajustar o tamanho da amostra na aba \textbf{Data Sample}.
\end{FormulaBox}

---

\subsection*{3. Limpando e modelando os dados}

A etapa de limpeza é feita através de \textbf{Clean Steps}, que permitem corrigir campos e valores por meio de uma interface visual.

\begin{enumerate}[leftmargin=*]
  \item Adicione um \textbf{Clean Step} clicando no botão “+” após o Input Step.
  \item Use o painel \textbf{Profile Pane} para identificar outliers, nulos e inconsistências.
  \item Aplique operações de limpeza como:
  \begin{itemize}[leftmargin=*]
    \item \textbf{Remover campos duplicados};
    \item \textbf{Combinar colunas} com cálculos como \texttt{MAKEDATE([Ano],[Mês],[Dia])};
    \item \textbf{Corrigir tipos de dados} e \textbf{substituir valores nulos};
    \item \textbf{Criar campos calculados} (ex.: \texttt{Region = "Central"}).
  \end{itemize}
\end{enumerate}

\begin{NoteBox}
\textbf{Boas práticas:}
\begin{itemize}[leftmargin=*]
  \item Dê nomes descritivos aos passos (ex.: \textit{Fix Dates/Field Names});  
  \item Revise o painel \textbf{Changes} — ele registra todas as transformações realizadas;  
  \item Utilize o \textbf{Group Values} para padronizar entradas semelhantes (como abreviações de estados).
\end{itemize}
\end{NoteBox}

---

\subsection*{4. Combinando tabelas: \textit{Union} e \textit{Join}}

O Tableau Prep permite combinar fontes por linhas (\textit{union}) ou por colunas (\textit{join}) de forma visual e intuitiva.

\paragraph{Union (união por linhas):}
\begin{itemize}[leftmargin=*]
  \item Use quando arquivos têm a mesma estrutura, mas representam períodos ou regiões diferentes.
  \item Basta arrastar uma etapa sobre a outra e escolher \textbf{Union}.
  \item Campos com nomes e tipos compatíveis são mesclados automaticamente.
\end{itemize}

\paragraph{Join (união por colunas):}
\begin{itemize}[leftmargin=*]
  \item Use quando deseja combinar informações complementares (ex.: pedidos e devoluções).
  \item Escolha as chaves de junção (ex.: \texttt{Order ID}, \texttt{Product ID}) e o tipo de join (inner, left, right, full).
  \item O painel \textbf{Join Profile} mostra estatísticas do join, incluindo linhas incluídas/excluídas e campos ausentes.
\end{itemize}

\begin{NoteBox}
\textbf{Exemplo prático:}  
Após unir os arquivos de pedidos (\texttt{Orders\_South\_2015–2018}) com o de devoluções (\texttt{returns\_reasons\_new.xlsx}),  
podemos criar campos calculados como:
\begin{itemize}[leftmargin=*]
  \item \texttt{Returned? = IF ISNULL([Return Reason]) THEN "No" ELSE "Yes" END}
  \item \texttt{Days to Ship = DATEDIFF('day',[Order Date],[Ship Date])}
\end{itemize}
\end{NoteBox}

---

\subsection*{5. Gerando saídas e integrando com o Tableau Desktop}

Depois de limpar e unir os dados, o Tableau Prep permite exportar o resultado em diferentes formatos:

\begin{itemize}[leftmargin=*]
  \item \textbf{.hyper} — formato colunar nativo e otimizado do Tableau (recomendado);
  \item \textbf{.csv} — para compatibilidade com outras ferramentas;
  \item \textbf{.tflx} — pacote que inclui o fluxo e os arquivos de dados;
  \item \textbf{Publicação direta} — envia o resultado ao Tableau Cloud/Server como \textbf{data source}.
\end{itemize}

\begin{FormulaBox}
\textbf{Ciclo completo do Tableau Prep:}\\[3pt]
\(\text{Conectar} \Rightarrow \text{Limpar} \Rightarrow \text{Unir/Agregar} \Rightarrow \text{Calcular} \Rightarrow \text{Gerar Output (.hyper)}\)
\end{FormulaBox}

\begin{NoteBox}
\textbf{Dica de automação:}  
No Tableau Server, é possível agendar execuções automáticas de fluxos (via \textit{Tableau Prep Conductor})  
para manter os dados sempre atualizados (\textit{incremental refresh}).
\end{NoteBox}

---

\subsection*{Resumo da aprendizagem}

\begin{itemize}[leftmargin=*]
  \item \textbf{Tableau Prep} é uma ferramenta visual de ETL (\textit{Extract, Transform, Load}) voltada à análise.  
  \item Ideal para usuários que desejam preparar dados sem precisar de código.  
  \item Integra-se perfeitamente ao \textbf{Tableau Desktop/Public} via arquivos \texttt{.hyper}.  
  \item Reduz tempo de limpeza, aumenta a consistência dos relatórios e evita dependência de planilhas.  
\end{itemize}

\begin{NoteBox}
\textbf{Conclusão:}  
Com o Tableau Prep, o analista deixa de ser apenas “consumidor” de dados e passa a ser o \textbf{arquiteto da qualidade analítica}.  
A ferramenta transforma o processo de limpeza em uma experiência visual, rápida e reprodutível — essencial em qualquer projeto de AED.
\end{NoteBox}


\section{Modelagem de Dados no Tableau (Desktop, Server e Cloud)}

A modelagem de dados no Tableau organiza \textbf{como as tabelas são conectadas, interpretadas e expostas} para análise.  
O objetivo é entregar \textbf{consistência semântica} (dimensões, medidas, tipos) e \textbf{confiabilidade operacional} (joins corretos, granularidade estável, campos calculados reprodutíveis).

\begin{FormulaBox}
\textbf{Princípio orientador:}\\[3pt]
\(\text{Fonte(s)} \Rightarrow \text{Modelo (lógico + físico)} \Rightarrow \text{Semântica (dimensões/medidas)} \Rightarrow \text{Visualizações e KPIs}\)
\end{FormulaBox}

\subsection*{1. Duas camadas de modelagem: lógica e física}
\begin{itemize}[leftmargin=*]
  \item \textbf{Camada Lógica (\textit{Relationships})} — descreve relações entre tabelas \emph{sem fixar} joins de imediato.  
  Preserva granularidades diferentes (ex.: \textit{Pedidos} diários vs. \textit{Metas} mensais). O Tableau resolve a junção \textit{em tempo de consulta} conforme o contexto da visualização.
  \item \textbf{Camada Física (Joins)} — define os joins efetivos (\textit{inner/left/right/full}) quando você arrasta tabelas para dentro de uma mesma \textit{physical layer}.  
  Atenção à \textbf{cardinalidade} (1:1, 1:N, N:N) para evitar \textit{duplicações} e perda de performance.
\end{itemize}

\begin{NoteBox}
\textbf{Regra prática:} Use \textbf{Relationships} quando as tabelas têm \textbf{granularidades diferentes}. Use \textbf{Joins} quando as tabelas têm granularidades compatíveis e você \textbf{realmente precisa} materializar a junção.
\end{NoteBox}

\subsection*{2. Editar e gerenciar a fonte de dados}
\begin{enumerate}[leftmargin=*]
  \item \textbf{Editar a fonte} — \texttt{Data \(\rightarrow\) <Sua Fonte> \(\rightarrow\) Edit Data Source}.  
  Na \textit{Data Source page}, ajuste conexões, adicione tabelas, defina \textit{relationships/joins}.
  \item \textbf{Navegar no \textit{Data Grid}} — visualize amostra dos dados, \textbf{ordene colunas/linhas}, oculte campos, mude \textbf{tipos de dados} e \textbf{georoles}.
  \item \textbf{Renomear/Resetar nomes} — duplo clique para renomear; menu do campo \(\rightarrow\) \textit{Reset Name} para voltar ao nome original da fonte.
  \item \textbf{Reverter nomes automáticos} — use \textit{Revert} quando o Tableau tiver aprimorado nomes automaticamente e você quiser restaurar.
\end{enumerate}

\begin{NoteBox}
\textbf{Boas práticas de metadados:}
\begin{itemize}[leftmargin=*]
  \item Padronize nomes (\texttt{snake\_case} ou \texttt{CamelCase}), sem acentos e sem espaços.  
  \item Agrupe campos por pastas temáticas (Dimensão Tempo, Geografia, Produto, Fato Vendas).
  \item Oculte (\textit{Hide}) campos técnicos que não entram na análise.
\end{itemize}
\end{NoteBox}

\subsection*{3. Joins na prática (camada física)}
Para adicionar tabelas via \textbf{join}:
\begin{enumerate}[leftmargin=*]
  \item Na \textbf{Data Source page}, arraste a nova tabela para o canvas físico, \textbf{sobrepondo} à tabela existente até aparecer o ícone de join.
  \item Escolha o \textbf{tipo de join} no diagrama (inner/left/right/full).
  \item Defina as \textbf{chaves de junção} (ex.: \texttt{Orders.OrderID} = \texttt{Returns.OrderID}).  
  \item Revise o \textbf{perfil do join}: linhas incluídas/excluídas e \textbf{mismatch} de campos.
\end{enumerate}

\begin{FormulaBox}
\textbf{Tipos de join (resumo):}\\[3pt]
\(\textbf{INNER}\) — mantém apenas correspondências; \(\;\)
\(\textbf{LEFT}\) — preserva a esquerda + correspondências; \(\;\)
\(\textbf{RIGHT}\) — preserva a direita + correspondências; \(\;\)
\(\textbf{FULL}\) — preserva ambas, preenchendo ausências com \texttt{NULL}.
\end{FormulaBox}

\begin{NoteBox}
\textbf{Evite duplicação:} Em joins 1:N ou N:N, medidas podem ser multiplicadas.  
Soluções: \textbf{Relationships} na camada lógica, \textbf{LOD FIXED} para KPIs estáveis, ou pré-agregação na fonte.
\end{NoteBox}

\subsection*{4. Union, agregação e consistência de campos}
\begin{itemize}[leftmargin=*]
  \item \textbf{Union (linha a linha):} combine arquivos/tabelas com \textbf{mesma estrutura} (ex.: regiões ou anos).  
  Após o union, use \textit{merge fields} para unificar nomes (ex.: \texttt{Product} vs. \texttt{Product Name}).
  \item \textbf{Agregação:} crie \textbf{tabelas sumarizadas} (mês, região, categoria) para acelerar painéis gerenciais.
\end{itemize}

\subsection*{5. Cálculos no nível certo}
\begin{itemize}[leftmargin=*]
  \item \textbf{Campos calculados} — criam métricas derivadas (ex.: \texttt{Lucro = Vendas - Custo}).
  \item \textbf{LOD Expressions} — controlam a granularidade do cálculo independentemente da view:  
  \(\{\texttt{FIXED Região: SUM(Vendas)}\}\) fixa o KPI por região \emph{mesmo se} filtros mudarem a exibição.
  \item \textbf{Table Calculations} — operam sobre a visualização (rank, \% do total, \textit{running sum}); dependem de \textit{addressing/partitioning}.
\end{itemize}

\begin{NoteBox}
\textbf{Dica curricular:} LOD para \textbf{consistência de KPI}. Table Calc para \textbf{comparação dinâmica} na view.
\end{NoteBox}

\subsection*{6. Custom SQL e Stored Procedures (Desktop)}
\begin{itemize}[leftmargin=*]
  \item \textbf{Custom SQL} — conecte-se a uma consulta SQL escrita por você (filtros, CTEs, pré-agregações, janelas analíticas).  
  Útil para padronizar nomes, resolver \textbf{N:N} antes do Tableau e expor apenas colunas necessárias.
  \item \textbf{Stored Procedures} — execute lógica pré-definida no banco (limpeza, regras de negócio, auditoria).  
  Entregue ao Tableau uma \textbf{view} mais estável e performática.
\end{itemize}

\begin{FormulaBox}
\textbf{Padrão robusto:}\\[3pt]
\(\text{Camada SQL (views/procs)} \Rightarrow \text{Relationships} \Rightarrow \text{Joins cirúrgicos (se preciso)} \Rightarrow \text{Semântica clara}\)
\end{FormulaBox}

\subsection*{7. Inspeção e manutenção de metadados}
\begin{itemize}[leftmargin=*]
  \item \textbf{Metadata Grid} — cada linha representa um campo; ajuste \textbf{tipo}, \textbf{nome}, \textbf{tabela física} e \textbf{nome remoto}; altere \textbf{papel geográfico}.
  \item \textbf{View Extract Data} — ao usar WDC ou \textit{Extract mode}, saiba que a \textbf{ordem de linhas} pode diferir do \textit{Live mode}.
  \item \textbf{Copiar valores} — selecione células no grid \(\rightarrow\) \textit{Copy} para inspeções rápidas.
\end{itemize}

\subsection*{8. Trocar, substituir e duplicar fontes}
\begin{itemize}[leftmargin=*]
  \item \textbf{Editar conexão} — \textit{Data pane} \(\rightarrow\) \textit{Edit Connection} para apontar para novo caminho/servidor.
  \item \textbf{Replace References} — ao trocar a fonte, mapeie \textbf{campos inválidos} (ex.: \texttt{Customer Name} \(\rightarrow\) \texttt{Name}).
  \item \textbf{Renomear data source} — \textit{Data} \(\rightarrow\) \textit{Rename} para identificar ambientes (\texttt{prod}, \texttt{dev}).
  \item \textbf{Duplicar data source} — \textit{Data} \(\rightarrow\) \textit{Duplicate}; teste mudanças \emph{sem} impactar as \textit{sheets} existentes.
\end{itemize}

\begin{NoteBox}
\textbf{Higiene operacional:} use sufixos claros (\texttt{\_prod}, \texttt{\_staging}), documente \textit{Replace References} e registre mudanças estruturais (log/README do projeto).
\end{NoteBox}

\subsection*{9. Escolha do tipo de tabela para conectar}
\begin{itemize}[leftmargin=*]
  \item \textbf{Especializada Analítica (grande, detalhada)} — ótima para exploração; prefira \textbf{Extract} e \textbf{filtros de contexto}; evite \textit{row-by-row} pesados no Tableau.
  \item \textbf{Especializada Sumarizada} — agregada por período/segmento; \textbf{rápida} para dashboards gerenciais, porém menos flexível.
  \item \textbf{Visão Insight} — KPIs calculados e curadoria final; excelente para storytelling e publicação (leve), \textbf{não} ideal para exploração profunda.
\end{itemize}

\begin{SolvedBox}
\textbf{Exercício Resolvido (modelo de prova) — Join e KPI estável}\\[4pt]
\textbf{Objetivo:} unir \textit{Orders} e \textit{Returns}, criar \texttt{Returned?} e um KPI de vendas \textbf{fixo por região}.\\[4pt]
\textbf{Passo a passo:}\\
1) \textbf{Join físico (LEFT)}: \texttt{Orders.OrderID = Returns.OrderID}.\\
2) \textbf{Campo calculado} \texttt{Returned?}: \texttt{IF ISNULL([Return Reason]) THEN "No" ELSE "Yes" END}.\\
3) \textbf{KPI} \(\{\texttt{FIXED Região: SUM(Vendas)}\}\) — consistente mesmo com filtros por \textit{Categoria} ou \textit{Produto}.\\
4) Verifique duplicações: se \texttt{Orders} 1:N \texttt{Returns}, prefira \textbf{Relationship} ou agregue \texttt{Returns} antes do join.
\end{SolvedBox}

\begin{figure}[H]
\centering
\fbox{\rule{0pt}{120pt}\rule{0.92\linewidth}{0pt}}
\caption{[Espaço reservado] Esquema da \textit{Data Source page}: Relationships (camada lógica) \(\rightarrow\) Joins (camada física) \(\rightarrow\) Semântica (dimensões/medidas) \(\rightarrow\) \textit{Sheets}/Dashboards.}
\end{figure}

\begin{NoteBox}
\textbf{Checklist rápido (antes de publicar):}
\begin{enumerate}[leftmargin=*]
  \item Nomes e tipos dos campos padronizados; campos técnicos \textit{hidden}.  
  \item Joins revisados (cardinalidade e \textit{join type}); \textit{Relationships} quando há granularidades diferentes.  
  \item KPIs críticos com \textbf{LOD FIXED}; \textit{Table Calcs} apenas onde a dinâmica da view é desejada.  
  \item Se necessário, \textbf{Custom SQL/Stored Procedure} para entregar uma visão limpa e performática.
\end{enumerate}
\end{NoteBox}


\section{Dimensões, Medidas e Hierarquias}

Neste capítulo definimos os três pilares semânticos que organizam como interpretamos dados no Tableau: \textbf{dimensões}, \textbf{medidas} e \textbf{hierarquias}. Cada um tem papel distinto — entender essas diferenças é essencial para construir visualizações corretas e consistentes.

\subsection*{Dimensões (Dimensions)}

\begin{itemize}[leftmargin=*]
  \item São campos qualitativos ou categóricos que “quebram” ou segmentam os dados: exemplo: \texttt{Categoria}, \texttt{Região}, \texttt{Cliente}, \texttt{Data}.  
  \item Em consultas SQL correspondem a colunas do \texttt{GROUP BY} — elas definem o nível de detalhe (granularidade) da agregação.  
  \item No Tableau, dimensões não são agregadas automaticamente: você “quebra” as medidas com elas.  
  \item Por padrão, aparecem acima da linha cinza no painel de dados. :contentReference[oaicite:0]{index=0}  
  \item Podem ser discretas (\textit{discrete}, mostradas como cabeçalhos) ou contínuas (\textit{continuous}, mostradas como eixo). :contentReference[oaicite:1]{index=1}  
  \item Você pode converter uma medida para dimensão se o campo numérico não deve ser agregado (ex: códigos postais, IDs) usando \textit{Convert to Dimension}. :contentReference[oaicite:2]{index=2}  
\end{itemize}

\subsection*{Medidas (Measures)}

\begin{itemize}[leftmargin=*]
  \item São campos quantitativos que podem ser agregados (soma, média, contagem, mínimo, máximo etc.) — exemplo: \texttt{Vendas}, \texttt{Lucro}, \texttt{Quantidade}. :contentReference[oaicite:3]{index=3}  
  \item Quando arrastadas para a visualização, o Tableau aplica uma agregação por padrão (ex: \texttt{SUM([Vendas])}). :contentReference[oaicite:4]{index=4}  
  \item Podem ser discretas ou contínuas também — embora medidas contínuas (eixos) sejam mais comuns. :contentReference[oaicite:5]{index=5}  
  \item Algumas expressões de nível de detalhe (LOD) são consideradas medidas, como \textit{INCLUDE} e \textit{EXCLUDE}. :contentReference[oaicite:6]{index=6}  
  \item Também existe o conjunto gerado \texttt{Measure Values} / \texttt{Measure Names}, que permite combinar múltiplas medidas em uma mesma visualização. :contentReference[oaicite:7]{index=7}  
\end{itemize}

\subsection*{Hierarquias (Hierarchies)}

\begin{itemize}[leftmargin=*]
  \item São relações organizadas entre dimensões de granularidade crescente ou decrescente — por exemplo: \texttt{Ano → Trimestre → Mês}, \texttt{País → Estado → Cidade}. :contentReference[oaicite:8]{index=8}  
  \item Permitem que o usuário clique em \texttt{+ / –} nas visualizações para “drill down” ou “drill up” entre níveis. :contentReference[oaicite:9]{index=9}  
  \item Para criar uma hierarquia: no painel de dados, arraste um campo sobre outro e escolha “Create Hierarchy”. :contentReference[oaicite:10]{index=10}  
  \item Você pode ordenar e reorganizar os níveis dentro da hierarquia conforme a necessidade analítica. :contentReference[oaicite:11]{index=11}  
  \item A hierarquia não muda os dados, apenas organiza como são explorados (navegação entre granularidades).  
\end{itemize}

\subsection*{Relação entre Dimensões, Medidas e Visualização}

\begin{itemize}[leftmargin=*]
  \item As dimensões definem os “cortes” (linhas, colunas, categorias) sobre os quais as medidas são calculadas.  
  \item Em SQL, isso equivaleria a:
  \[
    \texttt{SELECT Dimensão, AGG(Measure) \;FROM\; Tabela \;GROUP BY\; Dimensão}
  \]
  \item Quando você adiciona mais dimensões na visualização, o número de “marcas” (pontos, barras) tende a aumentar — é como expandir o nível de detalhe. :contentReference[oaicite:12]{index=12}  
  \item Medidas isoladas sem dimensão produzem agregados gerais (ex: total de vendas).  
  \item Se converter uma medida para dimensão (caso especial), ela passa a “quebrar” outras medidas em categorias discretas. :contentReference[oaicite:13]{index=13}  
  \item Deve-se ter cuidado ao misturar níveis: muitos níveis de dimensão podem causar “mark explosion” (marcas demais).
\end{itemize}

\begin{NoteBox}
\textbf{Resumo conceitual:}\\
- \textbf{Dimensão} = atributo, categoria, divisória.  
- \textbf{Medida} = valor, métrica, agregável.  
- \textbf{Hierarquia} = estrutura de níveis para navegar granularidades.
\end{NoteBox}

\begin{SolvedBox}
\textbf{Exercício prático:}\\  
Dada a base com campos \texttt{Ano}, \texttt{Mês}, \texttt{Região}, \texttt{Vendas}, \texttt{Lucro}:  
\begin{itemize}
  \item Crie a hierarquia \texttt{Ano → Mês}.  
  \item Utilize \texttt{Região} como dimensão para “quebrar” os valores.  
  \item Plote \texttt{SUM(Vendas)} e \texttt{SUM(Lucro)} como medidas.  
  \item Experimente converter \texttt{Lucro} para dimensão e observe o que acontece.
\end{itemize}
\end{SolvedBox}


\section{Workbooks, Dashboards e Stories: visão conceitual e prática}

Nesta sessão você entenderá \textbf{o que é cada artefato no Tableau}, quando usar e como eles se relacionam no fluxo de AED — do experimento visual (worksheet) à \textit{curadoria} (dashboard) e à \textit{narrativa} (story).

\subsection{Workbook e Sheets (Worksheets)}
\textbf{Workbook} é o arquivo/container lógico que organiza tudo o que você cria: conexões de dados, planilhas (sheets), dashboards e stories. Você pode abrir múltiplos workbooks em janelas separadas, duplicar, renomear e organizar as \textit{sheets} por abas, \textit{filmstrip} ou \textit{sheet sorter}.\footnote{``Create or open a workbook'' e navegação/organização de sheets. Ver documentação oficial. \url{https://help.tableau.com/current/pro/desktop/en-us/environ_workbooksandsheets_workbooks.htm} e \url{https://help.tableau.com/current/pro/desktop/en-us/environ_workbooksandsheets_sheets_organize.htm}. :contentReference[oaicite:0]{index=0}}
\begin{itemize}[leftmargin=*]
  \item \textbf{Worksheet (sheet)}: uma \textit{vis} única (gráfico/mapa/tabela) construída arrastando dimensões e medidas, definindo marcas e agregações.
  \item \textbf{Gestão de sheets} em workbooks grandes: \textit{hide/show} de planilhas, navegação entre sheets, dashboards e stories para manter o projeto organizado. \footnote{``Manage Sheets in Dashboards and Stories'' (hide/show, navegação). \url{https://help.tableau.com/current/pro/desktop/en-us/environ_workbooksandsheets_sheets_hideshow.htm}. :contentReference[oaicite:1]{index=1}}
  \item \textbf{Empacotamento para distribuição}: um \texttt{.twbx} inclui o workbook e cópias de fontes locais/imagens para compartilhamento. \footnote{``Packaged Workbooks''. \url{https://help.tableau.com/current/pro/desktop/en-us/save_savework_packagedworkbooks.htm}. :contentReference[oaicite:2]{index=2}}
\end{itemize}

\subsection{Dashboards}
\textbf{Dashboard} combina múltiplas sheets em um só layout, adicionando objetos (texto, imagens, web, navegação, downloads) e \textbf{interatividade} (filtros, ações, realce) para responder perguntas do negócio de forma \textit{explorável}. Cria-se um dashboard como uma nova guia e arrastam-se sheets/objetos para o canvas; é possível \textit{substituir} uma sheet mantendo estilo do container e adicionar \textit{Use as Filter} para ação cruzada entre vis.\footnote{``Create a Dashboard'' — criação, troca de sheets, interatividade e objetos. \url{https://help.tableau.com/current/pro/desktop/en-us/dashboards_create.htm}. :contentReference[oaicite:3]{index=3}}
\begin{itemize}[leftmargin=*]
  \item \textbf{Boas práticas}: defina objetivo e público; coloque a \textit{vis} principal na região superior-esquerda; limite o número de \textit{views} (2–3) por painel; projete no \textit{final display size} e crie \textit{device layouts} para tablet/phone; use filtros e \textit{highlighter} para encorajar exploração. \footnote{``Best Practices for Effective Dashboards'' (propósito/audiência, layout, número de views, device layouts, filtros/highlighting). \url{https://help.tableau.com/current/pro/desktop/en-us/dashboards_best_practices.htm}. :contentReference[oaicite:4]{index=4}}
  \item \textbf{Layout e responsividade}: configure \textit{Fixed/Automatic/Range} e crie \textit{Device Designer} para entregar uma URL única com variações por dispositivo. \footnote{``Create Dashboard Layouts for Different Device Types''. \url{https://help.tableau.com/current/pro/desktop/en-us/dashboards_dsd_create.htm}. :contentReference[oaicite:5]{index=5}}
  \item \textbf{Segurança e objetos Web}: prefira HTTPS em \textit{Web Page/Image objects} e ajuste políticas de web view (JS, pop-ups) ao publicar. \footnote{Seção de segurança em ``Create a Dashboard''. \url{https://help.tableau.com/current/pro/desktop/en-us/dashboards_create.htm}. :contentReference[oaicite:6]{index=6}}
\end{itemize}

\begin{NoteBox}
\textbf{Dica de AED}: o dashboard \textit{não} substitui a exploração em \textit{worksheets}. Primeiro \textit{descubra} (iterando em sheets), depois \textit{explique} (curadoria no dashboard). Evite painéis ``mural de gráficos'' sem hierarquia visual.
\end{NoteBox}

\subsection{Stories}
\textbf{Story} é uma \textit{sequência de pontos} (story points), cada um contendo uma sheet, dashboard ou texto, para conduzir a audiência por uma narrativa (mudança no tempo, contraste, \textit{drill down}, etc.). Stories são \textit{sheets} especiais: criam-se com \textit{New Story}, definem-se dimensões fixas e adicionam-se pontos com legendas; pode-se duplicar/atualizar pontos mantendo ligação às vis originais.\footnote{``Stories'' (conceito) e ``Create a Story'' (passo a passo). \url{https://help.tableau.com/current/pro/desktop/en-us/stories.htm} e \url{https://help.tableau.com/current/pro/desktop/en-us/story_create.htm}. :contentReference[oaicite:7]{index=7}}
\begin{itemize}[leftmargin=*]
  \item \textbf{Propósito e padrões narrativos}: esboce a \textit{jornada} e escolha um tipo (mudança no tempo, contraste, fatores, outliers, etc.); mantenha simples e planeje \textit{load times}. \footnote{``Best Practices for Telling Great Stories''. \url{https://help.tableau.com/current/pro/desktop/en-us/story_best_practices.htm}. (Seção referenciada a partir de Stories e exemplos). :contentReference[oaicite:8]{index=8}}
  \item \textbf{Ajuste de tamanho}: use \textit{Fit to Story} para que dashboards encaixem exatamente na dimensão do \textit{story}. \footnote{Ver ``Create a Story'' — Fit a dashboard to a story. \url{https://help.tableau.com/current/pro/desktop/en-us/story_create.htm}. :contentReference[oaicite:9]{index=9}}
  \item \textbf{Exemplo guiado}: o tutorial de \textit{Get Started} inclui a etapa ``Build a story to present'' com boas práticas de foco e legendas. \footnote{Tutorial ``Step 7: Build a story to present''. \url{https://help.tableau.com/current/guides/get-started-tutorial/en-us/get-started-tutorial-story.htm}. :contentReference[oaicite:10]{index=10}}
\end{itemize}

\subsection{Quando usar cada um (mapa mental rápido)}
\begin{itemize}[leftmargin=*]
  \item \textbf{Worksheet}: exploração tática (\textit{hypothesis testing}), criação de cálculos, comparação de alternativas de gráfico.
  \item \textbf{Dashboard}: síntese operacional/gerencial com 2–3 \textit{vistas-chave}, interações e objetos de suporte (legendas, filtros, navegação).
  \item \textbf{Story}: argumento persuasivo ou relatório sequencial (ex.: antes/depois, tendência, contraste por segmentos) com \textbf{pontos} e \textbf{captions}.
\end{itemize}

\subsection{Performance e experiência}
Antes do design final, \textit{conheça seus dados}, teste filtros (prefira \textit{Keep Only} a \textit{Exclude} quando possível), e considere extratos para acelerar. O guia de performance resume práticas para workbooks rápidos.\footnote{``Optimize Workbook Performance'' e referências de \textit{Designing Efficient Workbooks}. \url{https://help.tableau.com/current/pro/desktop/en-us/performance_tips.htm}. :contentReference[oaicite:11]{index=11}}

\begin{SolvedBox}
\textbf{Exercício resolvido (prova)}\\[-2pt]
\emph{Cenário}: você tem 6 worksheets (KPIs, Mapa, Série temporal, Ranking, Detalhe por produto, Tabela de metas). Monte um \textbf{dashboard} e uma \textbf{story} de 4 pontos.\\
\textbf{Solução (resumo)}:
\begin{enumerate}[leftmargin=*]
  \item \textbf{Dashboard}: tamanho fixo 1300$\times$700; layout em \emph{tiled}; vistas: KPIs (canto sup.\,esq.), Mapa (ocupa largura superior), Série temporal (inferior). Adicione filtros globais (Região, Período) e \emph{Use as Filter} no mapa. 
  \item \textbf{Story} (4 pontos): \emph{Visão geral} (dashboard); \emph{Queda no Sul} (filtro aplicado + legenda); \emph{Produtos críticos} (ranking/pareto); \emph{Plano de ação} (tabela de metas). Use \emph{Fit to Story} no dashboard e legendas curtas (1 frase/insight).
\end{enumerate}
\end{SolvedBox}

\begin{figure}[H]
\centering
\fbox{\rule{0pt}{110pt}\rule{0.92\linewidth}{0pt}}
\caption{[Espaço reservado] Exemplos de layout (dashboard) e sequência (story) para inserir imagens do Tableau.}
\end{figure}





\section{Campos Calculados e Ordem de Operações}

Nesta seção, aprofundamos o conceito de **campos calculados** no Tableau, explorando desde fórmulas simples até cálculos avançados (Tabela, LOD, janelas). Também explicamos a **ordem em que o Tableau executa filtros e cálculos**, o que é fundamental para evitar resultados incorretos.

\subsection{Camadas de Cálculo no Tableau}

Os campos calculados no Tableau podem ser aplicados em diferentes camadas — cada camada tem escopo e implicações distintas:

\begin{enumerate}[leftmargin=*]
  \item \textbf{Cálculos de campo simples (row-level)} — operam em cada linha da base (antes de agregações).
  \item \textbf{Cálculos agregados / de medida} — aplicam funções agregadas (\texttt{SUM}, \texttt{AVG}, etc.).
  \item \textbf{Cálculos de Tabela (Table Calculations)} — operam sobre a visualização resultante, com escopo definido por \textit{addressing/partitioning}.
  \item \textbf{Expressões de Nível de Detalhe (LOD)} — calculam agregações em granularidade distinta, independentemente da exibição (\texttt{FIXED}, \texttt{INCLUDE}, \texttt{EXCLUDE}).
\end{enumerate}

\subsection{Sintaxe e exemplos básicos}

Alguns operadores e funções comuns em campos calculados:

\begin{itemize}[leftmargin=*]
  \item \texttt{IF / THEN / ELSE} — lógica condicional  
  \item \texttt{CASE} — lógica condicional com muitos casos  
  \item \texttt{DATEPART(date, “month”)} — extrair partes da data  
  \item \texttt{ZN(expression)} — trata valores nulos (Null) como zero  
  \item \texttt{WINDOW\_SUM(expr, start, end)} — soma de valores na janela sobre a visualização  
\end{itemize}

\subsection{Cálculos de Tabela (Table Calculations)}

Cálculos de tabela são avaliados \textit{após} a agregação dos dados e atuam sobre as marcas exibidas. Dois parâmetros-chave definem o escopo:

\begin{itemize}[leftmargin=*]
  \item \textbf{Addressing} — o campo que “anda” pela dimensão de agregação (ex: soma cumulativa sobre meses).  
  \item \textbf{Partitioning} — como as marcas são agrupadas para fins de cálculo (ex: por região, por categoria).  
\end{itemize}

Exemplo:  
\[
\texttt{RUNNING\_SUM(SUM([Vendas]))}
\]  
é um cálculo de tabela que soma cumulativamente ao longo do eixo definido.

\subsection{Expressões de Nível de Detalhe (LOD)}

As expressões LOD permitem agregações independentes da visualização atual. Existem três modos:

\begin{itemize}[leftmargin=*]
  \item \texttt{FIXED} — fixa uma granularidade (ex: \(\{ \texttt{FIXED Região : SUM(Vendas)}\}\)).  
  \item \texttt{INCLUDE} — inclui uma dimensão extra no cálculo, mesmo que não esteja na visualização.  
  \item \texttt{EXCLUDE} — exclui uma dimensão da agregação, ainda que apareça no gráfico.
\end{itemize}

Exemplos mais avançados:

\begin{itemize}[leftmargin=*]
  \item \(\{ \texttt{FIXED Cliente : SUM(Vendas)}\}\) — soma de vendas por cliente, independente de filtros de categoria.  
  \item \(\{ \texttt{INCLUDE Categoria : SUM(Vendas)}\} / \{ \texttt{FIXED Região : SUM(Vendas)}\}\) — proporção da categoria dentro da região, mesmo se categoria não estiver no gráfico.  
  \item \(\{ \texttt{EXCLUDE Ano : SUM(Vendas)}\}\) — calcula vendas agregadas ignorando o nível “Ano”.
\end{itemize}

\subsection{Ordem de Operações (Order of Operations)}

O Tableau processa filtros e cálculos em uma sequência fixa. Entender essa ordem evita surpresas e incoerências nos resultados:

\begin{center}
\begin{tabular}{ll}
1. & \textbf{Extrato / Filtros de Fonte (extract / datasource filters)} \\
2. & \textbf{Filtros de Contexto} \\
3. & \textbf{LOD FIXED} \\
4. & \textbf{Dimensões / Medidas padrões} \\
5. & \textbf{LOD INCLUDE / EXCLUDE} \\
6. & \textbf{Cálculos de Tabela} \\
7. & \textbf{Ordenação, Top N, Ranks, Exceções}
\end{tabular}
\end{center}

Erros comuns:
\begin{itemize}[leftmargin=*]
  \item Usar \texttt{INCLUDE} onde \texttt{FIXED} deveria ser usado — pode gerar “duplicações invisíveis”.  
  \item Aplicar filtros depois de LOD FIXED — eles não afetam agregações FIXED.  
  \item Misturar cálculos de tabela com LOD sem cuidado no \textit{addressing/partitioning}.
\end{itemize}

\begin{SolvedBox}
\textbf{Exercício resolvido (nível intermediário)}\\[-3pt]
Em uma base com \texttt{Cliente}, \texttt{Categoria}, \texttt{Ano}, \texttt{Vendas}, crie um KPI “\%VendasClienteRegião” estável:

\begin{enumerate}[leftmargin=*]
  \item \texttt{SumVendasCliente} = \(\{ \texttt{FIXED Cliente : SUM(Vendas)}\}\)  
  \item \texttt{SumVendasRegião} = \(\{ \texttt{FIXED Região : SUM(Vendas)}\}\)  
  \item \texttt{\%VendasClienteRegião} = \texttt{SumVendasCliente / SumVendasRegião}  
\end{enumerate}

Mesmo se você filtrar por ano ou por categoria, essa porcentagem permanece coerente em comparação regional.
\end{SolvedBox}


\section{Performance: Princípios Essenciais e Boas Práticas}

A performance em Tableau refere-se à \textbf{velocidade e fluidez da análise}, desde a resposta das consultas até o carregamento de dashboards. Um workbook eficiente economiza tempo, reduz carga no servidor e melhora a experiência analítica.  
Os princípios abaixo resumem práticas extraídas da documentação oficial\footnote{Baseado em: \textit{Optimize Workbook Performance}, \url{https://help.tableau.com/current/pro/desktop/en-us/performance_tips.htm}.} e no whitepaper \textit{Designing Efficient Workbooks}.

---

\subsection{1. Conheça seus dados (nível de banco)}

Antes mesmo de abrir o Tableau, o desempenho começa na base de dados.  
Converse com o time de banco e avalie:

\begin{itemize}[leftmargin=*]
  \item \textbf{Integridade referencial}: habilite \textit{Assume Referential Integrity} em bancos que suportam — isso ativa o \textit{join culling}, que consulta apenas as tabelas necessárias.  
  \item \textbf{Índices}: crie índices para colunas usadas frequentemente em filtros ou relacionamentos.  
  \item \textbf{Permissões para tabelas temporárias}: o Tableau cria tabelas temporárias para agilizar consultas — certifique-se de que o usuário tem permissão para criá-las.  
  \item \textbf{Particionamento e redução de dados}: divida grandes tabelas em partições (por período ou região) para reduzir varreduras desnecessárias.  
  \item \textbf{Use bancos servidores}: arquivos como Excel e CSV podem ser lentos em consultas diretas — use banco relacional ou crie um \textit{extract}.  
\end{itemize}

---

\subsection{2. Trabalhe com \textit{Extracts} e não \textit{Live} quando possível}

Os \textbf{Extracts} (\texttt{.hyper}) são snapshots comprimidos otimizados para leitura analítica e consultas repetidas.  
Segundo o manual oficial\footnote{\textit{Test Your Data and Use Extracts}, Tableau Help. \url{https://help.tableau.com/current/pro/desktop/en-us/performance_extracts.htm}}, extratos:
\begin{itemize}[leftmargin=*]
  \item Reduzem o tempo de renderização e carga da rede;  
  \item Suportam cálculos pré-materializados (\textit{Compute Calculations Now});  
  \item Podem usar filtros e amostragens para diminuir volume;  
  \item Permitem agregação de dados nas dimensões visíveis para reduzir granularidade.  
\end{itemize}

Evite \textit{Custom SQL} em conexões ao vivo, pois o Tableau não pode otimizá-las.  
Quando precisar, crie um \textit{extract} para executar a consulta apenas uma vez.

---

\subsection{3. Simplifique cálculos e lógica}

Cálculos complexos e aninhados aumentam o custo de compilação e execução da query.  
Boas práticas segundo o \textit{Workbook Optimizer}\footnote{\textit{Create Efficient Calculations}, Tableau Help. \url{https://help.tableau.com/current/pro/desktop/en-us/performance_calculations.htm}}:
\begin{itemize}[leftmargin=*]
  \item Prefira \texttt{CASE} em vez de \texttt{IF/ELSE} para grandes condicionais;  
  \item Evite cálculos linha a linha em campos com milhões de registros — mova para SQL ou Python antes de importar;  
  \item Converta datas corretamente: \texttt{DATEPARSE} e \texttt{DATEDIFF} são mais rápidos que combinações \texttt{STRING → DATE};  
  \item Agregue medidas antes de trazer para o Tableau (\texttt{SUM}, \texttt{AVG}) sempre que possível;  
  \item Use LOD (\texttt{FIXED, INCLUDE, EXCLUDE}) com parcimônia — são poderosos, mas custosos.  
\end{itemize}

\begin{FormulaBox}
\textbf{Dica prática:}  
Evite cálculos aninhados como  
\texttt{IF [Lucro] > AVG([Lucro]) THEN ... ELSEIF [Lucro] > MEDIAN([Lucro]) ...}  
— isso força múltiplas varreduras.  
Crie agregações em camadas no SQL ou no \textit{extract}.
\end{FormulaBox}

---

\subsection{4. Design eficiente de \textit{views} e dashboards}

A maior causa de lentidão vem do design — \textbf{muitos gráficos, muitas marcas, muitos filtros}.  
Recomendações\footnote{\textit{Design for Performance While You Build a View} e \textit{Make Visualizations Faster}, Tableau Help.}:
\begin{itemize}[leftmargin=*]
  \item \textbf{Reduza a cardinalidade}: evite colocar dimensões com milhares de valores em \textit{detail} — cada valor vira uma marca.  
  \item \textbf{Menos é mais}: limite dashboards a 2–3 \textit{views} principais.  
  \item \textbf{Use filtros inteligentes}: prefira filtros de ação (\textit{Use as Filter}) a \textit{Show Relevant Values}, que reconsultam a base a cada interação.  
  \item \textbf{Evite \textit{blends}}: quando possível, use \textit{relationships} ou \textit{joins}. Blends fazem junções em memória e degradam performance.  
  \item \textbf{Fixe o tamanho do dashboard}: tamanhos automáticos impedem cache e forçam re-renderização.  
  \item \textbf{Agrupe visualmente}: evite excesso de containers e objetos flutuantes — cada elemento adicional gera cálculos de layout.  
\end{itemize}

\begin{NoteBox}
\textbf{Regra de ouro:} Cada marca = uma linha processada.  
Um mapa com 100 mil pontos é uma query com 100 mil linhas renderizadas — mesmo que apenas 10 sejam legíveis visualmente.
\end{NoteBox}

---

\subsection{5. Controle de atualizações e consultas}

\begin{itemize}[leftmargin=*]
  \item \textbf{Desative \textit{Automatic Updates}} (\texttt{F10}) durante construção de views grandes.  
  \item \textbf{Use o \textit{Performance Recorder}} (\textit{Help → Settings and Performance → Start Recording}) para medir tempo de consulta, renderização e cálculos.  
  \item \textbf{Analise o painel \textit{Performance Summary}} para identificar gargalos:  
    \begin{itemize}
      \item \textit{Executing Query} — lentidão no banco;  
      \item \textit{Compiling Query} — cálculos excessivos;  
      \item \textit{Layout Computations} — dashboards complexos;  
      \item \textit{Blending Data} — múltiplas fontes.  
    \end{itemize}
\end{itemize}

\begin{FormulaBox}
\textbf{Fluxo de medição de performance:}\\
1. Inicie o \textit{Performance Recorder};\\
2. Execute suas interações normais (filtros, drill-downs, navegação);\\
3. Pare a gravação e analise os eventos;\\
4. Simplifique cálculos e reduza \textit{marks} ou fontes duplicadas.
\end{FormulaBox}

---

\subsection{6. Estruture o workbook e fontes de forma enxuta}

\begin{itemize}[leftmargin=*]
  \item Feche fontes de dados não utilizadas (\texttt{Right Click → Close}).  
  \item Oculte campos não usados antes de criar \textit{extracts} (\texttt{Hide All Unused Fields}).  
  \item Divida workbooks muito grandes em arquivos menores focados em propósitos específicos.  
  \item Prefira relacionamentos bem definidos entre fontes — múltiplas conexões degradam o cache local.  
  \item Remova cálculos ou filtros redundantes.  
\end{itemize}

\begin{NoteBox}
\textbf{Regra prática:}  
Se você sente lentidão no Tableau Desktop, o problema persistirá (ou piorará) no Tableau Server.  
Otimize localmente antes de publicar.
\end{NoteBox}

---

\subsection{7. Checklist de Performance (resumo prático)}

\begin{longtable}{p{4cm}p{10cm}}
\toprule
\textbf{Categoria} & \textbf{Recomendação prática} \\ \midrule
\textbf{Dados} & Use índices, particione grandes tabelas e teste o desempenho direto no banco. \\
\textbf{Conexão} & Prefira \textbf{Extracts} (.hyper) com campos limitados e cálculos materializados. \\
\textbf{Cálculos} & Simplifique expressões; mova cálculos repetitivos para SQL/Python. \\
\textbf{Filtros} & Use filtros de ação e contextuais; evite \textit{Only Relevant Values}. \\
\textbf{Visualização} & Limite o número de views e de marcas; use dashboards fixos. \\
\textbf{Design} & Minimize containers, imagens e objetos flutuantes. \\
\textbf{Monitoramento} & Use o \textit{Performance Recorder} e otimize gargalos de query ou layout. \\
\bottomrule
\end{longtable}

---

\begin{SolvedBox}
\textbf{Exercício resolvido (avançado):}\\[-3pt]
Dado um dashboard com:
\begin{itemize}
  \item 7 views conectadas a 4 fontes distintas (CSV + SQL + API + Excel);
  \item 12 filtros, sendo 4 com \textit{Only Relevant Values};
  \item 5 cálculos LOD (\texttt{FIXED, INCLUDE, EXCLUDE});
\end{itemize}

\textbf{Otimização aplicada:}
\begin{enumerate}
  \item Combinar CSV e Excel no Tableau Prep → gerar um único \textit{extract} .hyper.  
  \item Reduzir filtros para 4 e substituir por \textit{action filters}.  
  \item Reescrever cálculos LOD como agregações fixas no SQL.  
  \item Fixar tamanho do dashboard (1300×700).  
  \item Executar o \textit{Performance Recorder} e medir redução no tempo médio de carregamento.  
\end{enumerate}

\textbf{Resultado:} tempo de renderização caiu de 11,2 s para 3,9 s (−65 \%).  
\end{SolvedBox}

\begin{figure}[H]
\centering
\fbox{\rule{0pt}{120pt}\rule{0.92\linewidth}{0pt}}
\caption{[Espaço reservado] Diagrama de fluxo de otimização: fonte → cálculos → visualização → gravação de performance.}
\end{figure}

\subsection{Python no Tableau \textit{Public}: limitações e alternativas práticas}
\textbf{Resumo}: o \textbf{Tableau Public} \emph{não} executa extensões analíticas (R/TabPy). Logo, campos \texttt{SCRIPT_…} não funcionam. Para ter Python “dinâmico”, é preciso Tableau Desktop (Creator) + Server/Cloud. No \textit{Public}, o caminho é \textbf{pré-processar fora} (Python) e publicar os \emph{resultados prontos} — ou usar uma \textbf{ponte via Google Sheets} (atualização limitada). Abaixo vai um passo-a-passo detalhado.

\subsubsection*{A. Pré-processar no Python (ex.: \textit{Titanic})}
\paragraph{O que pré-processar?}
\begin{itemize}[leftmargin=*]
\item \textbf{Limpeza e enriquecimento}: imputar faltantes, normalizar categorias, criar \emph{features}.
\item \textbf{Agregações/indicadores}: métricas de negócio (ex.: taxa de sobrevivência por classe/sexo), escores de modelos (probabilidades), \emph{rankings}, \emph{clusters}.
\item \textbf{Narrativas e \emph{insights}}: textos gerados (ex.: “Mulheres na 1ª classe tiveram 95% de sobrevivência”), importância de variáveis, \emph{explanations} (SHAP) \emph{flattened}.
\end{itemize}

\paragraph{Exemplo simples com \textit{Titanic}}
Dados: \texttt{titanic.csv} (Kaggle). Objetivo: publicar no \textit{Public} um conjunto enriquecido + previsões.

\textbf{Passos de pré-processamento} (roteiro típico):
\begin{enumerate}[leftmargin=*]
\item \textbf{Carregar e limpar}: remover \texttt{Cabin} (muitos NAs), imputar \texttt{Age} pela mediana por \texttt{Sex+Pclass}; preencher \texttt{Embarked} com modo.
\item \textbf{Criar \emph{features}}:
\begin{itemize}
\item \texttt{FamilySize = SibSp + Parch + 1}
\item \texttt{IsAlone = 1{FamilySize = 1}}
\item Extrair \texttt{Title} de \texttt{Name} (\texttt{Mr/Mrs/Miss/Master/Other})
\end{itemize}
\item \textbf{Codificar categorias}: \texttt{Sex}, \texttt{Embarked}, \texttt{Title} (one-hot ou ordinal coerente).
\item \textbf{Treinar um modelo} (ex.: \texttt{LogisticRegression} ou \texttt{XGBoost}) para \texttt{Survived}; salvar:
\begin{itemize}
\item \texttt{Survival_Proba} (% de sobrevivência prevista) e \texttt{Survival_Pred} (0/1 com limiar 0{,}5).
\item \texttt{FeatureImportance} (ou SHAP \emph{mean abs} por variável).
\end{itemize}
\item \textbf{Gerar \emph{insights} textuais} (opcional): frases por segmento (ex.: “\texttt{Pclass=3, male}: 13% real, 12% previsto”).
\item \textbf{Exportar} um \texttt{CSV} (ou \texttt{.hyper}) com:
\begin{itemize}
\item colunas originais úteis + \texttt{FamilySize}, \texttt{IsAlone}, \texttt{Title}, \texttt{Survival_Proba}, \texttt{Survival_Pred}, métricas agregadas por segmento (se quiser uma \emph{tabela wide} de indicadores), e um campo \texttt{InsightText}.
\end{itemize}
\end{enumerate}

\textbf{Publicação no \textit{Public}}:
\begin{enumerate}[leftmargin=*]
\item Abra o \textbf{Tableau Public Desktop}, conecte em \texttt{Text File} (\texttt{CSV}) ou \texttt{.hyper}.
\item Monte as \textbf{views} usando as colunas calculadas (\texttt{Survival_Proba}, \texttt{Title}, etc.).
\item Publique: \texttt{Server → Tableau Public → Save to Tableau Public}.
\end{enumerate}

\emph{Vantagem}: tudo roda rápido (sem TabPy); você controla a qualidade/versão dos dados.
\emph{Limite}: não é “tempo real”; precisa reprocessar e reenviar quando houver mudanças.

\subsubsection*{B. Ponte via Google Sheets (atualização leve)}
\textbf{Ideia}: seu script Python escreve numa planilha do Google; o \textit{Public} se conecta a ela. O \textit{Public} costuma \emph{atualizar cerca de 1x/dia} (ou sob ação manual do usuário/autor).

\textbf{Passo a passo}:
\begin{enumerate}[leftmargin=*]
\item \textbf{No Google}: crie uma planilha com abas (ex.: \texttt{base_titanic}, \texttt{dim_segmentos}, \texttt{insights}).
\item \textbf{No Python}: use uma lib como \texttt{gspread} ou \texttt{pygsheets} para \emph{overwrite} as abas com os dados pré-processados (mesmos campos do item A).
\item \textbf{No Tableau Public}: \texttt{Connect → Google Sheets}, selecione a planilha/abas e construa os \emph{dashboards}.
\item \textbf{Rotina de atualização}: agende seu script (cron/Task Scheduler/GitHub Actions) para atualizar a planilha (ex.: 1x por dia). O \textit{Public} refletirá isso no próximo refresh.
\end{enumerate}

\emph{Dicas}:
\begin{itemize}[leftmargin=*]
\item Mantenha \textbf{esquema estável} (nomes de colunas/tipos) para evitar quebras.
\item Separe abas “\texttt{facts}” e “\texttt{dims}”; no Tableau, relacione/junte conforme necessário.
\item Se precisar de \textbf{controle manual}, inclua uma aba de \texttt{Parametros} (ver abaixo).
\end{itemize}

\subsubsection*{C. Parâmetros e Ações (para simulações e navegação)}
Mesmo sem Python ao vivo, você mantém interatividade rica no \textit{Public}:

\paragraph{Parâmetros (what-if)}
\begin{itemize}[leftmargin=*]
\item \textbf{Ex.: Limiar de classificação} do Titanic: crie um \texttt{Parâmetro} \texttt{p_threshold} (0–1, step 0{,}01).
\item \textbf{Cálculo} \texttt{Pred_Dynamic}: \texttt{INT([Survival_Proba] >= [p_threshold])}.
\item Use \texttt{Pred_Dynamic} para colorir/filtrar; mostre KPIs “sensíveis” ao parâmetro.
\end{itemize}

\paragraph{Ações de \textit{Dashboard}}
\begin{itemize}[leftmargin=*]
\item \textbf{Filtro por seleção}: clique em um segmento (ex.: \texttt{Title = Miss}) para filtrar demais \emph{views} (Action → Filter).
\item \textbf{Highlight}: destaque itens relacionados sem filtrar (Action → Highlight).
\item \textbf{URL}: abra uma fonte externa (ex.: dicionário de títulos “Mr/Mrs/Miss”) em nova guia.
\end{itemize}

\paragraph{Parâmetro → Título dinâmico/explicação}
\begin{itemize}[leftmargin=*]
\item Campo calc. \texttt{Titulo_Dinamico}: \texttt{"Cenário: Limiar = " + STR([p_threshold])}.
\item Campo \texttt{Texto_Insight}: concatene \texttt{InsightText} (gerado no Python) filtrado pelo contexto atual.
\end{itemize}

\subsubsection*{D. Criando \emph{insights} com ajuda do Python (e \emph{agents})}
Sem TabPy, gere os \emph{insights} \emph{fora} e traga como \textbf{colunas/camadas}:

\paragraph{Tipos de \emph{insights} úteis de pré-cálculo}
\begin{itemize}[leftmargin=*]
\item \textbf{Segmentos chave}: \texttt{GroupBy} (ex.: \texttt{Pclass}×\texttt{Sex}×\texttt{Title}) com \texttt{count}, \texttt{survival_rate}, \texttt{lift} vs. média.
\item \textbf{Importância de variáveis}: \texttt{coef_abs} (logística) ou \texttt{gain} (XGBoost), normalizados em 0–100.
\item \textbf{Explicabilidade}: SHAP médio por variável/segmento (agregado) → tabela “wide” para \emph{bars}.
\item \textbf{Narrativas} (\texttt{InsightText}): frases curtas com \emph{templates}, ex.: \
“\emph{Mulheres na 1ª classe: 96\% sobreviveram (lift +2{,}1×). Importância: \texttt{Sex}, \texttt{Pclass}.}”
\end{itemize}

\paragraph{Com \emph{agents} (LLMs) no seu pipeline}
\begin{itemize}[leftmargin=*]
\item \textbf{Geração de texto/explicações}: depois de calcular métricas, passe \emph{resumos estruturados} a um agente (LLM) para transformar em bullets curtos (controle de tamanho/termos).
\item \textbf{Curadoria} (\emph{guardrails}): imponha limites (sem PII), \emph{templates} fixos, e \emph{whitelist} de termos.
\item \textbf{Output}: grave o texto final em uma coluna \texttt{InsightText} por segmento/aba, consumida pelo Tableau como \emph{tooltip} ou \emph{card}.
\end{itemize}

\paragraph{Como conectar esses \emph{insights} no \textit{Public}}
\begin{enumerate}[leftmargin=*]
\item \textbf{Inclua} \texttt{InsightText} como campo no CSV/Google Sheets.
\item No Tableau, crie um \textbf{Worksheet de texto} (ou \emph{tooltip viz}) que mostre \texttt{InsightText} conforme a seleção (Ação de Filtro).
\item Opcional: \textbf{parâmetro de idioma/nível de detalhe} para trocar entre “Executivo”, “Analista”, etc. (seu pipeline gera múltiplas versões).
\end{enumerate}

\subsubsection*{E. Boas práticas para o \textit{Public}}
\begin{itemize}[leftmargin=*]
\item \textbf{Performance}: pré-agregue o que puder, limite # de \emph{views}/marcas, use \textbf{camadas de detalhe sob demanda} (Ações).
\item \textbf{Estabilidade}: \emph{schema} consistente entre atualizações; evite mudar nomes de colunas.
\item \textbf{Transparência}: exiba “\emph{Dados atualizados em:} \texttt{YYYY-MM-DD HH:MM}” (campo gerado pelo Python).
\end{itemize}

\subsubsection*{F. Checklist rápido (Titanic \texorpdfstring{$\rightarrow$}{->} Public)}
\begin{enumerate}[leftmargin=*]
\item Python: limpar/engenhar \emph{features} + treinar modelo + gerar \texttt{Survival_Proba}, \texttt{Pred}, \texttt{FeatureImportance}, \texttt{InsightText}.
\item Exportar: \texttt{CSV} (ou escrever no \textbf{Google Sheets}).
\item Tableau Public: conectar, montar \emph{views}, criar \textbf{parâmetro de limiar}, ações de filtro, painel de \textbf{insights}.
\item Publicar (+, se Google Sheets, agendar \textbf{atualização diária} do script).
\end{enumerate}

\noindent\emph{Resultado}: você obtém dashboards públicos com “toque de ciência de dados” (previsões, explicações e narrativas) — \textbf{sem} depender de TabPy — e com atualizações controladas pelo seu pipeline Python.


\section{Exercício Resolvido — Conexão, Modelagem e Visualização }
\begin{SolvedBox}
\textbf{Problema:} Conecte um CSV de \textit{Vendas} (linhas: Data, Região, Categoria, Vendas, Custo), crie o campo \texttt{Lucro = Vendas - Custo}, publique um gráfico de barras com \%Lucro por Região e otimize a performance.

\textbf{Solução (passo a passo):}
\begin{enumerate}[leftmargin=*]
  \item \textbf{Conexão}: \textit{Data Source} $\rightarrow$ CSV (Extrato .hyper).
  \item \textbf{Campo Calculado}: \texttt{Lucro = [Vendas] - [Custo]}; \texttt{\%Lucro = [Lucro]/[Vendas]}.
  \item \textbf{Visual}: Dimensão = Região; Medida = AVG(\%Lucro); \textit{Label} de \%.
  \item \textbf{Performance}: filtro de contexto em Ano; extract atualizado; remover colunas não usadas.
  \item \textbf{Publicação}: Tableau Public (nomeando com título e descrição).
\end{enumerate}
\textbf{Comentário}: \%Lucro médio por Região destaca discrepâncias; LOD FIXED por Região pode ser usado para estabilizar o denominador.
\end{SolvedBox}



\begin{SolvedBox}
\textbf{Exercício 1 — Conexão e Modelagem Inicial}\\[4pt]
\textbf{Objetivo:} Conectar e modelar corretamente uma base de vendas.\\[4pt]
\textbf{Passos:}
\begin{enumerate}[leftmargin=*]
  \item Baixe o arquivo \texttt{vendas\_loja.csv} com campos: \texttt{Data}, \texttt{Região}, \texttt{Categoria}, \texttt{Vendas}, \texttt{Custo}.
  \item No Tableau Public, clique em \texttt{Conectar → Arquivo de Texto} e selecione o CSV.
  \item Na aba \textbf{Data Source}, renomeie os campos com convenção clara (\texttt{snake\_case}) e verifique tipos:  
  \texttt{Data} (Date), \texttt{Região} (String), \texttt{Vendas/Custo} (Number).
  \item Crie o campo calculado \texttt{Lucro = [Vendas] - [Custo]}.
  \item No painel de dados, verifique se \texttt{Lucro} aparece em “Medidas”.
\end{enumerate}
\textbf{Resultado:} A base está limpa e modelada com medidas e dimensões prontas.\\
\textbf{Insight esperado:} “A estruturação correta da fonte é o primeiro passo para visualizações coerentes.”
\end{SolvedBox}


\begin{SolvedBox}
\textbf{Exercício 2 — Join e KPI Estável com LOD}\\[4pt]
\textbf{Objetivo:} Combinar duas tabelas e garantir KPIs consistentes.\\[4pt]
\textbf{Cenário:} Tabelas \texttt{Orders} (pedidos) e \texttt{Returns} (devoluções).\\[4pt]
\textbf{Passos:}
\begin{enumerate}[leftmargin=*]
  \item Na aba \textbf{Data Source}, arraste \texttt{Returns} sobre \texttt{Orders} e escolha \textbf{Left Join}.
  \item Defina a chave de junção: \texttt{OrderID = OrderID}.
  \item Crie o campo calculado \texttt{Returned? = IF ISNULL([Return Reason]) THEN "No" ELSE "Yes" END}.
  \item Crie o KPI fixo por região:  
  \(\{\texttt{FIXED [Região] : SUM([Vendas])}\}\)
  \item Coloque \texttt{Região} em Colunas e \texttt{SUM([Vendas])} em Linhas.  
        Adicione \texttt{Returned?} em Cor.
\end{enumerate}
\textbf{Resultado:} O LOD garante estabilidade do KPI mesmo ao aplicar filtros.\\
\textbf{Insight esperado:} “O LOD FIXED preserva a coerência dos KPIs independentemente do contexto de filtro.”
\end{SolvedBox}


\begin{SolvedBox}
\textbf{Exercício 3 — Dimensões, Hierarquias e Drill-Down}\\[4pt]
\textbf{Objetivo:} Criar e explorar hierarquias temporais e geográficas.\\[4pt]
\textbf{Passos:}
\begin{enumerate}[leftmargin=*]
  \item Crie uma hierarquia temporal: arraste \texttt{Ano} sobre \texttt{Mês} e nomeie “Tempo”.
  \item Crie uma hierarquia espacial: arraste \texttt{País} sobre \texttt{Estado} e depois sobre \texttt{Cidade}.
  \item Construa um gráfico de linhas: Eixo X = \texttt{Ano}, Eixo Y = \texttt{SUM(Vendas)}.
  \item Clique no símbolo \texttt{+} ao lado de “Ano” para expandir até “Mês”.
  \item Use cor para representar \texttt{Região}.
\end{enumerate}
\textbf{Resultado:} Visualização hierárquica temporal e espacial pronta para exploração.\\
\textbf{Insight esperado:} “Hierarquias facilitam o drill-down analítico sem reescrever a visualização.”
\end{SolvedBox}


\begin{SolvedBox}
\textbf{Exercício 4 — Campos Calculados e Ordem de Operações}\\[4pt]
\textbf{Objetivo:} Demonstrar a diferença entre cálculo de tabela e LOD.\\[4pt]
\textbf{Passos:}
\begin{enumerate}[leftmargin=*]
  \item Use a base \texttt{vendas\_loja.csv}.
  \item Crie dois campos:
  \begin{itemize}
    \item \texttt{Lucro Médio por Região = \{FIXED [Região] : AVG([Lucro])\}}
    \item \texttt{Lucro Cumulativo = RUNNING_SUM(SUM([Lucro]))}
  \end{itemize}
  \item Coloque \texttt{Região} e \texttt{Data} em Colunas, \texttt{Lucro} em Linhas.
  \item Compare os dois cálculos aplicando filtros por “Categoria”.
\end{enumerate}
\textbf{Resultado:} O campo LOD permanece fixo; o cálculo de tabela varia conforme a view.\\
\textbf{Insight esperado:} “Cálculos de tabela dependem da visualização; LODs dependem da definição analítica.”
\end{SolvedBox}


\begin{SolvedBox}
\textbf{Exercício 5 — Integração Python + Tableau Public (Titanic)}\\[4pt]
\textbf{Objetivo:} Demonstrar o uso de Python para pré-processamento de dados antes da publicação no Tableau Public.\\[4pt]
\textbf{Cenário:} Usar a base \texttt{titanic.csv} e gerar previsões de sobrevivência.\\[4pt]
\textbf{Passos no Python:}
\begin{enumerate}[leftmargin=*]
  \item Carregue o CSV e limpe dados nulos:  
  \texttt{df['Age'] = df.groupby('Sex')['Age'].transform(lambda x: x.fillna(x.median()))}
  \item Crie \texttt{FamilySize = SibSp + Parch + 1}.
  \item Treine uma regressão logística para prever \texttt{Survived}.
  \item Salve as probabilidades em \texttt{Survival_Proba}.
  \item Exporte o resultado para \texttt{titanic\_insights.csv}.
\end{enumerate}
\textbf{No Tableau Public:}
\begin{enumerate}[leftmargin=*]
  \item Conecte o CSV e crie um parâmetro \texttt{Limiar} (0–1).
  \item Campo calculado: \texttt{Pred_Dynamic = INT([Survival_Proba] >= [Limiar])}.
  \item Crie um mapa de dispersão com \texttt{Age} e \texttt{Fare}, colorindo por \texttt{Pred_Dynamic}.
\end{enumerate}
\textbf{Resultado:} Dashboard com simulação interativa sem uso direto de TabPy.\\
\textbf{Insight esperado:} “Mesmo sem integração direta, Python pode enriquecer análises publicadas via pré-processamento inteligente.”
\end{SolvedBox}




\begin{appendices}

\chapter*{Apêndice do Capítulo 4 — Desafios Analíticos em Tableau}
\addcontentsline{toc}{chapter}{Apêndice do Capítulo 4 — Desafios Analíticos em Tableau}

Este apêndice apresenta uma coletânea de \textbf{perguntas orientadoras e desafios práticos} para aplicação no \textbf{Tableau}, utilizando as quatro bases de dados criadas na arquitetura de finanças:
\begin{itemize}
  \item \texttt{pessoas\_salarios.csv}
  \item \texttt{contas.csv}
  \item \texttt{cartoes.csv}
  \item \texttt{transacoes.csv}
\end{itemize}

Essas perguntas podem ser exploradas como \textbf{painéis interativos, dashboards executivos} ou \textbf{storytelling analítico}, permitindo ao aluno aplicar as técnicas de \textit{Análise Exploratória de Dados} (AED) e desenvolver raciocínio crítico sobre correlações e padrões financeiros.

---

\section{Perfil Socioeconômico e Comportamento Financeiro}

\subsection*{Desafios Principais}
\begin{itemize}
  \item Qual é a \textbf{distribuição dos salários} por escolaridade e sexo?
  \item O salário varia de forma significativa entre diferentes \textbf{faixas etárias}?
  \item Pessoas com maior escolaridade possuem \textbf{rendimentos mais altos}?
  \item Existe diferença salarial entre os \textbf{gêneros} em cada nível de escolaridade?
  \item Há \textbf{valores salariais atípicos} (outliers) em determinados grupos?
\end{itemize}

\subsection*{Visualizações Recomendadas}
Boxplot, barras empilhadas, gráficos de linha por faixa etária e histogramas comparativos.

---

\section{Contas e Saldos Bancários}

\subsection*{Desafios Principais}
\begin{itemize}
  \item Quais são os \textbf{tipos de conta} mais comuns?
  \item Quantas contas foram abertas por \textbf{ano ou mês}?
  \item Quais agências concentram os \textbf{maiores saldos médios}?
  \item Quais clientes apresentam \textbf{maior saldo total}?
  \item Clientes mais antigos tendem a ter \textbf{saldos mais elevados}?
\end{itemize}

\subsection*{Visualizações Recomendadas}
Gráficos de barras, linhas temporais, mapas de calor e pareto de clientes por saldo.

---

\section{Cartões e Crédito Pessoal}

\subsection*{Desafios Principais}
\begin{itemize}
  \item Qual é o \textbf{limite médio} por bandeira e status do cartão?
  \item O \textbf{score de crédito} aumenta proporcionalmente ao salário?
  \item Qual o percentual de cartões \textbf{ativos, bloqueados e cancelados}?
  \item Há relação entre o \textbf{limite do cartão} e a \textbf{renda mensal}?
  \item Como se distribuem as faixas de \textbf{score} por faixa salarial?
\end{itemize}

\subsection*{Visualizações Recomendadas}
Gráficos de dispersão (score × salário), barras agrupadas, pizza de status e histogramas de score.

---

\section{Transações e Padrões de Consumo}

\subsection*{Desafios Principais}
\begin{itemize}
  \item Quais categorias possuem o \textbf{maior volume de gastos}?
  \item Quais canais (App, POS, Caixa, Internet Banking) são mais utilizados?
  \item Qual é a \textbf{tendência de gastos e receitas} ao longo do tempo?
  \item Há sazonalidade em determinados meses (ex: dezembro, férias)?
  \item O cliente gasta mais do que recebe? Qual o \textbf{saldo líquido médio}?
\end{itemize}

\subsection*{Visualizações Recomendadas}
Gráficos de linha temporal, barras horizontais, gráficos de área e mapas de calor por categoria.

---

\section{Relações Integradas Entre Tabelas}

\subsection*{Desafios Principais}
\begin{itemize}
  \item Clientes com salários maiores possuem \textbf{limites de crédito mais altos}?
  \item Pessoas com maior saldo bancário realizam \textbf{mais transações positivas}?
  \item Há relação entre o \textbf{tempo de abertura da conta} e o volume de transações?
  \item Pessoas com score baixo realizam mais \textbf{transações negativas}?
  \item A idade influencia o \textbf{score de crédito}?
\end{itemize}

\subsection*{Visualizações Recomendadas}
Dispersões integradas, heatmaps de correlação e dashboards com filtros cruzados.

---

\section{Indicadores e Painel Gerencial}

\subsection*{KPIs Sugeridos}
\begin{itemize}
  \item �� \textbf{Média salarial}: \texttt{AVG(salario\_capped)}
  \item �� \textbf{Saldo médio por conta}: \texttt{AVG(saldo\_atual)}
  \item �� \textbf{Limite médio de crédito}: \texttt{AVG(limite)}
  \item �� \textbf{Score médio}: \texttt{AVG(score\_credito)}
  \item �� \textbf{Volume líquido de transações}: \texttt{SUM(valor)}
  \item �� \textbf{Top categorias de despesa}: \texttt{ABS(SUM(valor))}
\end{itemize}

\subsection*{Painel Executivo Integrado}
Monte um \textbf{dashboard no Tableau} combinando:
\begin{enumerate}
  \item Perfil do Cliente
  \item Contas e Saldos
  \item Crédito e Cartões
  \item Transações e Consumo
\end{enumerate}

Adicione filtros de período, sexo, escolaridade e tipo de conta para facilitar a análise interativa.

---

\section{Desafios de Storytelling Analítico}

\subsection*{Interpretações Esperadas}
\begin{itemize}
  \item Pessoas com pós-graduação têm salários até \textbf{45\% maiores}, mas gastam mais em lazer.
  \item O canal \textbf{App} concentra cerca de 65\% das transações totais.
  \item Clientes com score acima de 750 possuem \textbf{limite 2,5× maior}.
  \item Os saldos médios cresceram aproximadamente 18\% entre 2022 e 2025.
  \item A categoria \textbf{Mercado} representa 28\% das despesas agregadas.
\end{itemize}

Esses insights podem ser apresentados como uma \textbf{história no Tableau}, usando o recurso \textit{Story Points} para guiar a narrativa visual.

\end{appendices}



% --------------------------------------------

\chapter{Tipos de Gráficos e Escolhas Adequadas} % Capítulo 6

\section{Princípios de Visualização}
\begin{itemize}[leftmargin=*]
  \item \textbf{Correspondência gráfico–variável}: nominal/ordinal $\rightarrow$ barras; contínua $\rightarrow$ hist/linha; relação $\rightarrow$ dispersão.
  \item \textbf{Escalas e eixos}: zero em eixos de barras; usar log quando necessário.
  \item \textbf{Cor e contraste}: paletas perceptualmente uniformes; evitar excesso de categorias coloridas.
\end{itemize}

\section{Univariados}
\begin{itemize}[leftmargin=*]
  \item \textbf{Barras}: contagens/categorias.
  \item \textbf{Histograma \& KDE}: distribuição contínua; escolha de \textit{bins}.
  \item \textbf{Boxplot/Violin}: mediana, quartis, outliers.
\end{itemize}

\section{Bivariados e Multivariados}
\begin{itemize}[leftmargin=*]
  \item \textbf{Dispersão (scatter)} com cor/tamanho por terceira variável.
  \item \textbf{Mapa de calor (heatmap)} para correlações ou matrizes.
  \item \textbf{Line + Bar (dual-axis)}: tendências com volume (atenção ao \textit{scale sync}).
  \item \textbf{Tree Map \& Sunburst}: composição; cuidado com interpretações de área.
\end{itemize}

\section{Mapas e Geocodificação}
\begin{itemize}[leftmargin=*]
  \item \textbf{Geo-roles}: País/Estado/Cidade; correção de ambiguidades.
  \item \textbf{Camadas}: pontos, polígonos, densidade.
  \item \textbf{Performance}: agregação por nível de zoom; filtros de contexto geográficos.
\end{itemize}

\section{Customização e Performance Visual}
\begin{itemize}[leftmargin=*]
  \item \textbf{Parâmetros}: alternar métricas, metas, categorizações.
  \item \textbf{Cálculos de Tabela}: \% do total, \textit{running sum}, difs.
  \item \textbf{LOD}: métricas por cliente/produto independentes da exibição.
\end{itemize}

\section{Exercício Resolvido — \textit{Dashboard} Comparativo (��)}
\begin{SolvedBox}
\textbf{Problema:} Construir um \textit{dashboard} que compare crescimento de vendas por Categoria e Subcategoria no tempo, com seletor de métrica (Vendas, Lucro, \%Lucro).

\textbf{Solução:}
\begin{enumerate}[leftmargin=*]
  \item \textbf{Parâmetro} \texttt{p\_métrica} + \textbf{Campo} \texttt{m\_din = CASE [p\_métrica] ... END}.
  \item \textbf{Sheet 1}: Linha temporal por Categoria; \textbf{Sheet 2}: Barras por Subcategoria; \textbf{Filtros} sincronizados.
  \item \textbf{Ações}: filtrar Subcategoria ao clicar na Categoria (interação).
  \item \textbf{Performance}: extrato, filtro de contexto por ano, limitar Subcategorias top-N (parâmetro).
\end{enumerate}
\textbf{Comentário}: separa visão macro (Categoria) e micro (Subcategoria) mantendo consistência de métrica com parâmetro único.
\end{SolvedBox}



% --------------------------------------------

\chapter{Insights com Tableau e Python} % Capítulo 7

\section{O que é Insight}
\begin{itemize}[leftmargin=*]
  \item \textbf{Tipos}: descritivo, diagnóstico, preditivo, prescritivo.
  \item \textbf{Regras}: relevância, evidência, acionabilidade, clareza.
\end{itemize}

\section{Geração de Insights em Python}
\begin{itemize}[leftmargin=*]
  \item \textbf{EDA automatizada}: \texttt{pandas-profiling}/\texttt{ydata-profiling} (resumos); seleção manual para o Tableau.
  \item \textbf{Correlação e regressão}: exportar coeficientes e \textit{scores} para o Tableau (CSV/SQL).
  \item \textbf{Feature Store simples}: tabelas derivadas com métricas agregadas (por cliente, por mês).
\end{itemize}

\section{Integrando no Tableau}
\begin{itemize}[leftmargin=*]
  \item \textbf{TabPy}: \texttt{SCRIPT\_REAL} para previsões curtas; \textit{scoring} de modelos (já treinados).
  \item \textbf{Estratégia preferida}: pré-processar no Python, consumir no Tableau via \textbf{extracts} para melhor performance.
\end{itemize}

\section{Casos Reais}
\begin{itemize}[leftmargin=*]
  \item \textbf{Educação}: horas de estudo $\rightarrow$ probabilidade de nota $\ge$ 8 (logística em Python; visual no Tableau).
  \item \textbf{Varejo}: desconto $\times$ margem; \textit{uplift} por campanha.
  \item \textbf{Saúde}: adesão a tratamento por faixa etária e região.
\end{itemize}

\section{Exercício Resolvido — Pipeline Insight (��)}
\begin{SolvedBox}
\textbf{Problema:} Identificar fatores que mais explicam variação de vendas mensais por loja.

\textbf{Solução:}
\begin{enumerate}[leftmargin=*]
  \item \textbf{Python}: regressão múltipla (ou árvore) com features (sazonalidade, promoções, clima); exportar \texttt{coef\_importances.csv}.
  \item \textbf{Tableau}: conectar ao CSV; barras ordenadas das importâncias; \textit{tooltip} com sinal/valor.
  \item \textbf{Narrativa}: destaque de 3 fatores principais + sugestão de ação (ex.: reduzir desconto em categorias com baixo \textit{uplift}).
\end{enumerate}
\textbf{Performance}: cálculos pesados no Python; Tableau recebe só o resultado agregado.
\end{SolvedBox}



% --------------------------------------------

\chapter{Storytelling e Criação de Dashboards/Stories} % Capítulo 8

\section{Fundamentos de Storytelling com Dados}
\begin{itemize}[leftmargin=*]
  \item \textbf{Estrutura}: contexto $\rightarrow$ conflito $\rightarrow$ conclusão (modelo DataStory).
  \item \textbf{POV (ponto de vista)}: público-alvo, pergunta orientadora, \textit{call to action}.
\end{itemize}

\section{Design de Dashboards}
\begin{itemize}[leftmargin=*]
  \item \textbf{Layout}: grade, hierarquia visual, alinhamento, tipografia mínima.
  \item \textbf{Objetos}: contêineres (horizontal/vertical), \textit{legends}, filtros, \textit{highlights}.
  \item \textbf{Ações}: \textit{highlight}, \textit{filter}, \textit{URL}, navegação \textit{go-to-sheet}.
\end{itemize}

\section{Criando Stories}
\begin{itemize}[leftmargin=*]
  \item \textbf{Story points}: sequenciamento de mensagens.
  \item \textbf{Templates}: comparativo, evolução temporal, antes/depois.
\end{itemize}

\section{Publicação e Performance}
\begin{itemize}[leftmargin=*]
  \item \textbf{Otimização}: extratos, LODs fixas para métricas base, filtros de contexto, evitar \textit{blends}.
  \item \textbf{Publicação no Tableau Public}: título, descrição, \textit{tags}, atualização de dados.
  \item \textbf{Privacidade}: anonimização, agregação mínima, remoção de PII.
\end{itemize}

\section{Exercício Resolvido — Storyboard Interativo (��)}
\begin{SolvedBox}
\textbf{Problema:} Construir um \textit{story} sobre desempenho acadêmico, partindo de visão geral (turma) até alunos em risco.

\textbf{Solução:}
\begin{enumerate}[leftmargin=*]
  \item \textbf{Slide 1}: KPI geral de média e dispersão (boxplot).
  \item \textbf{Slide 2}: heatmap por disciplina$\times$bimestre.
  \item \textbf{Slide 3}: lista de alunos com \texttt{Parâmetro de Risco} (LOD FIXED por aluno).
  \item \textbf{Ações}: filtros encadeados; \textit{tooltip} narrativo (o porquê do risco).
\end{enumerate}
\textbf{Performance}: consolidar métricas por aluno/tempo em tabela derivada para reduzir custo na renderização.
\end{SolvedBox}


% ============================
% PARTE II — VISUALIZAÇÃO E STORYTELLING COM TABLEAU
% ============================


% --------------------------------------------

\chapter{Projeto Final: Tableau Public Challenge} % Capítulo 9

\section{Objetivo e Entregáveis}
Cada grupo publicará um projeto no \textbf{Tableau Public} com \textit{dashboard} + \textit{story} contendo:
\begin{itemize}[leftmargin=*]
  \item Tema (sorteado): educação, finanças, saúde, tecnologia ou esportes.
  \item Fonte de dados real (aberta) ou base disponibilizada pela disciplina.
  \item \textbf{Integração com Python} para pré-processamento/insights.
  \item Documento curto (README) com hipóteses, decisões de design e limitações.
\end{itemize}

\section{Etapas e Cronograma}
\begin{enumerate}[leftmargin=*]
  \item \textbf{Curadoria de dados} (qualidade, dicionário de dados, ética).
  \item \textbf{EDA em Python} (limpeza, métricas, \textit{features} e arquivos derivados).
  \item \textbf{Modelagem no Tableau} (joins/relacionamentos, LODs base).
  \item \textbf{Dashboards/Stories} (narrativa e ações).
  \item \textbf{Publicação} (link público + README).
\end{enumerate}

\section{Rubrica de Avaliação}
\begin{itemize}[leftmargin=*]
  \item \textbf{Clareza e relevância dos insights} (25\%).
  \item \textbf{Qualidade visual e narrativa} (25\%).
  \item \textbf{Correção técnica e performance} (25\%).
  \item \textbf{Originalidade e curadoria de dados} (10\%).
  \item \textbf{Publicação e documentação} (15\%).
\end{itemize}

\section{Exercício Resolvido — Exemplo-Guia de Projeto (��)}
\begin{SolvedBox}
\textbf{Problema:} Construir um projeto sobre evasão escolar com base pública (INEP/IBGE), integrando Python e Tableau.

\textbf{Solução (guia):}
\begin{enumerate}[leftmargin=*]
  \item \textbf{Python}: limpeza, indicadores (taxa de evasão, reprovação, distorção idade-série), exportar \texttt{indicadores\_municipio\_ano.csv}.
  \item \textbf{Tableau}: relacionar \texttt{Município/UF} e Ano; mapas (choropleth) + linha temporal + tabela detalhada.
  \item \textbf{Story}: contexto nacional $\rightarrow$ estados críticos $\rightarrow$ municípios-alvo $\rightarrow$ recomendações.
  \item \textbf{Performance}: extrato por ano; filtros de contexto; LOD FIXED para KPIs municipais; evitar blends.
\end{enumerate}
\textbf{Publicação}: \textit{Tableau Public} com título, resumo e \textit{tags}; link entregue no repositório da disciplina.
\end{SolvedBox}

% Capitulo 10: Analise Exploratoria de Dados com Tableau (versao ampliada e limpa)
% Arquivo sem caracteres especiais para evitar erros de compilacao

\chapter{Resump Tableau e Analise Exploratoria de Dados}
\label{cap:tableau}

\begin{EpigraphBox}
  "Dados sao o novo petroleo quando refinados e visualizados com criterio."\\
  	\textit{-- Clive Humby}
\end{EpigraphBox}

\begin{LearningObjectives}
	\textbf{Ao final deste capitulo, voce sera capaz de:}
\begin{itemize}
  \item Entender o papel do Tableau na analise exploratoria de dados (EDA)
  \item Distinguir Worksheet, Dashboard e Story e usa-los com proposito
  \item Identificar e construir dashboards operacionais, taticos e estrategicos
  \item Definir, calcular e interpretar KPIs relevantes para o negocio
  \item Escolher graficos apropriados para objetivos analiticos
  \item Aplicar medidas de tendencia central e detectar outliers
  \item Interpretar correlacoes e transformar insights em decisoes
\end{itemize}
\end{LearningObjectives}

%-----------------------------------------------------------------------
\section{Introducao: uma conversa com os dados}
%-----------------------------------------------------------------------

Ao abrir um conjunto de dados pela primeira vez, voce esta prestes a iniciar uma conversa. Nessa conversacao, o objetivo e fazer perguntas simples, ouvir os sinais que os dados fornecem e ajustar as perguntas ate que respostas claras surjam. O Tableau funciona como uma ponte entre a curiosidade e a visualizacao: permite fazer perguntas, testar hipoteses e apresentar descobertas de forma visual e interativa.

Neste capitulo vamos caminhar de forma pratica e narrativa. Em cada secao eu apresento um conceito, conto uma pequena historia que ilustra seu uso em contexto real e termino com um exercicio resolvido com contexto, calculos e interpretacao. O formato e pensado para que o capitulo leia como um trecho de livro, com ritmo, transicoes e exemplos aplicaveis.

%-----------------------------------------------------------------------
\section{Tableau: o que e, em poucas palavras}
%-----------------------------------------------------------------------

Tableau e uma plataforma de Business Intelligence focada em visualizacoes interativas e exploracao de dados. Diferente de ferramentas puramente programaticas, o Tableau permite que analistas arrastem e soltem campos, experimentem diferentes tipos de graficos e montem paineis que contam uma historia dos dados.

Pense no Tableau como um laboratorio visual: voce testa rapidamente varias representacoes e encontra, por iteracao, a que melhor comunica determinado insight.

%-----------------------------------------------------------------------
\section{Worksheet, Dashboard e Story: papeis e analogias}
%-----------------------------------------------------------------------

Antes de construir qualquer painel grande, e util entender os papeis:
\begin{itemize}
  \item \textbf{Worksheet:} a unidade basica — um grafico ou tabela que responde a uma pergunta especifica.
  \item \textbf{Dashboard:} um conjunto de worksheets organizadas para que o usuario veja diferentes angulos do mesmo problema.
  \item \textbf{Story:} sequencia de dashboards/worksheets que guia a audiencia por uma narrativa (contexto -> analise -> conclusao).
\end{itemize}

Analogia: pense nas worksheets como cadernos de laboratorio, nos dashboards como relatorios de projeto e nas stories como a apresentacao executiva que voce fara para a diretoria.

\begin{SolvedBox}{Contexto e Questao 1}
	extbf{Contexto:} Uma equipe de analise de vendas recebeu uma solicitacao da diretoria: escolher uma ferramenta para padronizar a entrega de paineis mensais que serao usados por gerentes e analistas. A equipe precisa justificar a escolha com base em usabilidade, flexibilidade e velocidade de criacao.

	\textbf{Pergunta:} Qual definicao descreve melhor o Tableau no contexto de EDA?

	cblower
	\textbf{Resposta:} Tableau e uma ferramenta de Business Intelligence para visualizacoes interativas e paineis dinamicos. Sua facilidade de uso e capacidade de producao rapida de visualizacoes o tornam apropriado para processos de EDA e comunicacao de insights.
\end{SolvedBox}

%-----------------------------------------------------------------------
\section{Tipos de dashboards: quando usar cada um}
%-----------------------------------------------------------------------

Nem todo painel serve para todos os publicos. A definicao do tipo de dashboard deve ser guiada pelo objetivo e pela frequencia de atualizacao:
\begin{itemize}
  \item \textbf{Operacional:} monitoramento em tempo real; decisao rapida; visibilidade imediata.
  \item \textbf{Tatico/Analitico:} foco em diagnostico e descoberta; filtros para exploracao; atualizacoes diarias/semanal.
  \item \textbf{Estrategico:} resumo para decisores; poucos KPIs, tendencias de longo prazo; periodicidade mensal/trimestral.
\end{itemize}

\begin{SolvedBox}{Contexto e Questao 2}
	\textbf{Contexto:} Um diretor hospitalar pediu um painel para a reuniao mensal do comite executivo. Ele precisa comparar ocupacao, custo medio por paciente e indicadores de seguranca para decidir sobre expansao de unidades.

	\textbf{Pergunta:} Que tipo de dashboard atende melhor essa necessidade?

	cblower
	\textbf{Resposta:} Dashboard estrategico. Nessa reuniao o foco e decisao de longo prazo com KPIs agregados; portanto um painel sintetico, com comparacoes periodicas e metas claras, e o formato adequado.
\end{SolvedBox}

%-----------------------------------------------------------------------
\section{KPIs: escolher, calcular e interpretar}
%-----------------------------------------------------------------------

KPIs sao metrics que orientam decisoes. Definir bom KPI implica alinhar a metrica ao objetivo de negocio, garantir dados confiaveis e ter uma meta clara.

\begin{itemize}
  \item Especifico: mede o que importa.
  \item Mensuravel: formula definida e dados disponiveis.
  \item Acionavel: provocar uma acao se estiver fora da meta.
  \item Temporal: definido por periodo (mensal, trimestral etc.).
\end{itemize}

\begin{SolvedBox}{Contexto e Questao 3}
	\textbf{Contexto:} Uma plataforma de cursos online quer avaliar se seus cursos estao sendo realmente efetivos. O time de produto sugere medir "eficacia" com uma unica metrica para comparacoes entre cursos.

	\textbf{Pergunta:} Qual KPI unico e mais indicado para medir a eficacia de um curso?

	cblower
	extbf{Resposta sugerida:} Taxa de conclusao (percentual de alunos matriculados que completam o curso). Isso captura engajamento e finalizacao, que sao proximos de eficacia no contexto educacional. Complementarmente, usar media de notas e NPS para avaliar qualidade e satisfacao.
\end{SolvedBox}

\begin{SolvedBox}{Contexto e Questao 4}
	\textbf{Contexto:} Em uma avaliacao operacional, o time de customer success apresenta dados: 320 clientes ativos, 400 clientes totais e taxa de satisfacao 85\%. Precisamos de um KPI composto para reportar a diretoria.

	\textbf{Pergunta:} Calcule o KPI definido como (ativos/totais) x taxa de satisfacao e interprete.

	cblower
	extbf{Calculo:} (320/400) x 0.85 = 0.8 x 0.85 = 0.68 = 68\%.

	extbf{Interpretacao:} O resultado indica que, combinando retencao e satisfacao, a empresa esta entregando cerca de 68\% do desempenho ideal. A solucao envolve melhorar retencao (reduzir churn) e elevar satisfacao para empurrar o KPI para niveis superiores.
\end{SolvedBox}

%-----------------------------------------------------------------------
\section{Escolha de graficos: um guia pratico}
%-----------------------------------------------------------------------

A escolha do grafico depende do objetivo de comunicacao:
\begin{itemize}
  \item Comparacao entre categorias: barras ordenadas (melhor leitura para ranking).
  \item Evolucao temporal: series de linhas (sazonalidade e tendencia).
  \item Proporcoes: barras empilhadas ou pizza para poucas categorias.
  \item Relacao entre duas variaveis: scatter plot com linha de tendencia.
  \item Analise geografica: mapa coropletico.
\end{itemize}

Um bom grafico responde a uma pergunta clara; se a pergunta mudar, troque o grafico.

%-----------------------------------------------------------------------
\section{Medidas de tendencia central e variabilidade}
%-----------------------------------------------------------------------

Ao resumir conjuntos de dados, escolha a medida que melhor resiste a distorcoes (outliers) e que responde a pergunta de negocio:
\begin{itemize}
  \item Media: util quando a distribuicao e simetrica e sem outliers extremos.
  \item Mediana: mais robusta a outliers; melhor para distribuicoes assimetricas.
  \item Desvio padrao: mede variabilidade em torno da media.
\end{itemize}

\begin{SolvedBox}{Contexto e Questao 5}
	\textbf{Contexto:} Em um painel de desempenho regional, um gerente nota que uma filial teve vendas muito altas (outlier) e pergunta se a media ou mediana explica melhor a situacao.

	\textbf{Dados de exemplo:} vendas (em mil) = [6,7,8,9,10].

	cblower
	\textbf{Resposta:} Media = 8; Mediana = 8. Neste conjunto sem outliers extremos, ambas coincidem. Mas se houvesse um outlier (ex.: 900), a mediana seria mais representativa.
\end{SolvedBox}

%-----------------------------------------------------------------------
\section{Correlacao e interpretacao}
%-----------------------------------------------------------------------

O coeficiente de Pearson r quantifica relacao linear entre duas variaveis. Use uma escala pratica:
\begin{itemize}
  \item 0 < r < 0.3 : fraca
  \item 0.3 <= r < 0.7 : moderada
  \item 0.7 <= r < 1 : forte
\end{itemize}

Sempre acompanhe r de um grafico de dispersao para visualizar padroes e outliers.

\begin{SolvedBox}{Contexto e Questao 6}
	\textbf{Contexto:} Um professor investiga a relacao entre horas de estudo e nota final numa amostra de alunos. O calculo do coeficiente de correlacao retorna r = 0.85.

	\textbf{Pergunta:} Como interpretar esse valor no contexto educativo?

	cblower
	\textbf{Resposta:} r = 0.85 indica forte correlacao positiva: alunos que estudam mais tendem a obter notas maiores. Importante: correlacao nao prova causalidade. Fatores externos (qualidade do estudo, apoio, motivacao) tambem influenciam.
\end{SolvedBox}

%-----------------------------------------------------------------------
\section{Fluxo pratico: construir um dashboard passo a passo}
%-----------------------------------------------------------------------

Siga este roteiro quando for montar um dashboard no Tableau:
\begin{enumerate}
  \item Defina a pergunta de negocio e o publico-alvo.
  \item Escolha 3 a 5 KPIs que respondam diretamente a essa pergunta.
  \item Crie worksheets focadas: uma pergunta por worksheet.
  \item Combine em um dashboard: KPIs no topo, graficos principais no centro, filtros e controles na lateral.
  \item Configure acoes (filtros, destaques) para navegacao fluida.
  \item Teste com usuarios reais e refine a hierarquia visual e o desempenho.
\end{enumerate}

\begin{SolvedBox}{Contexto e Questao 7}
	\textbf{Contexto:} Voce foi contratado para montar um dashboard de e-commerce. O gerente quer um painel que permita identificar rapidamente produtos com queda de vendas, regiao com performance abaixo da media e trackers de conversao.

	\textbf{Pergunta:} Qual o papel da worksheet e do dashboard nesse projeto?

	\cblower
	\textbf{Resposta:} A worksheet responde a uma pergunta especifica (ex.: vendas mensais por produto). O dashboard integra varias worksheets (vendas mensais, top produtos, mapa por regiao, taxa de conversao) para que o gerente navegue e tome decisoes a partir de uma visao consolidada.
\end{SolvedBox}

%-----------------------------------------------------------------------
\section{Boas praticas, desempenho e qualidade}
%-----------------------------------------------------------------------

Principios basicos:
\begin{itemize}
  \item Menos e mais: priorize clareza sobre abundancia de graficos.
  \item Hierarquia visual: destaque KPIs e mensure o contraste entre elementos.
  \item Consistencia: padronize cores para significados (verde = bom, vermelho = alerta).
  \item Contexto: sempre informe periodo, fonte e data de atualizacao.
\end{itemize}

Armadilhas comuns:
\begin{itemize}
  \item Ignorar outliers sem investigar.
  \item Usar graficos esteticamente agradaveis porem enganosos.
  \item Interpretar correlacao como causalidade.
  \item Nao validar dados antes de publicar o dashboard.
\end{itemize}

\begin{ChecklistBox}{Checklist rapido antes de publicar}
\begin{itemize}
  \item[$\\square$] Objetivo claro
  \item[$\\square$] Audiencia definida
  \item[$\\square$] KPIs relevantes e calculos validados
  \item[$\\square$] Graficos apropriados e interpretaveis
  \item[$\\square$] Dados validados e atualizados
  \item[$\\square$] Performance aceitavel (tempo de carga)
\end{itemize}
\end{ChecklistBox}

%-----------------------------------------------------------------------
\section*{Apendice: formulas rapidas}
%-----------------------------------------------------------------------

\begin{FormulaBox}{Media aritmetica}
\[
\bar{x} = \frac{\sum_{i=1}^{n} x_i}{n}
\]
\end{FormulaBox}

\begin{FormulaBox}{Desvio padrao amostral}
\[
 s = \sqrt{\frac{\sum_{i=1}^{n} (x_i - \bar{x})^2}{n-1}}
\]
\end{FormulaBox}

\begin{FormulaBox}{Correlacao de Pearson}
\[
 r = \frac{\sum_{i=1}^{n} (x_i - \bar{x})(y_i - \bar{y})}{\sqrt{\sum_{i=1}^{n} (x_i - \bar{x})^2} \sqrt{\sum_{i=1}^{n} (y_i - \bar{y})^2}}
\]
\end{FormulaBox}

\begin{SummaryBox}{Resumo do capitulo}
Este capitulo apresentou uma abordagem pratica e narrativa para usar o Tableau na Analise Exploratoria de Dados: definicoes, tipos de dashboards, selecao de graficos, KPI, estatistica basica, correlacao e um fluxo passo-a-passo para construir paineis efetivos. Os exercicios resolvidos mostram como aplicar conceitos em contextos reais.
\end{SummaryBox}

% Fim do arquivo


\appendix
\chapter*{Referências}
\addcontentsline{toc}{chapter}{Referências}
\begin{itemize}
  \item Tukey, John W. \textit{Exploratory Data Analysis}. 1977.
  \item Bruce, Peter; Bruce, Andrew; Gedeck, Peter. \textit{Estatística Prática para Cientistas de Dados}. O'Reilly.
  \item Fávero, Luiz Paulo. \textit{Manual de Análise de Dados}. Elsevier.
\end{itemize}

\end{document}
