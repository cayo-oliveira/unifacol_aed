\chapter{Análise Multivariada}

\section{Visão Geral}
A análise multivariada considera três ou mais variáveis simultaneamente, permitindo estudar relações complexas entre fatores.  
Exemplos práticos incluem: \textbf{matriz de correlação}, \textbf{regressão múltipla}, \textbf{análise de componentes principais (PCA)}, \textbf{clusterização} e \textbf{árvores de decisão}.  

Neste capítulo, vamos focar nos conceitos fundamentais de \textbf{probabilidade, probabilidade condicional, odds e odds ratio}, essenciais para responder perguntas do tipo:  
- Quais fatores mais influenciam a chance de um aluno alcançar nota alta?  
- Qual grupo apresenta maior risco ou chance relativa?  

---

\section{Probabilidade}
\subsection*{Definição}
A probabilidade mede a chance de ocorrência de um evento em um espaço de possibilidades.

\begin{FormulaBox}
Probabilidade: 
\[
P(A) = \frac{\text{número de casos favoráveis}}{\text{número total de casos}}
\]
\end{FormulaBox}

\subsection*{Exemplo}
Em uma turma de 20 alunos, 8 obtiveram nota $\ge 8$.  
\[
P(\text{Alta}) = \frac{8}{20} = 0,4 = 40\%
\]

---

\section{Probabilidade Condicional}
\subsection*{Definição}
A probabilidade condicional mede a chance de um evento ocorrer dado que outro já ocorreu.

\begin{FormulaBox}
Probabilidade condicional:
\[
P(A \mid B) = \frac{P(A \cap B)}{P(B)}
\]
\end{FormulaBox}

\subsection*{Exemplo}
Se entre 10 alunos que estudaram mais de 5 horas, 7 tiveram nota $\ge 8$:  
\[
P(\text{Alta} \mid \text{Horas}>5) = \frac{7}{10} = 70\%
\]

---

\section{Odds}
\subsection*{Definição}
Enquanto a probabilidade compara sucessos com o total de casos, as \textbf{odds} comparam sucessos com fracassos.

\begin{FormulaBox}
\[
\mathrm{Odds}(A) = \frac{P(A)}{1-P(A)}
\]
\end{FormulaBox}

\subsection*{Exemplo}
Se $P(\text{Alta})=0,4$:  
\[
\mathrm{Odds}(\text{Alta}) = \frac{0,4}{1-0,4} = \frac{0,4}{0,6} \approx 0,67
\]  
Ou seja, para cada 10 alunos, a razão esperada é $\approx 4$ com Alta contra $\approx 6$ sem Alta.

---

\section{Odds Ratio (OR)}
\subsection*{Definição}
O Odds Ratio compara duas razões de chances, mostrando quanto mais provável é um evento em um grupo em relação a outro.

\begin{FormulaBox}
\[
\mathrm{OR}_{A:B} = \frac{\mathrm{Odds}(A)}{\mathrm{Odds}(B)}
\]
\end{FormulaBox}

\subsection*{Exemplo}
- Grupo A: $P(\text{Alta})=0,7 \Rightarrow \mathrm{Odds}_A=\tfrac{0,7}{0,3}=2,33$  
- Grupo B: $P(\text{Alta})=0,4 \Rightarrow \mathrm{Odds}_B=\tfrac{0,4}{0,6}=0,67$  

\[
\mathrm{OR}_{A:B} = \frac{2,33}{0,67} \approx 3,5
\]

Interpretação: alunos do grupo A têm cerca de 3,5 vezes mais chance de Alta em comparação ao grupo B.

---

\section{Como identificar características que mais influenciam uma variável-alvo com Odds Ratio}
\begin{itemize}[leftmargin=*]
  \item Passo 1: Defina a variável-alvo binária (ex.: Nota Alta = sim/não).  
  \item Passo 2: Separe os dados em grupos (ex.: faixas de horas de estudo).  
  \item Passo 3: Calcule as probabilidades de sucesso em cada grupo.  
  \item Passo 4: Converta em Odds.  
  \item Passo 5: Compare com um grupo de referência via OR.  
\end{itemize}

\textbf{Exemplo prático:}  
Se para $\ge 8$ horas de estudo o OR = 4, e para $< 4$ horas o OR = 0,5, concluímos que estudar mais tempo aumenta significativamente a chance de nota Alta.

---
\clearpage
\section{Problemas Resolvidos — Odds e Odds Ratio}

\subsection*{Exemplo 1 — Horas de estudo e notas}


\begin{SolvedBox}
\textbf{Contexto:} A classificação \emph{Nota Alta/Baixa} segue as seguintes regras fornecidas: \\
\begin{tabular}{@{}c|c|c@{}}\toprule
Horas de estudo & Nota & Nota Alta/Baixa \\\midrule
$<4$   & $<6$   & Baixa \\
$<4$   & $\ge 6$& Baixa \\
$5$--$7$ & $<6$   & Alta \\
$5$--$7$ & $\ge 6$& Alta \\
$\ge 8$  & $<6$   & Alta \\
$\ge 8$  & $\ge 6$& Alta \\\bottomrule
\end{tabular}

\medskip
Para tornar o cálculo concreto, suponha \textbf{10 observações em cada linha} (total 60). Agregando por \emph{faixa de horas}:

\begin{center}
\begin{tabular}{@{}c|c|c|c@{}}\toprule
Grupo de Horas & Alta & Baixa & Total \\\midrule
$<4$     & 0  & 20 & 20 \\
$5$--$7$ & 20 & 0  & 20 \\
$\ge 8$  & 20 & 0  & 20 \\\bottomrule
\end{tabular}
\end{center}

\textbf{Passo 1 — Probabilidades de Alta por grupo}
\[
P(\text{Alta}\mid <4)=\tfrac{0}{20}=0,\quad
P(\text{Alta}\mid 5\text{--}7)=\tfrac{20}{20}=1,\quad
P(\text{Alta}\mid \ge 8)=\tfrac{20}{20}=1.
\]

\textbf{Passo 2 — \emph{Odds} de Alta por grupo (com e sem correção)}
\begin{FormulaBox}
\textbf{Odds (sem correção):}\;\; \mathrm{Odds}=\dfrac{\text{Alta}}{\text{Baixa}}
\end{FormulaBox}

\begin{FormulaBox}
\textbf{Odds (Haldane–Anscombe):}\;\; \mathrm{Odds}_{0{,}5}=\dfrac{\text{Alta}+0{,}5}{\text{Baixa}+0{,}5}
\end{FormulaBox}

\begin{itemize}[leftmargin=*]
\item $<4$: $\mathrm{Odds}=\tfrac{0}{20}=0$ (indefinida para OR); \quad
      $\mathrm{Odds}_{0{,}5}=\tfrac{0{,}5}{20{,}5}\approx 0{,}0244$.
\item $5$--$7$: $\mathrm{Odds}=\tfrac{20}{0}=\infty$; \quad
      $\mathrm{Odds}_{0{,}5}=\tfrac{20{,}5}{0{,}5}=41{,}0$.
\item $\ge 8$: $\mathrm{Odds}=\tfrac{20}{0}=\infty$; \quad
      $\mathrm{Odds}_{0{,}5}=41{,}0$.
\end{itemize}
\end{SolvedBox}



\begin{SolvedBox}
\textbf{Passo 3 — Odds Ratio (OR) usando o grupo $\ge 8$ como referência}

\begin{FormulaBox}
\textbf{Odds Ratio:}\;\; \mathrm{OR}_{A:\text{ref}}=\dfrac{\mathrm{Odds}(A)}{\mathrm{Odds}(\text{ref})}
\end{FormulaBox}

\[
\mathrm{OR}_{<4:\ge 8}=\frac{0{,}0244}{41{,}0}\approx 5{,}95\times 10^{-4},
\qquad
\mathrm{OR}_{5\text{--}7:\ge 8}=\frac{41{,}0}{41{,}0}=1{,}00.
\]

\textbf{Interpretação:}
\begin{itemize}[leftmargin=*]
\item Estudar \textbf{$<4$ horas} está associado a \emph{odds} de Alta \textbf{quase nulas} comparadas a $\ge 8$ horas.
\item Os grupos \textbf{$5$--$7$} e \textbf{$\ge 8$} têm \emph{odds} praticamente idênticas (OR $\approx 1$) neste cenário.
\item O uso da correção $+0{,}5$ é fundamental quando há \textbf{zeros} nas células, evitando divisões por zero e permitindo comparar grupos via OR.
\end{itemize}
\end{SolvedBox}



% -----------------------------------------------------

\subsection*{Exemplo 2 — Exercício físico e pressão arterial}

\begin{SolvedBox}
\textbf{Contexto:} Pesquisadores dividiram pessoas em dois grupos: praticam exercício físico regularmente (Sim) e não praticam (Não). O desfecho observado foi \emph{pressão arterial controlada (Sim ou Não)}.

\textbf{Tabela de frequências:}  

\begin{center}
\begin{tabular}{@{}c|c|c@{}}\toprule
Grupo & Pressão Controlada & Não Controlada \\\midrule
Exercício (Sim) & 18 & 12 \\
Exercício (Não) & 10 & 20 \\\bottomrule
\end{tabular}
\end{center}

\textbf{Passo 1 — Probabilidades}  
- Exercício (Sim): $P(\text{Controlada}) = 18/30 = 0,6$  
- Exercício (Não): $P(\text{Controlada}) = 10/30 \approx 0,33$

\textbf{Passo 2 — Odds}  
- Exercício (Sim): $\mathrm{Odds} = 18/12 = 1,5$  
- Exercício (Não): $\mathrm{Odds} = 10/20 = 0,5$

\textbf{Passo 3 — OR (referência = Não)}  
\[
\mathrm{OR} = \frac{1,5}{0,5} = 3
\]

\textbf{Conclusão:} Pessoas que praticam exercício têm 3 vezes mais chance de controlar a pressão arterial do que as sedentárias.
\end{SolvedBox}

% -----------------------------------------------------

\subsection*{Exemplo 3 — Uso de aplicativo de estudo e aprovação}

\begin{SolvedBox}
\textbf{Contexto:} Estudantes foram separados pelo uso de um aplicativo de estudo (Sim/Não). O desfecho foi aprovação na disciplina.

\textbf{Tabela de frequências:}  

\begin{center}
\begin{tabular}{@{}c|c|c@{}}\toprule
Grupo & Aprovado & Reprovado \\\midrule
App (Sim)  & 45 & 15 \\
App (Não)  & 20 & 25 \\\bottomrule
\end{tabular}
\end{center}

\textbf{Passo 1 — Probabilidades}  
- App (Sim): $P(\text{Aprovado}) = 45/60 = 0,75$  
- App (Não): $P(\text{Aprovado}) = 20/45 \approx 0,44$

\textbf{Passo 2 — Odds}  
- App (Sim): $\mathrm{Odds} = 45/15 = 3,0$  
- App (Não): $\mathrm{Odds} = 20/25 = 0,8$

\textbf{Passo 3 — OR (referência = Não)}  
\[
\mathrm{OR} = \frac{3,0}{0,8} = 3,75
\]

\textbf{Conclusão:} Estudantes que usam o aplicativo têm quase 4 vezes mais chance de aprovação do que os que não usam.
\end{SolvedBox}

% -----------------------------------------------------

\subsection*{Exemplo 4 — Consumo de frutas e obesidade}

\begin{SolvedBox}
\textbf{Contexto:} Em um estudo, indivíduos foram separados em dois grupos: consumo diário de frutas (Sim/Não). O desfecho observado foi obesidade (Sim/Não).

\textbf{Tabela de frequências:}  

\begin{center}
\begin{tabular}{@{}c|c|c@{}}\toprule
Grupo & Obeso & Não Obeso \\\midrule
Frutas (Sim)  & 12 & 48 \\
Frutas (Não)  & 30 & 30 \\\bottomrule
\end{tabular}
\end{center}

\textbf{Passo 1 — Probabilidades}  
- Frutas (Sim): $P(\text{Obeso}) = 12/60 = 0,2$  
- Frutas (Não): $P(\text{Obeso}) = 30/60 = 0,5$

\textbf{Passo 2 — Odds}  
- Frutas (Sim): $\mathrm{Odds} = 12/48 = 0,25$  
- Frutas (Não): $\mathrm{Odds} = 30/30 = 1,0$

\textbf{Passo 3 — OR (referência = Não)}  
\[
\mathrm{OR} = \frac{0,25}{1,0} = 0,25
\]

\textbf{Conclusão:} Consumir frutas diariamente reduz a chance de obesidade (odds 4 vezes menores).
\end{SolvedBox}

% -----------------------------------------------------

\subsection*{Exemplo 5 — Tabagismo e doença respiratória}

\begin{SolvedBox}
\textbf{Contexto:} Uma pesquisa analisou fumantes e não fumantes, verificando a ocorrência de doenças respiratórias.

\textbf{Tabela de frequências:}  

\begin{center}
\begin{tabular}{@{}c|c|c@{}}\toprule
Grupo & Doente & Saudável \\\midrule
Fumante     & 40 & 60 \\
Não Fumante & 20 & 80 \\\bottomrule
\end{tabular}
\end{center}

\textbf{Passo 1 — Probabilidades}  
- Fumante: $P(\text{Doente}) = 40/100 = 0,4$  
- Não Fumante: $P(\text{Doente}) = 20/100 = 0,2$

\textbf{Passo 2 — Odds}  
- Fumante: $\mathrm{Odds} = 40/60 = 0,67$  
- Não Fumante: $\mathrm{Odds} = 20/80 = 0,25$

\textbf{Passo 3 — OR (referência = Não)}  
\[
\mathrm{OR} = \frac{0,67}{0,25} \approx 2,68
\]

\textbf{Conclusão:} Fumantes têm quase 3 vezes mais chance de desenvolver doença respiratória em comparação com não fumantes.
\end{SolvedBox}

% -----------------------------------------------------

\subsection*{Exemplo 6 — Horas e notas}

\begin{SolvedBox}

\textbf{Contexto:} Uma turma foi dividida em 3 grupos segundo as horas de estudo: $<4$, $5$–$7$ e $\ge 8$. Foi registrada a nota final (Alta $\ge 8$ ou Baixa $<8$).  

\textbf{Tabela de frequências (dados simulados)}  

\begin{center}
\begin{tabular}{@{}c|c|c@{}}\toprule
Grupo de Horas & Nota Alta & Nota Baixa \\\midrule
$<4$       & 2 & 8 \\
$5$–$7$    & 5 & 5 \\
$\ge 8$    & 7 & 3 \\\bottomrule
\end{tabular}
\end{center}

\textbf{Passo 1 — Probabilidade condicional (ex.: Alta dado $5$–$7$)}  
\[
P(\text{Alta}\mid 5\text{–}7) = \frac{5}{10} = 0,5
\]

\textbf{Passo 2 — Cálculo das Odds}  
- Grupo $<4$: $\mathrm{Odds}=\frac{2}{8}=0,25$  
- Grupo $5$–$7$: $\mathrm{Odds}=\frac{5}{5}=1,0$  
- Grupo $\ge 8$: $\mathrm{Odds}=\frac{7}{3}\approx 2,33$

\textbf{Passo 3 — Cálculo dos OR (referência: $\ge 8$)}  
\[
\mathrm{OR}_{<4:\ge 8} = \frac{0,25}{2,33}\approx 0,11
\]
\[
\mathrm{OR}_{5\text{–}7:\ge 8} = \frac{1,0}{2,33}\approx 0,43
\]

\textbf{Conclusão:}  
- O grupo $\ge 8$ é o que mais favorece nota Alta.  
- Quanto menor o tempo de estudo, menor a chance de nota Alta em relação ao grupo de referência.
\end{SolvedBox}


---

\section*{Problemas Sugeridos — Análise Multivariada}
\begin{enumerate}[leftmargin=*]
  \item Em uma pesquisa com 100 pacientes, 40 melhoraram após tratamento (Alta) e 60 não. Calcule as probabilidades, odds e OR entre os grupos Tratamento e Placebo.  
  \item Considere 3 grupos de estudantes (pouco, médio, muito estudo). Monte tabelas $2\times 2$ e calcule os OR relativos ao grupo de muito estudo.  
  \item Em um jogo de basquete, 30 de 50 arremessos foram convertidos. Calcule $P$, Odds e interprete.  
  \item Explique a diferença entre $P(\text{Vitória}\mid \text{Casa})$ e $P(\text{Casa}\mid \text{Vitória})$ em uma temporada esportiva.  
  \item Se em uma pesquisa eleitoral 55\% votariam no candidato A, calcule a probabilidade, odds e interprete.  
  \item Monte um exemplo em que a probabilidade de sucesso é a mesma em dois grupos, mas o OR é diferente (discuta por quê).  
  \item Explique por que o OR é usado em regressão logística como medida de efeito.  
  \item Em um estudo de saúde, fumo $\times$ doença respiratória apresentaram OR=4. Interprete.  
  \item Calcule e compare Odds e OR em um campeonato de futebol entre times de alto e baixo investimento.  
  \item Discuta o impacto de outliers (jogadores extremos) nos cálculos de probabilidade e OR.
\end{enumerate}

---

\section*{Apêndice do Capítulo 4 — Odds e Apostas Online}

\begin{ProofBox}
\textbf{Odds nas apostas esportivas:}  
Nas casas de apostas, as \emph{odds} representam a chance atribuída a um evento.  

- Se $P(\text{Vitória})=0,5$, a odd justa seria $2,0$ (pagamento de 2 para cada 1 apostado).  
- Odds baixas (ex.: 1,2) indicam favorito; odds altas (ex.: 5,0) indicam azarão.

\textbf{Odds Ratio nas apostas:}  
Pode-se comparar o risco relativo entre dois resultados:  
- Ex.: OR entre “Time A vencer” e “Time B vencer” mostra qual lado tem mais chance relativa.  

\textbf{Prática:}  
Casas de apostas usam margens (ou “taxa da casa”) que reduzem a odd real para garantir lucro.  
Na AED e na estatística, o OR serve como medida análoga, mas aplicada em contextos de saúde, educação, marketing e ciências sociais.
\end{ProofBox}
