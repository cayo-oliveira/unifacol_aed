\chapter{Análise Bivariada}

\section{Introdução}
A \textbf{análise bivariada} é o estudo de duas variáveis ao mesmo tempo, com o objetivo de verificar a existência de associação, tendência ou dependência entre elas.  

Perguntas típicas que a análise bivariada ajuda a responder:
\begin{itemize}[leftmargin=*]
  \item Existe relação entre as horas de estudo e o desempenho em provas?
  \item A renda de um consumidor influencia seu gasto mensal?
  \item A temperatura está relacionada às vendas de sorvete?
\end{itemize}

\section{Tabelas de Contingência e Gráficos de Dispersão}
Antes das medidas numéricas, a exploração inicial pode ser feita por:
\begin{itemize}[leftmargin=*]
  \item \textbf{Tabelas de contingência}: aplicáveis quando as variáveis são qualitativas (ex.: gênero $\times$ preferência de produto).
  \item \textbf{Gráficos de dispersão}: aplicáveis quando as variáveis são quantitativas, permitindo observar padrões de associação (linearidade, tendência positiva/negativa, presença de outliers).
\end{itemize}

\section{Covariância}
\subsection*{Definição}
A \textbf{covariância} mede o grau em que duas variáveis variam conjuntamente.  
- Se valores altos de $X$ estão associados a valores altos de $Y$, a covariância tende a ser positiva.  
- Se valores altos de $X$ estão associados a valores baixos de $Y$, a covariância tende a ser negativa.  
- Se não há relação linear clara, a covariância tende a valores próximos de zero.

\begin{FormulaBox}
\[
\mathrm{Cov}(X,Y)=\dfrac{1}{n-1}\sum_{i=1}^n (x_i-\bar{x})(y_i-\bar{y})
\]
\end{FormulaBox}

\subsection*{Interpretação}
- $\mathrm{Cov}(X,Y)>0$: associação linear positiva.  
- $\mathrm{Cov}(X,Y)<0$: associação linear negativa.  
- $\mathrm{Cov}(X,Y)\approx 0$: ausência de relação linear clara.

\subsection*{Exemplo}
Considere as variáveis \emph{Horas de estudo} ($X$) e \emph{Nota final} ($Y$):

\begin{center}
\begin{tabular}{@{}c|c|c|c|c@{}}\toprule
$x_i$ & $y_i$ & $(x_i-\bar{x})$ & $(y_i-\bar{y})$ & Produto $(x_i-\bar{x})(y_i-\bar{y})$ \\\midrule
8 & 8  & 3   & 2,6  & 7,8 \\
4 & 6  & -1  & 0,6  & -0,6 \\
2 & 2  & -3  & -3,4 & 10,2 \\
1 & 1  & -4  & -4,4 & 17,6 \\
10& 10 & 5   & 4,6  & 23,0 \\\midrule
\multicolumn{4}{r|}{Soma} & \textbf{58,0} \\\bottomrule
\end{tabular}
\end{center}

\[
\mathrm{Cov}(X,Y)=\frac{58}{5-1}=14,5
\]

A covariância positiva indica que quanto mais horas de estudo, maior tende a ser a nota.

\subsection*{Limitações}
A covariância não é normalizada: o valor depende da unidade de medida das variáveis.  
Por exemplo, medir horas em minutos aumentaria numericamente a covariância sem mudar a relação.  
Por isso, usamos a \textbf{correlação} como medida padronizada.

\section{Correlação de Pearson}
\subsection*{Definição}
A \textbf{correlação de Pearson} é a medida mais usada para quantificar a força e a direção da relação linear entre duas variáveis quantitativas.  
Ela é obtida pela normalização da covariância.

\begin{FormulaBox}
\[
r = \dfrac{\mathrm{Cov}(X,Y)}{s_X s_Y}, \quad r\in[-1,1]
\]
\end{FormulaBox}

\subsection*{Interpretação dos valores de $r$}
\begin{itemize}[leftmargin=*]
  \item $r \approx 1$: correlação linear positiva muito forte.
  \item $r \approx -1$: correlação linear negativa muito forte.
  \item $r \approx 0$: ausência de relação linear clara.
\end{itemize}

\subsection*{Exemplo (continuação do anterior)}
Sabendo que $s_X=\sqrt{15}\approx 3,873$ e $s_Y=\sqrt{14,8}\approx 3,847$:

\[
r = \frac{14,5}{(3,873)(3,847)} \approx 0,973
\]

Isso indica uma \textbf{forte correlação positiva}: mais horas de estudo estão fortemente associadas a notas mais altas.

\subsection*{Observações importantes}
- Correlação mede \emph{associação}, não \emph{causalidade}.  
- Relações não lineares podem ter correlação próxima de zero, mesmo havendo forte associação (ex.: curva em forma de U).  
- A presença de outliers pode distorcer o valor de $r$.

\section{Resumo Comparativo}
\begin{itemize}[leftmargin=*]
  \item \textbf{Covariância}: indica direção da relação, mas depende da escala.  
  \item \textbf{Correlação}: padroniza a covariância, facilitando a interpretação ($-1 \leq r \leq 1$).  
\end{itemize}


\section{Problema Resolvido — Relação entre horas de estudo e notas}

\begin{SolvedBox}
\textbf{Contexto.} Foram registradas \emph{Horas de estudo} ($x$) e \emph{Notas finais} ($y$) de 5 estudantes. Queremos calcular a \textbf{covariância} e a \textbf{correlação de Pearson} entre as variáveis.

\textbf{Passo 1 — Tabela de dados}\\[-6pt]
\begin{center}
\begin{tabular}{@{}lccccc@{}}\toprule
Registro & R1 & R2 & R3 & R4 & R5 \\\midrule
Horas ($x$) & 8 & 4 & 2 & 1 & 10 \\
Notas ($y$) & 8 & 6 & 2 & 1 & 10 \\\bottomrule
\end{tabular}
\end{center}

\textbf{Passo 2 — Médias}\\
\[
\bar{x} = \frac{25}{5} = 5, 
\qquad 
\bar{y} = \frac{27}{5} = 5{,}4
\]

\textbf{Passo 3 — Tabela para produtos centrados}

\begin{center}
\begin{tabular}{@{}c|c|c|c|c@{}}\toprule
$x_i$ & $y_i$ & $(x_i-\bar{x})$ & $(y_i-\bar{y})$ & Produto $(x_i-\bar{x})(y_i-\bar{y})$ \\\midrule
8  & 8  & 3   & 2,6  & 7,8  \\
4  & 6  & -1  & 0,6  & -0,6 \\
2  & 2  & -3  & -3,4 & 10,2 \\
1  & 1  & -4  & -4,4 & 17,6 \\
10 & 10 & 5   & 4,6  & 23,0 \\\midrule
\textbf{Totais} & --- & --- & --- & \textbf{58,0} \\\bottomrule
\end{tabular}
\end{center}

\textbf{Passo 4 — Covariância}

\begin{FormulaBox}
$\mathrm{Cov}(X,Y) = \dfrac{\sum (x_i-\bar{x})(y_i-\bar{y})}{n-1}$
\end{FormulaBox}

\[
\mathrm{Cov}(X,Y) = \frac{58}{4} = 14,5
\]

\textbf{Passo 5 — Correlação de Pearson}

\begin{FormulaBox}
$r = \dfrac{\mathrm{Cov}(X,Y)}{s_X s_Y}$
\end{FormulaBox}

Sabemos: $s_X=\sqrt{15}\approx 3,873$, \quad $s_Y=\sqrt{14,8}\approx 3,847$.

\[
r = \frac{14,5}{(3,873)(3,847)} \approx \frac{14,5}{14,92} \approx 0,973
\]

\textbf{Conclusão:} $r \approx 0,97$, indicando \textbf{forte correlação linear positiva} entre horas de estudo e nota.
\end{SolvedBox}

\section*{Problemas Sugeridos — Análise Bivariada}

\begin{enumerate}[leftmargin=*]
  \item Calcule a covariância e a correlação entre os pares $(x,y)=\{(1,2),(2,3),(3,5),(4,4)\}$.  
  \item Um conjunto de dados registra altura (cm) e peso (kg) de 5 pessoas. Estime a correlação.  
  \item Construa um gráfico de dispersão para as horas de sono $\{5,6,7,8,9\}$ e desempenho $\{60,65,70,80,85\}$. Descreva a tendência.  
  \item Em uma empresa, vendas (mil R\$) e gastos em propaganda (mil R\$) foram:
  $$\{(10,3),(15,5),(20,7),(30,9)\}$$. Calcule $r$.  
  \item Mostre que se todas as observações de $y$ forem iguais, a correlação é indefinida.  
  \item Para as notas de matemática e português de 6 alunos, construa a tabela de produtos centrados e estime $r$.  
  \item Explique com exemplo quando $r$ próximo de zero não significa ausência de relação.  
  \item Compare os gráficos de dispersão de correlações $r=0,9$, $r=0$, e $r=-0,9$. Explique a diferença.  
  \item Uma base tem pares $(x,y)=\{(2,4),(4,8),(6,12),(8,16)\}$. Calcule $r$ e interprete.  
  \item Explique, com um exemplo de mercado, por que “correlação não implica causalidade”.
\end{enumerate}

\section*{Apêndice do Capítulo 3 — Demonstrações}

\begin{ProofBox}
\textbf{Por que a correlação é adimensional?}\\[4pt]
A covariância depende das unidades de medida de $X$ e $Y$.  
Ao dividir por $s_X s_Y$, normalizamos essa medida, garantindo $r \in [-1,1]$, sem unidade.  

\textbf{Valores de referência:}\\
$r \approx 1$: relação linear positiva forte.\\
$r \approx -1$: relação linear negativa forte.\\
$r \approx 0$: ausência de relação linear (mas podem existir relações não lineares).
\end{ProofBox}
