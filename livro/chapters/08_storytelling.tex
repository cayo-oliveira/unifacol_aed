
% --------------------------------------------

\chapter{Storytelling e Criação de Dashboards/Stories} % Capítulo 8

\section{Fundamentos de Storytelling com Dados}
\begin{itemize}[leftmargin=*]
  \item \textbf{Estrutura}: contexto $\rightarrow$ conflito $\rightarrow$ conclusão (modelo DataStory).
  \item \textbf{POV (ponto de vista)}: público-alvo, pergunta orientadora, \textit{call to action}.
\end{itemize}

\section{Design de Dashboards}
\begin{itemize}[leftmargin=*]
  \item \textbf{Layout}: grade, hierarquia visual, alinhamento, tipografia mínima.
  \item \textbf{Objetos}: contêineres (horizontal/vertical), \textit{legends}, filtros, \textit{highlights}.
  \item \textbf{Ações}: \textit{highlight}, \textit{filter}, \textit{URL}, navegação \textit{go-to-sheet}.
\end{itemize}

\section{Criando Stories}
\begin{itemize}[leftmargin=*]
  \item \textbf{Story points}: sequenciamento de mensagens.
  \item \textbf{Templates}: comparativo, evolução temporal, antes/depois.
\end{itemize}

\section{Publicação e Performance}
\begin{itemize}[leftmargin=*]
  \item \textbf{Otimização}: extratos, LODs fixas para métricas base, filtros de contexto, evitar \textit{blends}.
  \item \textbf{Publicação no Tableau Public}: título, descrição, \textit{tags}, atualização de dados.
  \item \textbf{Privacidade}: anonimização, agregação mínima, remoção de PII.
\end{itemize}

\section{Exercício Resolvido — Storyboard Interativo (��)}
\begin{SolvedBox}
\textbf{Problema:} Construir um \textit{story} sobre desempenho acadêmico, partindo de visão geral (turma) até alunos em risco.

\textbf{Solução:}
\begin{enumerate}[leftmargin=*]
  \item \textbf{Slide 1}: KPI geral de média e dispersão (boxplot).
  \item \textbf{Slide 2}: heatmap por disciplina$\times$bimestre.
  \item \textbf{Slide 3}: lista de alunos com \texttt{Parâmetro de Risco} (LOD FIXED por aluno).
  \item \textbf{Ações}: filtros encadeados; \textit{tooltip} narrativo (o porquê do risco).
\end{enumerate}
\textbf{Performance}: consolidar métricas por aluno/tempo em tabela derivada para reduzir custo na renderização.
\end{SolvedBox}

